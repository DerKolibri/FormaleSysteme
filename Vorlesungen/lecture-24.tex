\documentclass[aspectratio=1610,onlymath]{beamer}
% \documentclass[aspectratio=1610,onlymath,handout]{beamer}

% Macros used by all lectures, but not necessarily by excercises

%%% General setup and dependencies:

% \usetheme[ddcfooter,nosectionnum]{tud}
\usetheme[nosectionnum,pagenum,noheader]{tud}
% \usetheme[nosectionnum,pagenum]{tud}

% Increase body font size to a sane level:
\let\origframetitle\frametitle
% \renewcommand{\frametitle}[1]{\origframetitle{#1}\normalsize}
\renewcommand{\frametitle}[1]{\origframetitle{#1}\fontsize{10pt}{13.2}\selectfont}
\setbeamerfont{itemize/enumerate subbody}{size=\small} % tud defaults to scriptsize!
\setbeamerfont{itemize/enumerate subsubbody}{size=\small}
% \setbeamerfont{normal text}{size=\small}
% \setbeamerfont{itemize body}{size=\small}

\renewcommand{\emph}[1]{\textbf{#1}}

\def\arraystretch{1.3}% Make tables even less cramped vertically

\usepackage[ngerman]{babel}
\usepackage[utf8]{inputenc}
\usepackage[T1]{fontenc}

%\usepackage{graphicx}
\usepackage[export]{adjustbox} % loads graphicx
\usepackage{import}
\usepackage{stmaryrd}
\usepackage[normalem]{ulem} % sout command
% \usepackage{times}
\usepackage{txfonts}

% \usepackage[perpage]{footmisc} % reset footnote counter on each page -- fails with beamer (footnotes gone)
\usepackage{perpage}  % reset footnote counter on each page
\MakePerPage{footnote}

\usepackage{tikz}
\usetikzlibrary{arrows,positioning}
% Inspired by http://www.texample.net/tikz/examples/hand-drawn-lines/
\usetikzlibrary{decorations.pathmorphing}
\pgfdeclaredecoration{penciline}{initial}{
    \state{initial}[width=+\pgfdecoratedinputsegmentremainingdistance,
    auto corner on length=1mm,]{
        \pgfpathcurveto%
        {% From
            \pgfqpoint{\pgfdecoratedinputsegmentremainingdistance}
                      {\pgfdecorationsegmentamplitude}
        }
        {%  Control 1
        \pgfmathrand
        \pgfpointadd{\pgfqpoint{\pgfdecoratedinputsegmentremainingdistance}{0pt}}
                    {\pgfqpoint{-\pgfdecorationsegmentaspect
                     \pgfdecoratedinputsegmentremainingdistance}%
                               {\pgfmathresult\pgfdecorationsegmentamplitude}
                    }
        }
        {%TO 
        \pgfpointadd{\pgfpointdecoratedinputsegmentlast}{\pgfpoint{1pt}{1pt}}
        }
    }
    \state{final}{}
}
\tikzset{handdrawn/.style={decorate,decoration=penciline}}
\tikzset{every shadow/.style={fill=none,shadow xshift=0pt,shadow yshift=0pt}}
% \tikzset{module/.append style={top color=\col,bottom color=\col}}

% Use to make Tikz attributes with Beamer overlays
% http://tex.stackexchange.com/a/6155
\tikzset{onslide/.code args={<#1>#2}{%
  \only<#1| handout:0>{\pgfkeysalso{#2}} 
}}
\tikzset{onslideprint/.code args={<#1>#2}{%
  \only<#1>{\pgfkeysalso{#2}} 
}}

%%% Title -- always set this first

\newcommand{\defineTitle}[3]{
	\newcommand{\lectureindex}{#1}
	\title{Formale Systeme}
	\subtitle{\href{\lectureurl}{#1. Vorlesung: #2}}
	\author{\href{http://korrekt.org/}{Markus Kr\"{o}tzsch}}
%	\author{\href{http://www.sebastian-rudolph.de}{Sebastian Rudolph} in Vertretung von \href{http://korrekt.org/}{Markus Kr\"{o}tzsch}}
	\date{#3}
	\datecity{TU Dresden}
% 	\institute{Computational Logic}
}

%%% Table of contents:

\RequirePackage{ifthen}

\newcommand{\highlight}[2]{%
	\ifthenelse{\equal{#1}{\lectureindex}}{\alert{#2}}{#2}%
}

\def\myspace{-0.7ex}
\newcommand{\printtoc}{
\begin{tabular}{r@{$\quad$}l}
\highlight{1}{1.} & \highlight{1}{Willkommen/Einleitung formale Sprachen}\\[\myspace]
\highlight{2}{2.} & \highlight{2}{Grammatiken und die Chomsky-Hierarchie}\\[\myspace]
\highlight{3}{3.} & \highlight{3}{Endliche Automaten}\\[\myspace]
\highlight{4}{4.} & \highlight{4}{Complexity of FO query answering}\\[\myspace]
\highlight{5}{5.} & \highlight{5}{Conjunctive queries}\\[\myspace]
\highlight{6}{6.} & \highlight{6}{Tree-like conjunctive queries}\\[\myspace]
\highlight{7}{7.} & \highlight{7}{Query optimisation}\\[\myspace]
\highlight{8}{8.} & \highlight{8}{Conjunctive Query Optimisation / First-Order~Expressiveness}\\[\myspace]
\highlight{9}{9.} & \highlight{9}{First-Order~Expressiveness / Introduction to Datalog}\\[\myspace]
\highlight{10}{10.} & \highlight{10}{Expressive Power and Complexity of Datalog}\\[\myspace]
\highlight{11}{11.} & \highlight{11}{Optimisation and Evaluation of Datalog}\\[\myspace]
\highlight{12}{12.} & \highlight{12}{Evaluation of Datalog (2)}\\[\myspace]
\highlight{13}{13.} & \highlight{13}{Graph Databases and Path Queries}\\[\myspace]
\highlight{14}{14.} & \highlight{14}{Outlook: database theory in practice}
\end{tabular}
}

\newcommand{\overviewslide}{%
\begin{frame}\frametitle{Overview}
\printtoc
\medskip

Siehe \href{\lectureurl}{course homepage [$\Rightarrow$ link]} for more information and materials
\end{frame}
}

%%% Colours:

\usepackage{xcolor,colortbl}
\definecolor{redhighlights}{HTML}{FFAA66}
\definecolor{lightblue}{HTML}{55AAFF}
\definecolor{lightred}{HTML}{FF5522}
\definecolor{lightpurple}{HTML}{DD77BB}
\definecolor{lightgreen}{HTML}{55FF55}
\definecolor{darkred}{HTML}{CC4411}
\definecolor{darkblue}{HTML}{176FC0}%{1133AA}
\definecolor{nightblue}{HTML}{2010A0}%{1133AA}
\definecolor{alert}{HTML}{176FC0}
\definecolor{darkgreen}{HTML}{36AB14}
\definecolor{strongyellow}{HTML}{FFE219}
\definecolor{devilscss}{HTML}{666666}

\newcommand{\redalert}[1]{\textcolor{darkred}{#1}}

%%% Style commands

\newcommand{\quoted}[1]{\texttt{"}{#1}\texttt{"}}
\newcommand{\squote}{\texttt{"}} % straight quote
\newcommand{\Sterm}[1]{\ensuremath{\mathtt{\textcolor{purple}{#1}}}}    % letters in alphabets
\newcommand{\Snterm}[1]{\textsf{\textcolor{darkblue}{#1}}} % nonterminal symbols
\newcommand{\Sntermsub}[2]{\Snterm{#1}_{\Snterm{#2}}} % nonterminal symbols
\newcommand{\Slang}[1]{\textbf{\textcolor{black}{#1}}}    % languages
\newcommand{\Slangsub}[2]{\Slang{#1}_{\Slang{#2}}}    % languages
% Code
\newcommand{\Scode}[1]{\textbf{#1}}    % reserved words in program listings, e.g., "if"
\newcommand{\Scodelit}[1]{\textcolor{purple}{#1}}    % literals in program listings, e.g., strings
\newcommand{\Scomment}[1]{\textcolor{gray}{#1}}    % comment in program listings

\newcommand{\epstrastar}{\mathrel{\mathord{\stackrel{\epsilon}{\to}}{}^*}} % transitive reflexive closure of epsilon transitions in an epslion-NFA

\newcommand{\narrowcentering}[1]{\mbox{}\hfill#1\hfill\mbox{}}

\newcommand{\defeq}{\mathrel{:=}}

\newcommand{\Smach}[1]{\ensuremath{\mathcal{#1}}}    % machines

%%% Slide layout commands:

\newcommand{\sectionSlide}[1]{
\frame{\begin{center}
\LARGE
#1
\end{center}}
}
\newcommand{\sectionSlideNoHandout}[1]{
\frame<handout:0>{\begin{center}
\LARGE
#1
\end{center}}
}

\newcommand{\mydualbox}[3]{%
 \begin{minipage}[t]{#1}
 \begin{beamerboxesrounded}[upper=block title,lower=block body,shadow=true]%
    {\centering\usebeamerfont*{block title}#2}%
    \raggedright%
    \usebeamerfont{block body}
%     \small
    #3%
  \end{beamerboxesrounded}
  \end{minipage}
}
% 
\newcommand{\myheaderbox}[2]{%
 \begin{minipage}[t]{#1}
 \begin{beamerboxesrounded}[upper=block title,lower=block title,shadow=true]%
    {\centering\usebeamerfont*{block title}\rule{0pt}{2.6ex} #2}%
  \end{beamerboxesrounded}
  \end{minipage}
}

\newcommand{\mycontentbox}[2]{%
 \begin{minipage}[t]{#1}%
 \begin{beamerboxesrounded}[upper=block body,lower=block body,shadow=true]%
    {\centering\usebeamerfont*{block body}\rule{0pt}{2.6ex}#2}%
  \end{beamerboxesrounded}
  \end{minipage}
}

\newcommand{\mylcontentbox}[2]{%
 \begin{minipage}[t]{#1}%
 \begin{beamerboxesrounded}[upper=block body,lower=block body,shadow=true]%
    {\flushleft\usebeamerfont*{block body}\rule{0pt}{2.6ex}#2}%
  \end{beamerboxesrounded}
  \end{minipage}
}

% label=180:{\rotatebox{90}{{\footnotesize\textcolor{darkgreen}{Beispiel}}}}
% \hspace{-8mm}\ghost{\raisebox{-7mm}{\rotatebox{90}{{\footnotesize\textcolor{darkgreen}{Beispiel}}}}}\hspace{8mm}
\newcommand{\examplebox}[1]{%
	\begin{tikzpicture}[decoration=penciline, decorate]
		\pgfmathsetseed{1235}
		\node (n1) [decorate,draw=darkgreen, fill=darkgreen!10,thick,align=left,text width=\linewidth, inner ysep=2mm, inner xsep=2mm] at (0,0) {#1};
% 		\node (n2) [align=left,text width=\linewidth,inner sep=0mm] at (n1.92) {{\footnotesize\raisebox{3mm}{\textcolor{darkgreen}{Beispiel}}}};
% 		\node (n2) [decorate,draw=darkgreen, fill=darkgreen!10,thick, align=left,text width=\linewidth,inner sep=2mm] at (n1.90) {{\footnotesize\raisebox{0mm}{\textcolor{darkgreen}{Beispiel}}}};
	\end{tikzpicture}%
}%

\newcommand{\codebox}[1]{%
	\begin{tikzpicture}[decoration=penciline, decorate]
		\pgfmathsetseed{1236}
		\node (n1) [decorate,draw=strongyellow, fill=strongyellow!10,thick,align=left,text width=\linewidth, inner ysep=2mm, inner xsep=2mm] at (0,0) {#1};
	\end{tikzpicture}%
}%

\newcommand{\defbox}[1]{%
	\begin{tikzpicture}[decoration=penciline, decorate]
		\pgfmathsetseed{1237}
		\node (n1) [decorate,draw=darkred, fill=darkred!10,thick,align=left,text width=\linewidth, inner ysep=2mm, inner xsep=2mm] at (0,0) {#1};
	\end{tikzpicture}%
}%

\newcommand{\theobox}[1]{%
	\begin{tikzpicture}[decoration=penciline, decorate]
		\pgfmathsetseed{1240}
		\node (n1) [decorate,draw=darkblue, fill=darkblue!10,thick,align=left,text width=\linewidth, inner ysep=2mm, inner xsep=2mm] at (0,0) {#1};
	\end{tikzpicture}%
}%

\newcommand{\anybox}[2]{%
	\begin{tikzpicture}[decoration=penciline, decorate]
		\pgfmathsetseed{1240}
		\node (n1) [decorate,draw=#1, fill=#1!10,thick,align=left,text width=\linewidth, inner ysep=2mm, inner xsep=2mm] at (0,0) {#2};
	\end{tikzpicture}%
}%


\newsavebox{\mybox}%
\newcommand{\doodlebox}[2]{%
\sbox{\mybox}{#2}%
	\begin{tikzpicture}[decoration=penciline, decorate]
		\pgfmathsetseed{1238}
		\node (n1) [decorate,draw=#1, fill=#1!10,thick,align=left,inner sep=1mm] at (0,0) {\usebox{\mybox}};
	\end{tikzpicture}%
}%

% Common notation

\usepackage{amsmath,amssymb,amsfonts}
\usepackage{xspace}

\newcommand{\lectureurl}{https://iccl.inf.tu-dresden.de/web/FS2016}

\DeclareMathAlphabet{\mathsc}{OT1}{cmr}{m}{sc} % Let's have \mathsc since the slide style has no working \textsc

% Dual of "phantom": make a text that is visible but intangible
\newcommand{\ghost}[1]{\raisebox{0pt}[0pt][0pt]{\makebox[0pt][l]{#1}}}

\newcommand{\tuple}[1]{\langle{#1}\rangle}

%%% Annotation %%%

\usepackage{color}
\newcommand{\todo}[1]{{\tiny\color{red}\textbf{TODO: #1}}}



%%% Old macros below; move when needed

\newcommand{\blank}{\text{\textvisiblespace}} % empty tape cell for TM

% table syntax
\newcommand{\dom}{\textbf{dom}}
\newcommand{\adom}{\textbf{adom}}
\newcommand{\dbconst}[1]{\texttt{"#1"}}
\newcommand{\pred}[1]{\textsf{#1}}
\newcommand{\foquery}[2]{#2[#1]}
\newcommand{\ground}[1]{\textsf{ground}(#1)}
% \newcommand{\foquery}[2]{\{#1\mid #2\}} %% Notation as used in Alice Book
% \newcommand{\foquery}[2]{\tuple{#1\mid #2}}

\newcommand{\quantor}{\mathord{\reflectbox{$\text{\sf{Q}}$}}} % the generic quantor

% logic syntax
\newcommand{\Inter}{\mathcal{I}} %used to denote an interpretation
\newcommand{\Jnter}{\mathcal{J}} %used to denote another interpretation
\newcommand{\Knter}{\mathcal{K}} %used to denote yet another interpretation
\newcommand{\Zuweisung}{\mathcal{Z}} %used to denote a variable assignment

% query languages
\newcommand{\qlang}[1]{{\sf #1}} % Font for query languages
\newcommand{\qmaps}[1]{\textbf{QM}({\sf #1})} % Set of query mappings for a query language

%%% Complexities %%%

\hyphenation{Exp-Time} % prevent "Ex-PTime" (see, e.g. Tobies'01, Glimm'07 ;-)
\hyphenation{NExp-Time} % better that than something else

% \newcommand{\complclass}[1]{{\sc #1}\xspace} % font for complexity classes
\newcommand{\complclass}[1]{\ensuremath{\mathsc{#1}}\xspace} % font for complexity classes

\newcommand{\ACzero}{\complclass{AC$_0$}}
\newcommand{\LogSpace}{\complclass{L}}
\newcommand{\NLogSpace}{\complclass{NL}}
\newcommand{\PTime}{\complclass{P}}
\newcommand{\NP}{\complclass{NP}}
\newcommand{\coNP}{\complclass{coNP}}
\newcommand{\PH}{\complclass{PH}}
\newcommand{\PSpace}{\complclass{PSpace}}
\newcommand{\NPSpace}{\complclass{NPSpace}}
\newcommand{\ExpTime}{\complclass{ExpTime}}
\newcommand{\NExpTime}{\complclass{NExpTime}}
\newcommand{\ExpSpace}{\complclass{ExpSpace}}
\newcommand{\TwoExpTime}{\complclass{2ExpTime}}
\newcommand{\NTwoExpTime}{\complclass{N2ExpTime}}
\newcommand{\ThreeExpTime}{\complclass{3ExpTime}}
\newcommand{\kExpTime}[1]{\complclass{#1ExpTime}}
\newcommand{\kExpSpace}[1]{\complclass{#1ExpSpace}}


\defineTitle{24}{Horn-Logik und Komplexitätstheorie}{28. Januar 2021}

\begin{document}

\maketitle

\sectionSlideNoHandout{Rückblick}

\begin{frame}\frametitle{Logisches Schließen}

\emph{Allgemeine Fragestellungen des logischen Schließens}
\begin{itemize}
\item Logische Folgerung: Gilt $\mathcal{F}\models G$?
\item Allgemeingültigkeit: Gilt $\models F$?
\item Unerfüllbarkeit: Gilt $\models \neg F$, d.h. $F\models\bot$?
\end{itemize}\bigskip

\emph{Einige einfache Methoden des logischen Schließens}
\begin{itemize}
\item Wahrheitswertetabelle
\item Erfüllbarkeitstest mit DNF
\item Widerlegbarkeitstest mit KNF
\item Erfüllbarkeitstest mit Resolution
\end{itemize}
$\leadsto$ Alle schlimmstenfalls exponentiell und oft zu ineffizient

\end{frame}

\sectionSlide{Horn-Logik}

\begin{frame}\frametitle{Horn-Logik}

Bisher waren alle unsere Ansätze für aussagenlogisches Schließen exponentiell -- geht es auch einfacher?
\bigskip\pause

\emph{Idee:} Beschränke die Form von Formeln
\bigskip

\defbox{Eine \redalert{Horn-Klausel}${}^*$ ist eine Klausel, die höchstens ein nichtnegiertes Literal enthält.
Eine \redalert{Horn-Formel} ist eine Formel in KNF, welche nur Horn-Klauseln enthält.\\
{\tiny ${}^*$) nach Alfred Horn, 1918--2001, US-amerikanischer Mathematiker}}

\pause
\examplebox{\emph{Beispiele:}
\begin{itemize}
\item $\neg p\vee \neg q \vee r$ ist eine Horn-Klausel
\item $q\vee p$ ist keine Horn-Klausel
\item $p$ ist eine Horn-Klausel
\item $\neg p\vee \neg q$ ist eine Horn-Klausel
\item $p\vee p$ ist keine Horn-Klausel (aber äquivalent zu einer)
\end{itemize}
}

\end{frame}

\begin{frame}\frametitle{Horn-Klauseln als Implikationen}

Jede Horn-Klausel kann als eine Implikation ohne Negation und Disjunktion ausgedrückt werden
\bigskip

\examplebox{\emph{Beispiele:}
\begin{align*}
\neg p\vee \neg q \vee r \qquad \equiv &&(p\wedge q) &\to r\\
p \qquad \equiv &&\top&\to p\\
\neg p\vee \neg q \qquad \equiv &&(p\wedge q)&\to \bot
\end{align*}
}

\begin{itemize}
\item Wir verwenden $\top$ für die leere Prämisse und $\bot$ für die leere Konsequenz (weil so die gewünschte logische Äquivalenz gilt)
\item Als Implikationen geschriebene Horn-Klauseln werden oft als \redalert{(Horn-)Regeln} bezeichnet
\item Es ist üblich, die Klammern der Konjunktion in der Prämisse wegzulassen
\end{itemize}

\end{frame}

% \begin{frame}\frametitle{Ausblick}
% 
% \emph{Nächste Vorlesung:} Wir werden logisches Schließen für Horn-Regeln noch etwas genauer betrachten
% \begin{itemize}
% \item Resolution kann in diesem Fall spezialisiert werden
% \item Dadurch erhält man polynomielle Algorithmen
% \end{itemize}
% 
% \end{frame}


% \begin{frame}\frametitle{Rückblick: Horn-Logik}
% 
% \defbox{Eine \redalert{Horn-Klausel}${}^*$ ist eine Klausel, die höchstens ein nichtnegiertes Literal enthält.
% Eine \redalert{Horn-Formel} ist eine Formel in KNF, welche nur Horn-Klauseln enthält.\\
% {\tiny ${}^*$) nach Alfred Horn, 1918--2001, US-amerikanischer Mathematiker}}
% 
% Man kann Horn-Formeln als Implikationen auffassen \\
% (sogenannte \redalert{Horn-Regeln}):
% 
% \examplebox{Beispiele:\vspace{-4ex}
% \begin{align*}
% \neg p\vee \neg q \vee r \qquad \equiv &&(p\wedge q) &\to r\\
% p \qquad \equiv &&\top&\to p\\
% \neg p\vee \neg q \qquad \equiv &&(p\wedge q)&\to \bot
% \end{align*}
% }
% 
% \begin{itemize}
% \item Wir verwenden $\top$ für die leere Prämisse und $\bot$ für die leere Konsequenz (weil so die gewünschte logische Äquivalenz gilt)
% % \item Als Implikationen geschriebene Horn-Klauseln werden oft als \redalert{(Horn-)Regeln} bezeichnet
% \item Es ist üblich, die Klammern der Konjunktion in der Prämisse wegzulassen
% \end{itemize}
% 
% \end{frame}

\begin{frame}\frametitle{Beispiel}

Viele einfache rekursive Algorithmen können in Horn-Aussagenlogik beschrieben werden.
\bigskip

\examplebox{Beispiel: Berechnung der Variablen $V_\epsilon$ einer CFG, welche in $\epsilon$ umgeschrieben
werden können (vgl. Vorlesung 2, Folie 32). Wir betrachten die folgende Menge $\mathcal{F}$ von Horn-Regeln:
\begin{align*}
\top & \to p_{\Snterm{A}} & \text{für jede Grammatikregel $\Snterm{A}\to\epsilon$}\\
p_{\Sntermsub{A}{1}}\wedge\ldots\wedge p_{\Sntermsub{A}{n}} & \to p_{\Snterm{B}} & \text{für jede Grammatikregel $\Snterm{B}\to\Sntermsub{A}{1}\cdots\Sntermsub{A}{n}$}
\end{align*}
Dann gilt $\mathcal{F}\models p_{\Snterm{A}}$ genau dann wenn $\Snterm{A}\in V_\epsilon$.
}

\end{frame}


\begin{frame}\frametitle{Resolution bei Horn-Klauseln}

Die Resolution von Horn-Klauseln = Verknüpfung von Regeln:
%
\[\frac{p_1\wedge\ldots\wedge p_n\to \redalert{q}\qquad\redalert{q}\wedge q_1\wedge\ldots\wedge q_m\to r}{p_1\wedge\ldots\wedge p_n\wedge q_1\wedge\ldots\wedge q_m\to r}\]\pause

Man kann zeigen (ohne Beweis):

\theobox{\emph{Satz:} Das Resolutionskalkül für Horn-Formeln bleibt auch dann vollständig, wenn man sich auf Resolventen
beschränkt, bei denen eine Klausel die Form $\{p\}$ (also $\top\to p$) hat.
}

Diese Einschränkung entspricht der Entfernung von bereits erfüllten Vorbedingungen aus der Prämisse einer Regel:
\[\frac{\top\to\redalert{q}\qquad\redalert{q}\wedge q_1\wedge\ldots\wedge q_m\to r}{q_1\wedge\ldots\wedge q_m\to r}\]

\end{frame}

\begin{frame}\frametitle{Hyperresolution = Anwendung von Regeln}

\emph{Es ist leicht zu sehen:} Im eingeschränkten Resolutionskalkül für Horn-Formeln kann eine Resolvente aus einer Regel $q_1\wedge\ldots\wedge q_m\to r$
nur dann zur Ableitung von $\bot$ nützlich sein, wenn letztlich alle Vorbedingungen $q_1,\ldots, q_m$ erfüllt sind.

$\leadsto$ Wir können die Resolution aufschieben, bis dies gegeben ist
\bigskip\pause

\defbox{%
\redalert{Hyperresolution:} Mehrere Klauseln mit positiven Literalen werden parallel mit einer Klausel mit vielen negativen Literalen resolviert.
%
\[\frac{\top\to q_1\qquad\ldots\qquad\top\to q_m\qquad q_1\wedge\ldots\wedge q_m\to r}{\top\to r}\]}
%
Dies entspricht der intuitiven Idee einer \alert{Regelanwendung}: "`Wenn die Regel $q_1\wedge\ldots\wedge q_m\to r$ und die Atome $q_1$, \ldots, $q_m$ gelten, dann kann $r$ abgeleitet werden."'

\end{frame}

\begin{frame}\frametitle{Komplexität des Schließens mit Horn-Formeln}

\emph{Vorteil der Hyperresolution bei Horn-Klauseln:}
\begin{itemize}
\item Jede Resolvente hat die Form $\top\to p$
\item Es gibt nur linear viele solche Resolventen (für Atome $p$, die in der Formel vorkommen)
\end{itemize}
$\leadsto$ Resolution muss nach linear vielen Schritten terminieren
\pause\bigskip

Dies führt zu einer polynomiellen Laufzeit, nicht zu einer linearen (da jeder Resolutionsschritt einige Zeit benötigt).
\medskip

Es geht aber auch noch besser:

\theobox{\emph{Satz (Dowling \& Gallier):} Die Erfüllbarkeit einer Horn-Formel kann in linearer Zeit entschieden werden.}

(Ohne Beweis)
% \bigskip

%%% Folgendes ist im Kontext von NP relevant:
% \emph{Intuition:} Horn-Logik ist die \redalert{deterministische} Form der Aussagenlogik (da Regeln immer genau ein
% Atom implizieren). 

\end{frame}
% 
% \begin{frame}\frametitle{Horn-Logik}
% 
% Vereinfachung des logischen Schließens durch Beschränkung auf Horn-Logik:
% \begin{itemize}
% \item (Hyper-)Resolution liefert polynomiellen Algorithmus
% \item Linearer Algorithmus möglich 
% \end{itemize}\bigskip
% 
% 
% \end{frame}



% \begin{frame}\frametitle{Kompaktheit}
% 
% Eine interessante Eigenschaft der Aussagenlogik folgt aus dem, was wir bereits wissen:
% 
% \theobox{\emph{Satz (Kompaktheit der Aussagenlogik):} Wenn $\mathcal{F}$ eine unendliche Formelmenge ist und $\mathcal{F}\models G$ gilt, dann gibt es eine endliche Teilmenge $\mathcal{F}'\subseteq\mathcal{F}$ mit $\mathcal{F}'\models G$.}
% 
% \emph{Beweis:} Die Vollständigkeit der Resolution 
% 
% \end{frame}

\sectionSlide{Logisches Schließen als Wortproblem}

\begin{frame}\frametitle{Schließen als Entscheidungsproblem}

Wir wissen bereits, dass Schließen ein Entscheidungsproblem ist,\ghost{ z.B.:}

\defbox{\redalert{Erfüllbarkeit}:
\begin{itemize}
\item \emph{Eingabe:} Formel $F$
\item \emph{Ausgabe:} "`ja"' wenn $F$ erfüllbar ist; sonst "`nein."'
\end{itemize}
}

Wie wir wissen, können solche Probleme als Definition einer Sprache angesehen werden:
\[\Slang{SAT} = \{ \textsf{enc}(F)\mid F\text{ ist erfüllbar}\}\]
wobei $\textsf{enc}(F)$ eine (vernünftige) Kodierung der Formel $F$ als Wort über einem endlichen Alphabet (z.B.\ $\{\Sterm{0},\Sterm{1}\}$) ist. Vor allem müssen wir beliebig viele Atome mit nur endlich vielen Zeichen kodieren.
\bigskip\pause

\anybox{purple}{%
\narrowcentering{Was für eine Sprache ist $\Slang{SAT}$?}\\
\narrowcentering{Durch welches Automatenmodell wird sie erkannt?}}

\end{frame}

\begin{frame}[t]\frametitle{Schließen als Sprache (1)}

\anybox{purple}{%
\narrowcentering{Was für eine Sprache ist $\Slang{SAT}$?}\\
\narrowcentering{Durch welches Automatenmodell wird sie erkannt?}}\medskip

Wir wissen, dass $\Slang{SAT}$ entscheidbar ist und daher von Typ 0 ist.\pause\\
Der folgende naive Algorithmus entscheidet $\Slang{SAT}$ auf einer TM:
% Auch für Typ~1 gibt es ein positives Resultat:

\codebox{%Der folgende Algorithmus entscheidet $\Slang{SAT}$ auf einem LBA:
Naiver Erfüllbarkeitstest (Skizze):
\begin{itemize}
\item Wir prüfen systematisch jede Wertzuweisung auf den Atomen der gegebenen Formel
\item Dazu iteriert die TM über alle Wertzuweisungen und berechnet für jede rekursiv den Wahrheitswert der Formel
\item Falls eine erfüllende Zuweisung gefunden wird, dann hält die TM und akzeptiert
\item Falls alle Wertzuweisungen ohne Erfolg durchlaufen worden sind, dann hält die TM und verwirft
\end{itemize}
}

\end{frame}

\newcommand{\vtriple}[3]{\small\begin{pmatrix}#1\\[-1.3ex]#2\\[-1.3ex]\Sterm{#3}\end{pmatrix}}
\newcommand{\vtripletext}[3]{\footnotesize\begin{array}{r}\text{#1}\\[-1.0ex]\text{#2}\\[-1.0ex]\text{#3}\end{array}}

\begin{frame}[t]\frametitle{Schließen als Sprache (2)}

\anybox{purple}{%
\narrowcentering{Was für eine Sprache ist $\Slang{SAT}$?}\\
\narrowcentering{Durch welches Automatenmodell wird sie erkannt?}}\smallskip

Der naive Erfüllbarkeitstest kann in linearem Speicher ablaufen, d.h. auf einem LBA \alert{$\leadsto$ $\Slang{SAT}$ ist von Typ 1}\pause
% Auch für Typ~1 gibt es ein positives Resultat:

\codebox{%
Naive Erfüllbarkeit auf LBA (Skizze):\footnotesize
\begin{itemize}%
\item Wir verwenden ein Arbeitsalphabet mit mehreren "`Spuren"': (1)~Eingabe, (2)~aktuell ermittelte Wahrheitswerte aller Teilformeln, (3)~aktuell getestete Wertzuweisung\vspace{-0.5ex}
\item Skizze der Kodierung (ohne Binärkodierung der Formel):\\[1ex]
\narrowcentering{$
\vtripletext{Spur 3}{Spur 2}{Spur 1}
% 
\vtriple{}{}{(}
\vtriple{}{}{(}
\vtriple{\mytrue}{\mytrue}{p}
\vtriple{}{\myfalse}{\to}
\vtriple{\myfalse}{\myfalse}{q}
\vtriple{}{}{)}
\vtriple{}{\mytrue}{\vee}
\vtriple{}{}{(}
\vtriple{}{\myfalse}{q}
\vtriple{}{\mytrue}{\to}
\vtriple{\myfalse}{\myfalse}{r}
\vtriple{}{}{)}
\vtriple{}{}{)}
$}
\item Man kann nun systematisch Spur 3 iterieren, Spur 2 rekursiv berechnen und den Wert der Gesamtformel prüfen
(machbar, wir verzichten auf die Details)
% \item Dabei ordnen wir jeder Teilformel eine eindeutige Position zu: bei Atomen ist dies das
% erste Zeichen der Kodierung des Atoms, bei zusammengesetzten Formeln das erste Zeichen der Kodierung des Junktors
% \item Der aktuelle Wahrheitswert der Teilformel wir in Spur~2 an ihrer Position gespeichert.
% \item Die aktuelle Wertzuweisung ergibt sich aus den Wahrheitswerten der Atome
% \item Wenn man das erste Vorkommen jedes Atoms in der Formel markiert, dann ist es nicht schwer, die jeweils nöchste Wertzuweisung zu berechnen (durch binäre Addition von $1$).
\end{itemize}\vspace{-3.5ex}~}

\end{frame}

\begin{frame}\frametitle{Welches Berechnungsmodell passt zu $\Slang{SAT}$?}

Erkenntnisse bisher:
\begin{itemize}
\item LBAs sind stark genug für $\Slang{SAT}$ (d.h. $\Slang{SAT}$ ist von Typ 1)
\item $\Slang{SAT}$ ist keine reguläre Sprache, da FAs korrekte Klammerung nicht erkennen können (d.h. $\Slang{SAT}$ ist nicht von Typ 3)
% \item ??? $\Slang{SAT}$ ist keine kontextfreie Sprache (d.h. $\Slang{SAT}$ ist nicht von Typ 2)
\end{itemize}\bigskip\pause

Man kann außerdem zeigen:
\theobox{\narrowcentering{$\Slang{SAT}$ ist nicht kontextfrei.}}
(zumindest wenn eine offensichtliche Kodierung verwendet wird; siehe
\url{http://cstheory.stackexchange.com/questions/37322/is-sat-a-context-free-language})
\bigskip\pause

\anybox{purple}{%
Können wir $\Slang{SAT}$ noch genauer charakterisieren als mit LBAs?}\\\pause
$\leadsto$ Ja, mithilfe anderer Arten von eingeschränkten TMs.

\end{frame}

\sectionSlide{Komplexität}

\begin{frame}\frametitle{Turingmaschinen beschränken}

TMs verwenden zwei Ressourcen, die man beschränken kann:
\begin{itemize}
\item \alert{Speicher:} die Zahl der verwendeten Speicherzellen
\item \alert{Zeit:} die Zahl der durchgeführten Berechnungsschritte
\end{itemize}\pause
Feste Schranken ergeben wenig Sinn (endliche Automaten)\\
\begin{enumerate}[$\leadsto$]
\item Schranken werden als Funktion in der Länge der Eingabe angegeben
\end{enumerate}
\bigskip

\examplebox{\emph{Beispiel:} LBAs beschränken den verfügbaren Speicher auf die Anzahl der Symbole in der Eingabe.
Dies entspricht einer Funktion, welche die Länge $n$ der Eingabe auf den Maximalwert von $n$ Speicherzellen abbildet.}

\end{frame}

\begin{frame}\frametitle{Zur Erinnerung: $O$-Notation}

Die $O$-Notation (mit großem $O$!) charakterisiert Funktionen nach ihrem asymptotischen Verhalten und
versteckt lineare Faktoren.

\defbox{Für Funktionen $f,g: \mathbb{N}\to\mathbb{R}$ schreiben wir \redalert{$f\in O(g)$}
wenn gilt:\\[1ex]
%
\narrowcentering{Es gibt eine Zahl $c>0$ und eine Zahl $n_0\in\mathbb{N}$,}\\
\narrowcentering{so dass für jedes $n>n_0$ gilt: $f(n)\leq c\cdot g(n)$}\\[1ex]
%
Das bedeutet, $f$ wächst nicht wesentlich schneller als $g$.
}\pause

\emph{Notation 1:} Manchmal schreibt man statt $f\in O(g)$ auch $f=O(g)$\\ (allerdings ist $=$ dann eine
asymmetrische Relation).\medskip

\emph{Notation 2:} Oft schreibt man statt $f\in O(g)$ auch $f(n)\in O(g(n))$\\ (bzw. für $f=O(g)$ auch $f(n)=O(g(n))$).

\examplebox{\emph{Beispiele:}
\begin{minipage}[t]{7cm}\vspace{-1.7ex}
\begin{itemize}
\item $(10 n^3+42 n^2 - n + 100)\in O(n^3)$
\item $(2^n + n^{2000})\in O(2^n)$
\item $2^{729}\in O(1)$
\end{itemize}
\end{minipage}
}

\end{frame}

\begin{frame}\frametitle{Schranken für Zeit und Raum}

Die $O$-Notation wird verwendet, um allgemeine Ressourcenschranken für TMs anzugeben.

\defbox{%
Sei $f:\mathbb{N}\to\mathbb{R}$ eine Funktion und $\Smach{M}$ eine Turingmaschine.
\begin{itemize}
\item $\Smach{M}$ heißt \redalert{$O(f)$-zeitbeschränkt} wenn es eine Funktion $g\in O(f)$ gibt, so
dass $\Smach{M}$ für eine beliebige Eingabe $w\in\Sigma^*$ nach maximal $g(|w|)$ Schritten anhält.
%
\item $\Smach{M}$ heißt \redalert{$O(f)$-speicherbeschränkt} wenn es eine Funktion $g\in O(f)$ gibt, so
dass $\Smach{M}$ für eine beliebige Eingabe $w\in\Sigma^*$ hält und zuvor maximal $g(|w|)$ Speicherzellen verwendet.
\end{itemize}
}\pause

\examplebox{\emph{Beispiel:} Ein LBA entspricht einer $O(n)$-speicherbeschränkten \ghost{TM.${}^*$}}

\examplebox{\emph{Beispiel:} Der naive Erfüllbarkeitstest für $\Slang{SAT}$ ist \mbox{$O(2^n)$-zeitbeschränkt}.}

{\tiny ${}^\ast)$ Wobei man die bei LBAs erzwungene Speicherbeschränkung in das Programm der TM einbauen muss.
Unsere LBA-Definition erlaubt keine linearen Faktoren für den Speicherbedarf. Diese haben
keinen Einfluss auf die Rechenstärke von LBAs, da man lineare Speichergewinne durch ein größeres %[-1.5ex]
Arbeitsalphabet simulieren \ghost{kann.}\\
}

\end{frame}

\begin{frame}\frametitle{Zeit und Raum, deterministisch}

Beschränkte TMs können verwendet werden, um viele weitere Sprachklassen zu definieren.
\medskip

\defbox{Sei $f:\mathbb{N}\to\mathbb{R}$ eine Funktion.
\begin{itemize}
\item \redalert{\Scomplclass{DTIME}(f(n))} ist die Klasse aller Sprachen $\Slang{L}$, welche durch eine $O(f)$-zeitbeschränkte Turingmaschine entschieden werden können.
\item \redalert{\Scomplclass{DSPACE}(f(n))} ist die Klasse aller Sprachen $\Slang{L}$, welche durch eine $O(f)$-speicherbeschränkte Turingmaschine entschieden werden können.
\end{itemize}
}\pause

\examplebox{\emph{Beispiel:} $\Slang{SAT}\in\Scomplclass{DTIME}(2^n)$ und $\Slang{SAT}\in\Scomplclass{DSPACE}(n)$.}\pause

\examplebox{\emph{Beispiel:} Das Halteproblem ist in keiner der Klassen $\Scomplclass{DTIME}(f(n))$ oder $\Scomplclass{DSPACE}(f(n))$. Unentscheidbare Probleme benötigen uneingeschränkten Zugang zu beliebig vielen Ressourcen einer TM.}

\end{frame}

\begin{frame}\frametitle{Maschinenmodelle}

Es gibt viele unterschiedliche Versionen von deterministischen Turingmaschinen (z.B. Mehrband-Maschinen).\\

\anybox{purple}{Sind $\Scomplclass{DTIME}(f)$ und $\Scomplclass{DSPACE}(f)$ für jedes TM-Modell gleich?}\pause

\emph{Antwort:} "`Bei vielen kleineren Variationen sind sie es, aber nicht in allen Fällen (oder zumindest ist dies nicht immer bekannt)."'

\examplebox{\emph{Beispiel:} Jede $O(f(n))$-zeitbeschränkte $k$-Band-TM kann durch eine $O(k\cdot f^2(n))$-zeitbeschränkte
1-Band TM simuliert werden (wie in Vorlesung 18 gezeigt). Einfacher gesagt: Der Verzicht auf mehrere Bänder verursacht maximal quadratische Zeitkosten ($k$ ist hier ein linearer Faktor).}\pause

\begin{enumerate}[$\leadsto$]
\item Es ist sinnvoll, \alert{noch allgemeinere Sprachklassen} zu betrachten, die auch gegenüber polynomiellen Änderungen der Ressourcen robust sind
\end{enumerate}

\end{frame}

\begin{frame}\frametitle{Wichtige Komplextitätsklassen}

Die wichtigen deterministischen Komplexitätsklassen fassen jeweils ganze Familien von
zeit- oder speicherbeschränkten Klassen zusammen. Wir erwähnen hier nur die praktisch wichtigsten:

{\footnotesize
\begin{align*}
\Scomplclass{P} = \Scomplclass{PTime} &= \bigcup_{d\geq 1} \Scomplclass{DTime}(n^d)
	& \text{\redalert{polynomielle Zeit}}
%   & \Scomplclass{NP} &= \bigcup_{k\geq 1} \Scomplclass{NTime}(n^k)
  \\[1ex]
\hspace{-1.5cm}\Scomplclass{Exp} = \Scomplclass{ExpTime} &= \bigcup_{d\geq 1} \Scomplclass{DTime}(2^{n^d})
    & \text{\redalert{exponentielle Zeit}${}^*$}
% %   & \Scomplclass{NExp} = \Scomplclass{NExpTime} &= \bigcup_{k\geq 1} \Scomplclass{NTime}(2^{n^k})
%   \\[1ex]
% \hspace{-1.5cm}\Scomplclass{2Exp} = \Scomplclass{2ExpTime} &= \bigcup_{d\geq 1} \Scomplclass{DTime}(2^{2^{n^d}})
% %   & \Scomplclass{N2Exp} = \Scomplclass{N2ExpTime} &= \bigcup_{k\geq 1} \Scomplclass{NTime}(2^{2^{n^k}})
% 	& \text{doppelt-exponentielle Zeit}
%   \\[1ex]
% \Scomplclass{E} = \Scomplclass{ETime} &= \bigcup_{d\geq 1} \Scomplclass{DTime}(2^{dn})
%    & \text{exp.\ Zeit mit linearem Exponent}
   \\[4ex]
\hspace{-1.5cm}\Scomplclass{L} = \Scomplclass{LogSpace} &= \Scomplclass{DSpace}(\log n)
%   & \Scomplclass{NL} = \Scomplclass{NLogSpace} &= \Scomplclass{NSpace}(\log n)
	& \text{\redalert{logarithmischer Speicher}}
  \\[1ex]
\Scomplclass{PSpace} &= \bigcup_{d\geq 1} \Scomplclass{DSpace}(n^d)
	& \text{\redalert{polynomieller Speicher}}
%   \\[1ex]
% \Scomplclass{ExpSpace} &= \bigcup_{d\geq 1} \Scomplclass{DSpace}(2^{n^d})
% 	& \text{exponentieller Speicher}
\end{align*}
}

\bigskip
{\tiny ${}^*$) Anmerkung: Dies ist die praktisch wichtigste Definition von "`exponentieller Zeit"'. Es gibt daneben auch
$\Scomplclass{E} = \Scomplclass{ETime} = \bigcup_{d\geq 1} \Scomplclass{DTime}(2^{dn})$ (exponentielle Zeit mit
linearem Exponenten).\\
}

\end{frame}

\begin{frame}\frametitle{\Scomplclass{LogSpace}? Wie soll das gehen?}

Für $n>1$ gilt \alert{$\log(n)<n$}. Auch beliebige lineare Faktoren können das nur für kleine $n$ kompensieren.\\
\medskip

Eine $O(\log(n))$-speicherbeschränkte TM darf also weniger Speicher verwenden als ihre Eingabe benötigt.
\alert{$\leadsto$ Wie soll das gehen?}
\bigskip\pause

\defbox{Man definiert \redalert{$O(\log(n))$-speicherbeschränkte Turingmaschinen} als besondere Mehrband-TMs:
\begin{itemize}
\item Das erste Band ist das \redalert{Eingabeband}. Es enthält die Eingabe und darf nur gelesen aber nicht beschrieben werden.
\item Das zweite Band ist das \redalert{Arbeitsband}. Es darf beliebig gelesen und beschrieben werden, aber es ist
auf $O(\log(n))$-Speicherzellen beschränkt.
\end{itemize}}

{\footnotesize
Das genügt zur Erkennung von Sprachen. Wenn die TM eine Ausgabe berechnen soll, dann wird dafür ein 
drittes \redalert{Ausgabeband} verwendet, auf dem man beliebig viele Zeichen einmalig schreiben aber nicht lesen kann.\\
}

\end{frame}

\begin{frame}\frametitle{Beziehungen der Komplexitätsklassen}

\alert{Eine wichtige Frage der Komplexitätstheorie ist, was man über die Beziehungen der
Komplexitätsklassen aussagen kann.}
\medskip\pause

Offensichtlich führen (asymptotisch) höhere Ressourcenschranken zu größeren Sprachklassen.
Oft ist aber nicht klar, ob man mit mehr Ressourcen auch wirklich mehr (oder einfach nur gleich viele) Probleme lösen kann. Bei
unseren Klassen ist das aber bekannt:\medskip

\theobox{\emph{Fakt:} Es gilt $\Scomplclass{P}\subsetneq\Scomplclass{Exp}$ und
$\Scomplclass{LogSpace}\subsetneq\Scomplclass{PSpace}$.
}\medskip\pause

Weiterhin kann man Speicher mit Zeit in Beziehung bringen:
\begin{itemize}
\item In Zeit $n$ kann man nur $n$ Speicherzellen nutzen
\item Alle möglichen Konfigurationen auf $n$ Speicherzellen kann man in exponentieller Zeit (bezüglich $n$) berechnen\\[-1ex]
{\tiny (für LBAs haben wir das in Vorlesung 20 besprochen; siehe "`Konfigurationsgraph"')}
\end{itemize}

\theobox{\emph{Fakt:} Es gilt $\Scomplclass{LogSpace}\subseteq\Scomplclass{P}\subseteq\Scomplclass{PSpace}\subseteq\Scomplclass{Exp}$.
}

\end{frame}

\begin{frame}\frametitle{Beispiele}

Unsere Klassen sind recht robust: Details der Implementierung haben oft keinen Einfluss
auf die Einordnung eines Problems.\\
\alert{Oft genügt eine Implementierungsskizze um zu zeigen, dass eine Sprache
in einer dieser Klassen liegt.}\pause

\examplebox{\emph{Beispiel:} Erfüllbarkeit von propositionaler Horn-Logik ist in $\Scomplclass{P}$. Unser Resolutionsalgorithmus
liefert allerdings keinen Hinweis auf Machbarkeit in $\Scomplclass{LogSpace}$.}\pause

\examplebox{\emph{Beispiel:} $\Slang{SAT}$ ist in $\Scomplclass{ExpTime}$ aber auch in $\Scomplclass{PSpace}$.}\pause

\examplebox{\emph{Beispiel:} Das Wortproblem jeder regulären Sprache ist in $\Scomplclass{LogSpace}$. Tatsächlich benötigt ein DFA
gar keinen Speicher.}\pause

\examplebox{\emph{Beispiel:} Die Transformation einer Formel in Negationsnormalform ist in $\Scomplclass{LogSpace}$ möglich. Dies ist kein Spracherkennungsproblem sondern eine Berechnung mit Ein- und Ausgabe.}

\end{frame}

\begin{frame}\frametitle{Ressourcen nichtdeterministischer TMs}

Bei NTMs gibt es viele mögliche Berechnungspfade.\\
$\leadsto$ Welche Pfade meinen wir, wenn wir Ressourcen beschränken?\\\pause
~\hfill{\alert{-- Alle!}}
\medskip

\defbox{%
Sei $f:\mathbb{N}\to\mathbb{R}$ eine Funktion und $\Smach{M}$ eine nichtdeterministische \ghost{TM.}
\begin{itemize}
\item $\Smach{M}$ heißt \redalert{$O(f)$-zeitbeschränkt} wenn es eine Funktion $g\in O(f)$ gibt, so
dass $\Smach{M}$ für eine beliebige Eingabe $w\in\Sigma^*$ \alert{auf jedem Berechnungspfad} nach maximal $g(|w|)$ Schritten anhält.
%
\item $\Smach{M}$ heißt \redalert{$O(f)$-speicherbeschränkt} wenn es eine Funktion $g\in O(f)$ gibt, so
dass $\Smach{M}$ für eine beliebige Eingabe $w\in\Sigma^*$ hält und zuvor \alert{auf jedem Berechnungspfad} maximal $g(|w|)$ Speicherzellen verwendet.
\end{itemize}
}

Eine zeit- oder speicherbeschränkte NTM muss also auch auf erfolglosen Pfaden ("`falsch geratene Übergänge"') garantiert
innerhalb der Ressourcengrenzen halten.

\end{frame}

\begin{frame}\frametitle{Zeit und Raum, nichtdeterministisch}

Die entsprechenden Sprachklassen werden genau wie bei deterministischen TMs definiert:\medskip

\defbox{Sei $f:\mathbb{N}\to\mathbb{R}$ eine Funktion.
\begin{itemize}
\item \redalert{\Scomplclass{NTIME}(f(n))} ist die Klasse aller Sprachen $\Slang{L}$, welche durch eine $O(f)$-zeitbeschränkte NTM entschieden werden können.
\item \redalert{\Scomplclass{NSPACE}(f(n))} ist die Klasse aller Sprachen $\Slang{L}$, welche durch eine $O(f)$-speicherbeschränkte NTM entschieden werden können.
\end{itemize}
}%\pause
% 
% \examplebox{\emph{Beispiel:} $\Slang{SAT}\in\Scomplclass{DTIME}(2^n)$ und $\Slang{SAT}\in\Scomplclass{DSPACE}(n)$.}\pause
% 
% \examplebox{\emph{Beispiel:} Das Halteproblem ist in keiner der Klassen $\Scomplclass{DTIME}(f(n))$ oder $\Scomplclass{DSPACE}(f(n))$. Unentscheidbare Probleme benötigen uneingeschränkten Zugang zu beliebig vielen Ressourcen einer TM.}

\end{frame}


\begin{frame}\frametitle{Nichtdeterministische Komplextitätsklassen}

Auch hier beschränken wir uns auf einige wichtige Fälle:

{\footnotesize
\begin{align*}
\Scomplclass{NP} = \Scomplclass{NPTime} &= \bigcup_{d\geq 1} \Scomplclass{NTime}(n^d)
	& \text{\redalert{nichtdet. polynomielle Zeit}}
%   & \Scomplclass{NP} &= \bigcup_{k\geq 1} \Scomplclass{NTime}(n^k)
  \\[1ex]
\hspace{-1.5cm}\Scomplclass{NExp} = \Scomplclass{NExpTime} &= \bigcup_{d\geq 1} \Scomplclass{NTime}(2^{n^d})
    & \text{\redalert{nichtdet. exponentielle Zeit}}
% %   & \Scomplclass{NExp} = \Scomplclass{NExpTime} &= \bigcup_{k\geq 1} \Scomplclass{NTime}(2^{n^k})
%   \\[1ex]
% \hspace{-1.5cm}\Scomplclass{2Exp} = \Scomplclass{2ExpTime} &= \bigcup_{d\geq 1} \Scomplclass{DTime}(2^{2^{n^d}})
% %   & \Scomplclass{N2Exp} = \Scomplclass{N2ExpTime} &= \bigcup_{k\geq 1} \Scomplclass{NTime}(2^{2^{n^k}})
% 	& \text{doppelt-exponentielle Zeit}
%   \\[1ex]
% \Scomplclass{E} = \Scomplclass{ETime} &= \bigcup_{d\geq 1} \Scomplclass{DTime}(2^{dn})
%    & \text{exp.\ Zeit mit linearem Exponent}
   \\[4ex]
\hspace{-1.5cm}\Scomplclass{NL} = \Scomplclass{NLogSpace} &= \Scomplclass{NSpace}(\log n)
%   & \Scomplclass{NL} = \Scomplclass{NLogSpace} &= \Scomplclass{NSpace}(\log n)
	& \text{\redalert{nichtdet. logarithmischer Speicher}}
  \\[1ex]
\Scomplclass{NPSpace} &= \bigcup_{d\geq 1} \Scomplclass{NSpace}(n^d)
	& \text{\redalert{nichtdet. polynomieller Speicher}}
%   \\[1ex]
% \Scomplclass{ExpSpace} &= \bigcup_{d\geq 1} \Scomplclass{DSpace}(2^{n^d})
% 	& \text{exponentieller Speicher}
\end{align*}
}\pause

\examplebox{\emph{Beispiel:} $\Slang{SAT}\in\Scomplclass{NP}$. Man kann unseren naiven Erfüllbarkeitstest auf einer NTM
so implementieren, dass nur eine (erfüllende) Wertzuweisung geraten wird. Der Wahrheitswert der Formel für diese Zuweisung kann in polynomieller Zeit überprüft werden.
}

\end{frame}

\begin{frame}\frametitle{Deterministisch vs. nichtdeterministisch}

%
Welche Beziehungen haben diese Klassen zu anderen?\pause
{\footnotesize
\begin{itemize}
\item Die Beziehungen zwischen nichtdeterministischen Klassen sind analog zu denen im deterministischen Fall:\\[1ex]
\narrowcentering{$\Scomplclass{NL}\subseteq \Scomplclass{NP}\subseteq \Scomplclass{NPSpace}\subseteq \Scomplclass{NExp}$}
\pause
\item Eine DTM kann als NTM aufgefasst werden, d.h. die nichtdeterministischen Klassen sind immer stärker:\\[1ex]
\narrowcentering{$\Scomplclass{L}\subseteq \Scomplclass{NL}\qquad \Scomplclass{P}\subseteq \Scomplclass{NP}\qquad \Scomplclass{PSpace}\subseteq \Scomplclass{NPSpace}\qquad \Scomplclass{Exp}\subseteq \Scomplclass{NExp}$}
\pause
\item Man kann NTMs mit DTMs simulieren, aber das ist oft mit exponentiellen Mehrkosten verbunden (Vorlesung 19).
\pause
\item Der berühmte \alert{Satz von Savitch} besagt, dass speicherbeschränkte NTMs durch DTMs mit nur quadratischen Mehrkosten simuliert werden können. Insbesondere gilt damit $\Scomplclass{PSpace}= \Scomplclass{NPSpace}$.
\end{itemize}}
% Insgesamt ergeben sich folgende Beziehungen:
Zusammenfassung der wichtigsten bekannten Beziehungen:
\theobox{
\[\Scomplclass{L}\subseteq\Scomplclass{NL}\subseteq \Scomplclass{P}\subseteq\Scomplclass{NP}\subseteq\Scomplclass{PSpace}= \Scomplclass{NPSpace}\subseteq \Scomplclass{Exp}\subseteq \Scomplclass{NExp}\]
\vspace{-2.5ex}
}

\end{frame}

\begin{frame}\frametitle{Die Grenzen unseres Wissens}

Wir wissen:

\theobox{
\[\Scomplclass{L}\subseteq\Scomplclass{NL}\subseteq \Scomplclass{P}\subseteq\Scomplclass{NP}\subseteq\Scomplclass{PSpace}= \Scomplclass{NPSpace}\subseteq \Scomplclass{Exp}\subseteq \Scomplclass{NExp}\]
\vspace{-2.5ex}
}

\begin{itemize}
\item Wir wissen nicht, ob irgendeines dieser $\subseteq$ sogar $\subsetneq$ ist.
\item Insbesondere wissen wir nicht, ob $\Scomplclass{P}\subsetneq\Scomplclass{NP}$ oder $\Scomplclass{P}=\Scomplclass{NP}$.
\item Wir wissen nicht einmal, ob $\Scomplclass{L}\subsetneq\Scomplclass{NP}$ oder $\Scomplclass{L}=\Scomplclass{NP}$.
\end{itemize}\pause
Es wird aber vermutet, dass alle  $\subseteq$ eigentlich $\subsetneq$ sind.
% 
Bekannt ist das aber nur bei exponentiell großen Ressourcenunterschieden:

\theobox{Es gilt:
\begin{itemize}
\item $\Scomplclass{NL}\subsetneq\Scomplclass{PSpace}$
\item $\Scomplclass{P}\subsetneq\Scomplclass{Exp}$
\item $\Scomplclass{NP}\subsetneq\Scomplclass{NExp}$
\end{itemize}}

\end{frame}


\begin{frame}\frametitle{Zusammenfassung und Ausblick}

\redalert{Komplexitätsklassen} sind Mengen von Sprachen, die man (grob) einteilt entsprechend der Ressourcen,
die eine TM zur Entscheidung ihres Wortproblems benötigt.
\bigskip


\redalert{Horn-Logik} erlaubt logisches Schließen in $\Scomplclass{P}$.
\bigskip

Das propositionale Erfüllbarkeitsproblem \redalert{\Slang{SAT} ist in $\Scomplclass{NP}$}.
\bigskip

\anybox{yellow}{
Offene Fragen:
\begin{itemize}
\item Gibt es auch sub-exponentielle Algorithmen für aussagenlogisches Schließen?
\item Gilt $\Scomplclass{P}\neq \Scomplclass{NP}$?
\end{itemize}
}

\end{frame}


\end{document}
