\documentclass[onlymath]{beamer}
% \documentclass[onlymath,handout]{beamer}

% Macros used by all lectures, but not necessarily by excercises

%%% General setup and dependencies:

% \usetheme[ddcfooter,nosectionnum]{tud}
\usetheme[nosectionnum,pagenum,noheader]{tud}
% \usetheme[nosectionnum,pagenum]{tud}

% Increase body font size to a sane level:
\let\origframetitle\frametitle
% \renewcommand{\frametitle}[1]{\origframetitle{#1}\normalsize}
\renewcommand{\frametitle}[1]{\origframetitle{#1}\fontsize{10pt}{13.2}\selectfont}
\setbeamerfont{itemize/enumerate subbody}{size=\small} % tud defaults to scriptsize!
\setbeamerfont{itemize/enumerate subsubbody}{size=\small}
% \setbeamerfont{normal text}{size=\small}
% \setbeamerfont{itemize body}{size=\small}

\renewcommand{\emph}[1]{\textbf{#1}}

\def\arraystretch{1.3}% Make tables even less cramped vertically

\usepackage[ngerman]{babel}
\usepackage[utf8]{inputenc}
\usepackage[T1]{fontenc}

%\usepackage{graphicx}
\usepackage[export]{adjustbox} % loads graphicx
\usepackage{import}
\usepackage{stmaryrd}
\usepackage[normalem]{ulem} % sout command
% \usepackage{times}
\usepackage{txfonts}

% \usepackage[perpage]{footmisc} % reset footnote counter on each page -- fails with beamer (footnotes gone)
\usepackage{perpage}  % reset footnote counter on each page
\MakePerPage{footnote}

\usepackage{tikz}
\usetikzlibrary{arrows,positioning}
% Inspired by http://www.texample.net/tikz/examples/hand-drawn-lines/
\usetikzlibrary{decorations.pathmorphing}
\pgfdeclaredecoration{penciline}{initial}{
    \state{initial}[width=+\pgfdecoratedinputsegmentremainingdistance,
    auto corner on length=1mm,]{
        \pgfpathcurveto%
        {% From
            \pgfqpoint{\pgfdecoratedinputsegmentremainingdistance}
                      {\pgfdecorationsegmentamplitude}
        }
        {%  Control 1
        \pgfmathrand
        \pgfpointadd{\pgfqpoint{\pgfdecoratedinputsegmentremainingdistance}{0pt}}
                    {\pgfqpoint{-\pgfdecorationsegmentaspect
                     \pgfdecoratedinputsegmentremainingdistance}%
                               {\pgfmathresult\pgfdecorationsegmentamplitude}
                    }
        }
        {%TO 
        \pgfpointadd{\pgfpointdecoratedinputsegmentlast}{\pgfpoint{1pt}{1pt}}
        }
    }
    \state{final}{}
}
\tikzset{handdrawn/.style={decorate,decoration=penciline}}
\tikzset{every shadow/.style={fill=none,shadow xshift=0pt,shadow yshift=0pt}}
% \tikzset{module/.append style={top color=\col,bottom color=\col}}

% Use to make Tikz attributes with Beamer overlays
% http://tex.stackexchange.com/a/6155
\tikzset{onslide/.code args={<#1>#2}{%
  \only<#1| handout:0>{\pgfkeysalso{#2}} 
}}
\tikzset{onslideprint/.code args={<#1>#2}{%
  \only<#1>{\pgfkeysalso{#2}} 
}}

%%% Title -- always set this first

\newcommand{\defineTitle}[3]{
	\newcommand{\lectureindex}{#1}
	\title{Formale Systeme}
	\subtitle{\href{\lectureurl}{#1. Vorlesung: #2}}
	\author{\href{http://korrekt.org/}{Markus Kr\"{o}tzsch}}
%	\author{\href{http://www.sebastian-rudolph.de}{Sebastian Rudolph} in Vertretung von \href{http://korrekt.org/}{Markus Kr\"{o}tzsch}}
	\date{#3}
	\datecity{TU Dresden}
% 	\institute{Computational Logic}
}

%%% Table of contents:

\RequirePackage{ifthen}

\newcommand{\highlight}[2]{%
	\ifthenelse{\equal{#1}{\lectureindex}}{\alert{#2}}{#2}%
}

\def\myspace{-0.7ex}
\newcommand{\printtoc}{
\begin{tabular}{r@{$\quad$}l}
\highlight{1}{1.} & \highlight{1}{Willkommen/Einleitung formale Sprachen}\\[\myspace]
\highlight{2}{2.} & \highlight{2}{Grammatiken und die Chomsky-Hierarchie}\\[\myspace]
\highlight{3}{3.} & \highlight{3}{Endliche Automaten}\\[\myspace]
\highlight{4}{4.} & \highlight{4}{Complexity of FO query answering}\\[\myspace]
\highlight{5}{5.} & \highlight{5}{Conjunctive queries}\\[\myspace]
\highlight{6}{6.} & \highlight{6}{Tree-like conjunctive queries}\\[\myspace]
\highlight{7}{7.} & \highlight{7}{Query optimisation}\\[\myspace]
\highlight{8}{8.} & \highlight{8}{Conjunctive Query Optimisation / First-Order~Expressiveness}\\[\myspace]
\highlight{9}{9.} & \highlight{9}{First-Order~Expressiveness / Introduction to Datalog}\\[\myspace]
\highlight{10}{10.} & \highlight{10}{Expressive Power and Complexity of Datalog}\\[\myspace]
\highlight{11}{11.} & \highlight{11}{Optimisation and Evaluation of Datalog}\\[\myspace]
\highlight{12}{12.} & \highlight{12}{Evaluation of Datalog (2)}\\[\myspace]
\highlight{13}{13.} & \highlight{13}{Graph Databases and Path Queries}\\[\myspace]
\highlight{14}{14.} & \highlight{14}{Outlook: database theory in practice}
\end{tabular}
}

\newcommand{\overviewslide}{%
\begin{frame}\frametitle{Overview}
\printtoc
\medskip

Siehe \href{\lectureurl}{course homepage [$\Rightarrow$ link]} for more information and materials
\end{frame}
}

%%% Colours:

\usepackage{xcolor,colortbl}
\definecolor{redhighlights}{HTML}{FFAA66}
\definecolor{lightblue}{HTML}{55AAFF}
\definecolor{lightred}{HTML}{FF5522}
\definecolor{lightpurple}{HTML}{DD77BB}
\definecolor{lightgreen}{HTML}{55FF55}
\definecolor{darkred}{HTML}{CC4411}
\definecolor{darkblue}{HTML}{176FC0}%{1133AA}
\definecolor{nightblue}{HTML}{2010A0}%{1133AA}
\definecolor{alert}{HTML}{176FC0}
\definecolor{darkgreen}{HTML}{36AB14}
\definecolor{strongyellow}{HTML}{FFE219}
\definecolor{devilscss}{HTML}{666666}

\newcommand{\redalert}[1]{\textcolor{darkred}{#1}}

%%% Style commands

\newcommand{\quoted}[1]{\texttt{"}{#1}\texttt{"}}
\newcommand{\squote}{\texttt{"}} % straight quote
\newcommand{\Sterm}[1]{\ensuremath{\mathtt{\textcolor{purple}{#1}}}}    % letters in alphabets
\newcommand{\Snterm}[1]{\textsf{\textcolor{darkblue}{#1}}} % nonterminal symbols
\newcommand{\Sntermsub}[2]{\Snterm{#1}_{\Snterm{#2}}} % nonterminal symbols
\newcommand{\Slang}[1]{\textbf{\textcolor{black}{#1}}}    % languages
\newcommand{\Slangsub}[2]{\Slang{#1}_{\Slang{#2}}}    % languages
% Code
\newcommand{\Scode}[1]{\textbf{#1}}    % reserved words in program listings, e.g., "if"
\newcommand{\Scodelit}[1]{\textcolor{purple}{#1}}    % literals in program listings, e.g., strings
\newcommand{\Scomment}[1]{\textcolor{gray}{#1}}    % comment in program listings

\newcommand{\epstrastar}{\mathrel{\mathord{\stackrel{\epsilon}{\to}}{}^*}} % transitive reflexive closure of epsilon transitions in an epslion-NFA

\newcommand{\narrowcentering}[1]{\mbox{}\hfill#1\hfill\mbox{}}

\newcommand{\defeq}{\mathrel{:=}}

\newcommand{\Smach}[1]{\ensuremath{\mathcal{#1}}}    % machines

%%% Slide layout commands:

\newcommand{\sectionSlide}[1]{
\frame{\begin{center}
\LARGE
#1
\end{center}}
}
\newcommand{\sectionSlideNoHandout}[1]{
\frame<handout:0>{\begin{center}
\LARGE
#1
\end{center}}
}

\newcommand{\mydualbox}[3]{%
 \begin{minipage}[t]{#1}
 \begin{beamerboxesrounded}[upper=block title,lower=block body,shadow=true]%
    {\centering\usebeamerfont*{block title}#2}%
    \raggedright%
    \usebeamerfont{block body}
%     \small
    #3%
  \end{beamerboxesrounded}
  \end{minipage}
}
% 
\newcommand{\myheaderbox}[2]{%
 \begin{minipage}[t]{#1}
 \begin{beamerboxesrounded}[upper=block title,lower=block title,shadow=true]%
    {\centering\usebeamerfont*{block title}\rule{0pt}{2.6ex} #2}%
  \end{beamerboxesrounded}
  \end{minipage}
}

\newcommand{\mycontentbox}[2]{%
 \begin{minipage}[t]{#1}%
 \begin{beamerboxesrounded}[upper=block body,lower=block body,shadow=true]%
    {\centering\usebeamerfont*{block body}\rule{0pt}{2.6ex}#2}%
  \end{beamerboxesrounded}
  \end{minipage}
}

\newcommand{\mylcontentbox}[2]{%
 \begin{minipage}[t]{#1}%
 \begin{beamerboxesrounded}[upper=block body,lower=block body,shadow=true]%
    {\flushleft\usebeamerfont*{block body}\rule{0pt}{2.6ex}#2}%
  \end{beamerboxesrounded}
  \end{minipage}
}

% label=180:{\rotatebox{90}{{\footnotesize\textcolor{darkgreen}{Beispiel}}}}
% \hspace{-8mm}\ghost{\raisebox{-7mm}{\rotatebox{90}{{\footnotesize\textcolor{darkgreen}{Beispiel}}}}}\hspace{8mm}
\newcommand{\examplebox}[1]{%
	\begin{tikzpicture}[decoration=penciline, decorate]
		\pgfmathsetseed{1235}
		\node (n1) [decorate,draw=darkgreen, fill=darkgreen!10,thick,align=left,text width=\linewidth, inner ysep=2mm, inner xsep=2mm] at (0,0) {#1};
% 		\node (n2) [align=left,text width=\linewidth,inner sep=0mm] at (n1.92) {{\footnotesize\raisebox{3mm}{\textcolor{darkgreen}{Beispiel}}}};
% 		\node (n2) [decorate,draw=darkgreen, fill=darkgreen!10,thick, align=left,text width=\linewidth,inner sep=2mm] at (n1.90) {{\footnotesize\raisebox{0mm}{\textcolor{darkgreen}{Beispiel}}}};
	\end{tikzpicture}%
}%

\newcommand{\codebox}[1]{%
	\begin{tikzpicture}[decoration=penciline, decorate]
		\pgfmathsetseed{1236}
		\node (n1) [decorate,draw=strongyellow, fill=strongyellow!10,thick,align=left,text width=\linewidth, inner ysep=2mm, inner xsep=2mm] at (0,0) {#1};
	\end{tikzpicture}%
}%

\newcommand{\defbox}[1]{%
	\begin{tikzpicture}[decoration=penciline, decorate]
		\pgfmathsetseed{1237}
		\node (n1) [decorate,draw=darkred, fill=darkred!10,thick,align=left,text width=\linewidth, inner ysep=2mm, inner xsep=2mm] at (0,0) {#1};
	\end{tikzpicture}%
}%

\newcommand{\theobox}[1]{%
	\begin{tikzpicture}[decoration=penciline, decorate]
		\pgfmathsetseed{1240}
		\node (n1) [decorate,draw=darkblue, fill=darkblue!10,thick,align=left,text width=\linewidth, inner ysep=2mm, inner xsep=2mm] at (0,0) {#1};
	\end{tikzpicture}%
}%

\newcommand{\anybox}[2]{%
	\begin{tikzpicture}[decoration=penciline, decorate]
		\pgfmathsetseed{1240}
		\node (n1) [decorate,draw=#1, fill=#1!10,thick,align=left,text width=\linewidth, inner ysep=2mm, inner xsep=2mm] at (0,0) {#2};
	\end{tikzpicture}%
}%


\newsavebox{\mybox}%
\newcommand{\doodlebox}[2]{%
\sbox{\mybox}{#2}%
	\begin{tikzpicture}[decoration=penciline, decorate]
		\pgfmathsetseed{1238}
		\node (n1) [decorate,draw=#1, fill=#1!10,thick,align=left,inner sep=1mm] at (0,0) {\usebox{\mybox}};
	\end{tikzpicture}%
}%

% Common notation

\usepackage{amsmath,amssymb,amsfonts}
\usepackage{xspace}

\newcommand{\lectureurl}{https://iccl.inf.tu-dresden.de/web/FS2016}

\DeclareMathAlphabet{\mathsc}{OT1}{cmr}{m}{sc} % Let's have \mathsc since the slide style has no working \textsc

% Dual of "phantom": make a text that is visible but intangible
\newcommand{\ghost}[1]{\raisebox{0pt}[0pt][0pt]{\makebox[0pt][l]{#1}}}

\newcommand{\tuple}[1]{\langle{#1}\rangle}

%%% Annotation %%%

\usepackage{color}
\newcommand{\todo}[1]{{\tiny\color{red}\textbf{TODO: #1}}}



%%% Old macros below; move when needed

\newcommand{\blank}{\text{\textvisiblespace}} % empty tape cell for TM

% table syntax
\newcommand{\dom}{\textbf{dom}}
\newcommand{\adom}{\textbf{adom}}
\newcommand{\dbconst}[1]{\texttt{"#1"}}
\newcommand{\pred}[1]{\textsf{#1}}
\newcommand{\foquery}[2]{#2[#1]}
\newcommand{\ground}[1]{\textsf{ground}(#1)}
% \newcommand{\foquery}[2]{\{#1\mid #2\}} %% Notation as used in Alice Book
% \newcommand{\foquery}[2]{\tuple{#1\mid #2}}

\newcommand{\quantor}{\mathord{\reflectbox{$\text{\sf{Q}}$}}} % the generic quantor

% logic syntax
\newcommand{\Inter}{\mathcal{I}} %used to denote an interpretation
\newcommand{\Jnter}{\mathcal{J}} %used to denote another interpretation
\newcommand{\Knter}{\mathcal{K}} %used to denote yet another interpretation
\newcommand{\Zuweisung}{\mathcal{Z}} %used to denote a variable assignment

% query languages
\newcommand{\qlang}[1]{{\sf #1}} % Font for query languages
\newcommand{\qmaps}[1]{\textbf{QM}({\sf #1})} % Set of query mappings for a query language

%%% Complexities %%%

\hyphenation{Exp-Time} % prevent "Ex-PTime" (see, e.g. Tobies'01, Glimm'07 ;-)
\hyphenation{NExp-Time} % better that than something else

% \newcommand{\complclass}[1]{{\sc #1}\xspace} % font for complexity classes
\newcommand{\complclass}[1]{\ensuremath{\mathsc{#1}}\xspace} % font for complexity classes

\newcommand{\ACzero}{\complclass{AC$_0$}}
\newcommand{\LogSpace}{\complclass{L}}
\newcommand{\NLogSpace}{\complclass{NL}}
\newcommand{\PTime}{\complclass{P}}
\newcommand{\NP}{\complclass{NP}}
\newcommand{\coNP}{\complclass{coNP}}
\newcommand{\PH}{\complclass{PH}}
\newcommand{\PSpace}{\complclass{PSpace}}
\newcommand{\NPSpace}{\complclass{NPSpace}}
\newcommand{\ExpTime}{\complclass{ExpTime}}
\newcommand{\NExpTime}{\complclass{NExpTime}}
\newcommand{\ExpSpace}{\complclass{ExpSpace}}
\newcommand{\TwoExpTime}{\complclass{2ExpTime}}
\newcommand{\NTwoExpTime}{\complclass{N2ExpTime}}
\newcommand{\ThreeExpTime}{\complclass{3ExpTime}}
\newcommand{\kExpTime}[1]{\complclass{#1ExpTime}}
\newcommand{\kExpSpace}[1]{\complclass{#1ExpSpace}}


\defineTitle{21}{Aussagenlogik}{9. Januar 2017}

\begin{document}

\maketitle

% \begin{frame}\frametitle{}
% 
% ~\hfill
% \includegraphics[height=6.5cm]{a4}
% \hfill~
% 
% \end{frame}

\sectionSlide{Besprechung Lehrevaluation}

\newcommand{\praise}[1]{\textcolor{darkgreen}{"`#1"'}}
\newcommand{\critique}[1]{~\hfill\textcolor{darkred}{"`#1"'}}

\begin{frame}\frametitle{Kommentare: Allgemeines}

\narrowcentering{{\large \alert{Danke für die vielen freundlichen Kommentare!}}}
\bigskip

\emph{Konkrete technische Verbesserungswünsche werden soweit möglich umgesetzt:}
\begin{itemize}
\item "`Die Folien direkt nach der Vorlesung verfügbar haben"'
\item "`Das Mikro ist im Raum E02 im HSZ regelmäßig zu leise"'
\item "`Eine bessere Justierung des Beamers im HSZ/003 ;)"'
\item "`Besser auf Vorwissen aus AuD und Programmierung abstimmen"'
\item "`jede Woche 1 Baby-Foto"' (mal sehen \ldots)
\end{itemize}

\end{frame}

\begin{frame}\frametitle{Kommentare: Beispiele und Praxisbezug}

\textbf{\textcolor{darkgreen}{\ldots besonders gut gefallen?}}\hfill\textbf{\textcolor{darkred}{Verbesserungswünsche?}}
\bigskip

\praise{Sehr viele Beispiele}\\[1ex]
\critique{Ich finde es besser, wenn es ein bisschen mehr Beispiele gibt.}\\[1ex]
\praise{Viele einfache verständliche Beispiele}\\[1ex]
\critique{Mehr Beispiele.}\\[3ex]
\praise{macht eher `trockenen' Stoff spannend}\\[1ex]
\critique{Praxisbezug erhöhen}\\[1ex]
\critique{Mehr Praxisbezug}\\[1ex]
\critique{Mehr Praxis-Beispiele}

\end{frame}

\begin{frame}\frametitle{Kommentare: Beweise und Tempo}

\textbf{\textcolor{darkgreen}{\ldots besonders gut gefallen?}}\hfill\textbf{\textcolor{darkred}{Verbesserungswünsche?}}
\bigskip

\praise{Beweise <3}\\[1ex]
\critique{Ein oder zwei Beweise weniger wären gut. ;)}\\[1ex]
\critique{Weniger Beweise}\\[1ex]
\praise{Beweise gut erklärt}\\[1ex]
\critique{Beweise teilweise \emph{zu schnell} erklärt}\\[1ex]
\critique{Tempo etwas \emph{verlangsamen}}\\[1ex]
\critique{\emph{Weniger ausführliche} Beweise}\\[1ex]
\critique{Manchmal werden Themen etwas \emph{zu schnell} behandelt, wenn man selbst Notizen machen will, vor allem Beweise}\\[1ex]
\critique{\emph{Tempo könnte} an einigen Stellen deutlich \emph{höher sein} (vor allem Beweise)}

\end{frame}

\begin{frame}\frametitle{Kommentare: Zusammenfassungen und Wiederholungen}

\textbf{\textcolor{darkgreen}{\ldots besonders gut gefallen?}}\hfill\textbf{\textcolor{darkred}{Verbesserungswünsche?}}
\bigskip

\praise{Regelmäßiges Zusammenfassen des Stoffes}\\[1ex]
\praise{Wiederholungen}\\[1ex]
\praise{Wiederholungen zur Festigung des Stoffes}\\[1ex]
\critique{Die Zusammenfassungen und Wiederholungen des bereits abgeschlossenen Stoffes sind viel zu Ausführlich}\\[1ex]
\critique{nicht alles 3 mal erklären, man versteht es beim ersten mal}\\[1ex]
\critique{Mehr Wiederholungen. Explain it like I'm 5}

\end{frame}

\begin{frame}\frametitle{Kommentare: Tafel vs. Beamer}

\textbf{\textcolor{darkgreen}{\ldots besonders gut gefallen?}}\hfill\textbf{\textcolor{darkred}{Verbesserungswünsche?}}
\bigskip

% \praise{gute Folien}\\[1ex]
\praise{Sehr gute Folien (interaktiv)}\\[0.5ex]
\praise{Bunte Folien, Schemata und Veranschaulichungen leicht verständlich}\\[0.5ex]
\praise{Sehr übersichtl. Folien}\\[0.5ex]
\praise{die Folien, die im Internet zur Verfügung stehen, finde ich prima}\\[0.5ex]
\praise{excellente Latex-Folien}\\[0.5ex]
\praise{Verständliche Folien}\\[0.5ex]
\praise{sehr anschauliche und erklärende Folien, die auch nachvollziehbar online sind}\\[1ex]
\critique{Tafelschriebe helfen mir konzentrierter zu bleiben}\\[1ex]
\critique{Verhältnis Powerpoint<->Tafel mehr zu Gunsten der Tafel verschieben. Bei ppt-Folien bleibt zu wenighängen bzw. das Tempo erhöht sich.}

\end{frame}

\begin{frame}\frametitle{Kommentare: Politik}


\textbf{\textcolor{darkgreen}{\ldots besonders gut gefallen?}}\hfill\textbf{\textcolor{darkred}{Verbesserungswünsche?}}
\bigskip

\praise{`politische' Statements}\\[1ex]
\praise{Blick über den Tellerrand der Vorlesung, durch Kommentare zur Politischen Situation}\\[1ex]
\critique{Zu viel Politik}\\[1ex]
\critique{bitte keine Werbung für `\underline{\href{https://herzstatthetze.jimdo.com/}{Herz-statt-Hetze}}' oder ähnliches}\\[1ex]
\critique{keine/weniger politische Statements in Vorlesungen, Forschung + Lehre sollten neutral sein}

\end{frame}

\begin{frame}\frametitle{Ankündigung}

\begin{center}
Weitere Kommentare\\
(gesellschaftspolitisch oder anderweitig)?\\[1ex]
Dann kommen Sie zum\\[4ex]

{\LARGE Professorenstammtisch}\\[2ex]

\large
11.1.2017\\
18:30 Uhr\\
Im "`Campus"'
\end{center}

\end{frame}

% \sectionSlideNoHandout{Rückblick}
% 
% \begin{frame}\frametitle{Typ 0 und Typ 1}
% 
% \ldots
% 
% \end{frame}

\sectionSlide{Logik}


\begin{frame}\frametitle{Logik für Informatiker}

\pause
\begin{center}
\Huge Warum?
\end{center}

\end{frame}

\begin{frame}\frametitle{Logik für Informatiker: Darum (1)}

\begin{center}
\LARGE 
\[ \frac{\text{Logik}}{\text{Informatik}} = \frac{\text{Analysis}}{\text{Ingenieurwesen}}\]

\bigskip
\large
"`Wer rechnende Systeme verstehen und konstruieren will, der benötigt passende mathematische Modelle.\\Dieser Weg führt oftmals zur Logik."'
\end{center}


\begin{itemize}
\item Modellierung von Programm\alert{logik} und \alert{logischen} Schaltungen
\item Berechnung praktisch relevanter Eigenschaften durch logisches Schließen
\item Hauptanwendung: \redalert{Verifikation von Systemen}
\end{itemize}

\end{frame}

\begin{frame}\frametitle{Logik für Informatiker: Darum (2)}

\begin{center}
\LARGE 
Logik = Wissenschaft vom folgerichtigen Denken

\bigskip
\large
"`Wer intelligente Software entwickeln will, der muss logische Schlussfolgerungen algorithmisch umsetzen.\\
Die Logik liefert die nötigen Methoden."'
\end{center}


\begin{itemize}
\item Kodierung von gültigen Zusammenhängen und Regeln
\item Logisches Schließen als Simulation von intelligentem Denken
\item Hauptanwendung: \redalert{Künstliche Intelligenz}
\end{itemize}

\end{frame}


\begin{frame}\frametitle{Logik für Informatiker: Darum (3)}

\begin{center}
\LARGE 
Logisches Schließen = Problemlösen

\bigskip
\large
"`Bedeutende Klassen von (schweren) Problemen lassen sich durch logische Schlussfolgerung lösen.\\
Algorithmen aus der Logik sind in vielen anderen Bereichen anwendbar."'
\end{center}


\begin{itemize}
\item Logik als Spezifikationssprache für komplexe Probleme
\item Logisches Schließen als Suche nach zulässigen Lösungen
\item Anwendungen: \redalert{Constraint-Satisfaction-Probleme} und verwandte Optimierungsaufgaben
\end{itemize}

\end{frame}

\begin{frame}\frametitle{Logik für Informatiker: Darum (4)}

\begin{center}
\LARGE 
Logiken = Beschreibungssprachen

\bigskip
\large
"`Überall wo Informationen maschinell kodiert werden und wo exakt spezifiziert ist, was eine Anwendung\\ aus dieser Kodierung ableiten darf,\\hat man es mit einer Art Logik zu tun."'
\end{center}


\begin{itemize}
\item Logik als Oberbegriff exakt spezifizierter Datenformate
\item Schlussfolgerung zur \ghost{Interpretation/Analyse/Optimierung}
\item Anwendungen: \redalert{Wissensrepräsentation} und \redalert{Datenbanken}
\end{itemize}
\end{frame}

\begin{frame}\frametitle{Was ist Logik?}
\pause

"`Logik"' ist ein allgemeiner Oberbegriff für viele mathematische und technische Formalismen,
gekennzeichnet durch:
\begin{itemize}
\item \redalert{Syntax:} Sprache einer Logik (normalerweise Formeln mit logischen Operatoren)
\item \redalert{Semantik:} Definition der Bedeutung (Worauf beziehen sich die Formeln? Wann ist eine Formel wahr oder falsch?)
% \item \redalert{Inferenz:} Spezifikation der korrekten logische Schlussfolgerungen (Welche neuen Formeln kann man aus den bekannten ableiten?)
\end{itemize}\bigskip\pause

Typische Zielstellung: \redalert{Logische Schlussfolgerung}
\begin{itemize}
\item Welche Schlüsse kann man aus einer gegebenen (Menge von) Formel(n) ziehen?
\item Spezifikation der korrekten Schlussfolgerungen sollte sich aus Semantik ergeben
\item Praktische Berechnung von Schlussfolgerungen ist oft kompliziert
\end{itemize}

\end{frame}

\begin{frame}\frametitle{Viele Logiken}

Es gibt sehr viele Logiken, z.B.
\begin{itemize}
\item Aussagenlogik
\item Prädikatenlogik (erster Stufe)
\item Prädikatenlogik zweiter Stufe
\item Beschreibungslogiken (Wissensrepräsentation und KI)
\item Temporallogiken (z.B. Verifikation zeitlicher Abläufe)
\item Logikprogramme (Answer Set Programming, Prolog, \ldots)
\item Nicht-klassische Logiken (z.B. intuitionistische Logik)
\item Mehrwertige Logiken (z.B. probabilistische Logik)
\item \ldots und viele andere mehr
\end{itemize}

$\leadsto$ In dieser Vorlesung lernen wir zunächst \alert{Aussagenlogik} kennen
\bigskip

(Mehr gibt es in der Vorlesung "`Theoretische Informatik und Logik"')

\end{frame}

\sectionSlide{Aussagenlogik}

\begin{frame}\frametitle{Aussagenlogik}

Die Aussagenlogik untersucht \alert{logische Verknüpfungen von atomaren Aussagen}.
\bigskip

\redalert{Atomare Aussagen} sind Behauptungen, die wahr oder falsch sein können, z.B.:

\examplebox{\vspace{-2ex}
\begin{enumerate}[{A}1]
\item "`Morgen schneit es."'
\item "`Wir werden einen Schneemann bauen."'
% \item "`"'
\end{enumerate}\vspace{-1ex}
\begin{enumerate}[{B}1]
\item "`Die Vorstellung von der globalen Erderwärmung wurde von den Chinesen erfunden."'
\item "`Die Temperatur der Ozeane ist seit 1998 in unverminderter Geschwindigkeit angestiegen."'
\end{enumerate}\vspace{-1ex}
\begin{enumerate}[{C}1]
\item "`Typ-2-Sprachen sind unter Schnitt abgeschlossen."'
\item "`Typ-2-Sprachen sind unter Komplement abgeschlossen."'
\item "`Typ-2-Sprachen sind unter Vereinigung abgeschlossen."'
\end{enumerate}
}

\end{frame}

\begin{frame}\frametitle{Aussagen verknüpfen}

Atomare Aussagen können mithilfe logischer \alert{Junktoren} verknüpft werden:

\examplebox{\vspace{-2ex}
\begin{itemize}
\item $\text{A1}\to \text{A2}$: "`Wenn es morgen schneit, dann werden wir einen Schneemann bauen."'
\item $\text{A1}\vee \neg\text{A1}$: "`Entweder schneit es morgen oder nicht."'
\item $\text{B1}\to \neg\text{B2}$: "`Falls die Chinesen die globale Erwärmung erfunden haben, dann steigt die Temperatur der Ozeane nicht unvermindert an."'
\item $(\text{C1}\wedge\text{C2})\to \text{C3}$: "`Sind Typ-2-Sprachen unter Schnitt und Komplement abgeschlossen, dann sind sie auch unter Vereinigung abgeschlossen."'
\end{itemize}}

\end{frame}

% \begin{frame}
% 
% ~\hfill
% \includegraphics[height=6.5cm]{a6}
% \hfill~
% % \rotatebox{90}{\tiny Randall Munroe, \url{http://xkcd.com/208/}, CC-BY-NC 2.5}
% 
% \end{frame}

\begin{frame}\frametitle{Logisches Schließen}

Wenn einige Formeln als wahr angenommen werden, dann kann die Wahrheit anderer Formeln daraus abgeleitet werden.
\bigskip

\examplebox{%
Beispiele:
\begin{itemize}
\item Aus $\text{B2}$ und $\text{B1}\to \neg\text{B2}$ folgt $\neg\text{B1}$
\item Aus $\text{C1}$, $\neg\text{C3}$ und $(\text{C1}\wedge\text{C2})\to \text{C3}$ folgt $\neg\text{C2}$
\item Aus $\text{A1}\to \text{A2}$ und $\neg\text{A1}$ folgt \redalert{nicht} $\neg\text{A2}$
\end{itemize}}\pause\bigskip

\anybox{purple}{
Die Gültigkeit bestimmter Schlussfolgerungen hat nichts mit Schneemännern, Erderwärmung oder Typ-2-Sprachen zu tun!\\
Sie ist eine rein logische Konsequenz.
}

\end{frame}

\newcommand{\Aname}{Anna}
\newcommand{\Bname}{Barbara}
\newcommand{\Cname}{Chris}

\begin{frame}\frametitle{Beispiel: Logelei}

\Aname{} behauptet: "`\Bname{} lügt!"'
\bigskip

\Bname{} behauptet: "`\Cname{} lügt!"'
\bigskip

\Cname{} behauptet: "`\Aname{} und \Bname{} lügen!"'
\bigskip\bigskip

{\Large
\narrowcentering{\alert{Wer lügt?}}
}

\medskip
{\footnotesize\narrowcentering{(Und wie kann man das beweisen?)}}
% \bigskip

% 
% $A\leftrightarrow\neg B$
% $B\leftrightarrow\neg C$
% $C\leftrightarrow\neg A \wedge \neg B$

\end{frame}

\begin{frame}\frametitle{Aussagenlogik: Syntax}

Wir betrachten eine abzählbar unendliche Menge $\Slang{P}$ von
\redalert{atomaren Aussagen} (auch bekannt als: \redalert{aussagenlogische Variablen}, \redalert{Propositionen} oder schlicht \redalert{Atome})
\medskip

\defbox{Die Menge der \redalert{aussagenlogischen Formeln} ist induktiv\footnote{Das bedeutet: Die Definition ist selbstbezüglich und soll die kleinste\\ Menge an Formeln beschreiben, die alle Bedingungen erfüllen.} definiert:
\begin{itemize}
\item Jedes Atom $p\in\Slang{P}$ ist eine aussagenlogische Formel
\item Wenn $F$ und $G$ aussagenlogische Formeln sind, so auch:
	\begin{itemize}
	\item $\neg F$: \redalert{Negation}, "`nicht $F$"'
	\item $(F\wedge G)$: \redalert{Konjunktion}, "`$F$ und $G$"'
	\item $(F\vee G)$: \redalert{Disjunktion}, "`$F$ oder $G$"'
	\item $(F\to G)$: \redalert{Implikation}, "`$F$ impliziert $G$"'
	\item $(F\leftrightarrow G)$: \redalert{Äquivalenz}, "`$F$ ist äquivalent zu $G$"'
	\end{itemize}
\end{itemize}
}

Wir verzichten hier oft auf "`aussagenlogisch"' und sprechen z.B. einfach von "`Formeln"'.

\end{frame}

\begin{frame}\frametitle{Beispiele}

Die folgenden Ausdrücke sind aussagenlogische Formeln:
\begin{itemize}
\item $p$
\item $((p\to q)\to p)\to p$
\item $\neg \neg \neg p$
\end{itemize}
\medskip

Die folgenden Ausdrücke sind \redalert{keine} aussagenlogischen Formeln:
\begin{itemize}
\item $p\wedge q \vee r$ (fehlende Klammern)\\
\anybox{strongyellow}{%
\alert{Vereinfachung:} Äußere Klammern dürfen wegfallen, d.h. wir erlauben z.B. $p\to q$ anstatt auf $(p\to q)$ zu bestehen.
}
\item $(p\leftarrow q)$ (Operator $\leftarrow$ undefiniert)
% \item $(p\leftrightarrow q)$ (Operator $\leftrightarrow$ undefiniert)
% \anybox{strongyellow}{%
% \alert{Notation:} Wir erlauben $(p\leftarrow q)$ als alternative Schreibweise für $(q\to p)$ und $(p\leftrightarrow q)$
% als Abkürzung für $((p\to q)\wedge (q\to p))$.}
\item $\overline{p\wedge q}$ (wir verwenden $\overline{\phantom{x}}$ nicht als Negation)
\end{itemize}

\end{frame}

\begin{frame}\frametitle{Teilformeln}

Wir können Formeln als Wörter über (endlichen Teilmengen aus) dem unendlichen Alphabet
$\Slang{P}\cup\{\Sterm{\neg},\Sterm{\wedge},\Sterm{\vee},\Sterm{\to},\Sterm{\leftrightarrow},\Sterm{(},\Sterm{)}\}$ sehen:

\[ \Snterm{F}\to \Slang{P}\mid \Sterm{\neg}\Snterm{F}\mid\Sterm{(}\Snterm{F}\Sterm{\wedge}\Snterm{F}\Sterm{)} \mid\Sterm{(}\Snterm{F}\Sterm{\vee}\Snterm{F}\Sterm{)} \mid\Sterm{(}\Snterm{F}\Sterm{\to}\Snterm{F}\Sterm{)} \mid\Sterm{(}\Snterm{F}\Sterm{\leftrightarrow}\Snterm{F}\Sterm{)}
\]\pause
%
Eine \redalert{Teilformel} ist ein Teilwort (Infix) einer Formel, welches selbst eine Formel ist.
Teilformeln werden auch \redalert{Unterformeln} genannt.\pause
\medskip

Alternativ kann man Teilformeln auch rekursiv definieren:

\defbox{Die \redalert{Menge $\textsf{Sub}(F)$ der Teilformeln} einer Formel $F$ ist definiert als:
\vspace{-2ex}
\[ \textsf{Sub}(F) =\left\{\begin{array}{l@{~~}l@{}}
	\{F\} & \text{falls }F\in\Slang{P} \\
	\{\neg G\}\cup\textsf{Sub}(G) & \text{falls }F=\neg G \\
	\{(G_1\wedge G_2)\}\cup\textsf{Sub}(G_1)\cup\textsf{Sub}(G_2) & \text{falls }F=(G_1\wedge G_2) \\
	\{(G_1\vee G_2)\}\cup\textsf{Sub}(G_1)\cup\textsf{Sub}(G_2) & \text{falls }F=(G_1\vee G_2) \\
	\{(G_1\to G_2)\}\cup\textsf{Sub}(G_1)\cup\textsf{Sub}(G_2) & \text{falls }F=(G_1\to G_2) \\
	\{(G_1\leftrightarrow G_2)\}\cup\textsf{Sub}(G_1)\cup\textsf{Sub}(G_2) & \text{falls }F=(G_1\leftrightarrow G_2)
\end{array}\right.\]\vspace{-2ex}
}

\end{frame}

\begin{frame}\frametitle{Semantik}

\alert{Was bedeutet eine aussagenlogische Formel?}\pause
\begin{itemize}
\item Atome an sich bedeuten zunächst nichts\\
$\leadsto$ sie können einfach wahr oder falsch sein\pause
\item Je nachdem, welche Atome wahr sind und welche falsch, ergeben sich verschiedene "`Interpretationen"'\\
$\leadsto$ dargestellt durch \redalert{Wertzuweisungen}\\
\mbox{}\phantom{$\leadsto$} (Funktionen von $\Slang{P}$ nach $\{\mytrue,\myfalse\}$; $\mytrue$="`wahr"' und $\myfalse$="`falsch"')\pause
\item Die Wahrheit (oder Falschheit) einer Formel ergibt sich aus dem Wahrheitswert der in ihr vorkommenden Atome\\
$\leadsto$ Wertzuweisungen machen Formeln wahr oder falsch
\end{itemize}

\anybox{purple}{
Die Bedeutung einer Formel besteht darin, dass sie uns Informationen darüber liefert,
welche Wertzuweisungen möglich sind, wenn die Formel wahr sein soll.
}

\end{frame}

\begin{frame}\frametitle{Wertzuweisungen können Formeln erfüllen}

Eine \redalert{Wertzuweisung} ist eine Funktion $w:\Slang{P}\to\{\mytrue,\myfalse\}$
\bigskip

% Die Wahrheit einer Formel kann rekursiv definiert werden:

\defbox{Eine Wertzuweisung $w$ \redalert{erfüllt} eine Formel $F$, in Symbolen \redalert{$w\models F$}, wenn
eine der folgenden rekursiven Bedingungen gilt:\smallskip

\begin{tabular}{rll}
\rowcolor{darkred!70!gray}
\textcolor{white}{Form von $F$} & \textcolor{white}{$w\models F$ wenn:} & \textcolor{white}{$w\not\models F$ wenn:}\\
$F\in \Slang{P}$: & $w(F)=\mytrue$ & $w(F)=\myfalse$\\
\rowcolor{lightred!20}
$F=\neg G$ & $w\not\models G$ & $w\models G$\\
$F=(G_1\wedge G_2)$ & $w\models G_1$ und $w\models G_2$ & $w\not\models G_1$ oder $w\not\models G_2$\\
\rowcolor{lightred!20}
$F=(G_1\vee G_2)$ & $w\models G_1$ oder $w\models G_2$ & $w\not\models G_1$ und $w\not\models G_2$\\
$F=(G_1\to G_2)$ & $w\not\models G_1$ oder $w\models G_2$ & $w\models G_1$ und $w\not\models G_2$\\
\rowcolor{lightred!20}
$F=(G_1\leftrightarrow G_2)$ & $w\models G_1$ und $w\models G_2$ & $w\models G_1$ und $w\not\models G_2$\\%[-1ex]
\rowcolor{lightred!20}
	& \multicolumn{1}{c}{\raisebox{1ex}{oder}} & \multicolumn{1}{c}{\raisebox{1ex}{oder}} \\[-2ex]
\rowcolor{lightred!20}
	& $w\not\models G_1$ und $w\not\models G_2$ & $w\not\models G_1$ und $w\models G_2$\\[-1ex]
\\[-1.5ex]
\end{tabular}

% Dabei schließt "`oder"' immer den Fall ein, dass beide Möglichkeiten gelten.
Dabei bedeutet "`A oder B"' immer "`A oder B oder beides"'.

% 
% \begin{tabular}{rl}
% $F\in \Slang{P}$: & $w\models F$ gdw. $w(F)=\mytrue$\\
% $F=\neg G$: & $w\models F$ gdw. $w\not\models G$\\
% $F=(G_1\wedge G_2)$: & $w\models F$ gdw. $w\models G_1$ und $w\models G_2$
% \end{tabular}
% 
% \begin{itemize}
% \item $F\in \Slang{P}$ mit $w(F)=\mytrue$
% \item $F=\neg G$ mit $w\not\models G$
% \item $F=(G_1\wedge G_2)$ mit $w\models G_1$ und $w\models G_2$
% \item $F=(G_1\vee G_2)$ mit $w\models G_1$ oder $w\models G_2$ (oder beides!)
% \item $F=(G_1\to G_2)$ mit $w\not\models G_1$ oder $w\models G_2$
% \item $F=(G_1\leftrightarrow G_2)$ mit $w\models G_1$ genau dann wenn $w\models G_2$
% \end{itemize}
% 
% Der \redalert{Wahrheitswert $\hat{w}(F)$} einer Formel $F$ unter einer Wertezuweisung $w$ ist
% wie folgt definiert:
% \vspace{-2ex}
% \[ \hat{w}(F) =\left\{\begin{array}{l@{~~}l@{}}
% 	w(F) & \text{falls }F\in\Slang{P} \\
% 	\{\neg G\}\cup\textsf{Sub}(G) & \text{falls }F=\neg G \\
% 	\{(G_1\wedge G_2)\}\cup\textsf{Sub}(G_1)\cup\textsf{Sub}(G_2) & \text{falls }F=(G_1\wedge G_2) \\
% 	\{(G_1\vee G_2)\}\cup\textsf{Sub}(G_1)\cup\textsf{Sub}(G_2) & \text{falls }F=(G_1\vee G_2) \\
% 	\{(G_1\to G_2)\}\cup\textsf{Sub}(G_1)\cup\textsf{Sub}(G_2) & \text{falls }F=(G_1\to G_2) \\
% 	\{(G_1\leftrightarrow G_2)\}\cup\textsf{Sub}(G_1)\cup\textsf{Sub}(G_2) & \text{falls }F=(G_1\leftrightarrow G_2)
% \end{array}\right.\]\vspace{-2ex}
}

\end{frame}

\begin{frame}\frametitle{Formeln Wahrheitswerte zuweisen}

Wir können Wertzuweisungen von Atomen auf Formeln erweitern:
\[ w(F)=\left\{\begin{array}{rl}\mytrue & \text{falls } w\models F\\
\myfalse & \text{falls } w\not\models F
\end{array}\right.\]\pause

\redalert{Wahrheitswertetabellen} illustrieren die Semantik der Junktoren:\medskip

\narrowcentering{%
\begin{tabular}{cc}
\rowcolor{lightblue!20}
$w(F)$ & $w(\neg F)$\\
\myfalse & \mytrue \\
\rowcolor{gray!10}
\mytrue & \myfalse \\
\end{tabular}}\medskip

\begin{tabular}{cccccc}
\rowcolor{lightblue!20}
$w(F)$ & $w(G)$ & $w(F\wedge G)$ & $w(F\vee G)$ & $w(F\to G)$ & $w(F\leftrightarrow G)$\\
\myfalse & \myfalse & \myfalse & \myfalse& \mytrue & \mytrue\\
\rowcolor{gray!10}
\mytrue & \myfalse & \myfalse & \mytrue& \myfalse & \myfalse\\
\myfalse & \mytrue & \myfalse & \mytrue& \mytrue & \myfalse\\
\rowcolor{gray!10}
\mytrue & \mytrue & \mytrue & \mytrue& \mytrue & \mytrue\\
\end{tabular}

\end{frame}

\begin{frame}\frametitle{Wahrheitswerte von Formeln bestimmen}

\begin{itemize}
\item Der Wahrheitswert einer Formel hängt nur vom Wahrheitswert der (endlich vielen) Atome ab, die in
ihr vorkommen.\\
$\leadsto$ Wir geben oft nur diese an.
\item Der Wahrheitswert einer Formel ergibt sich rekursiv aus dem Wahrheitswert ihrer Teilformeln.\\
$\leadsto$ Darstellung in Wahrheitswertetabelle
\end{itemize}

\examplebox{Beispiel: Für die Formel $F=((p\to q)\leftrightarrow(\neg q\to\neg p))$ und Wertzuweisung 
$w$ mit $w(p)=\myfalse$ und $w(q)=\mytrue$ ergibt sich die folgende Tabelle:
\medskip

\begin{tabular}{ccccccc}
\rowcolor{darkgreen!30}
$w(p)$ & $w(q)$ & $w(p\to q)$ & $w(\neg p)$ & $w(\neg q)$ & $w(\neg q\to \neg p)$ & $w(F)$\\
\myfalse & \mytrue & \mytrue & \mytrue & \myfalse & \mytrue & \mytrue\\
\end{tabular}

}

\end{frame}

\begin{frame}\frametitle{Beispiel: Logelei}

\Aname{} behauptet: "`\Bname{} lügt!"'\hfill \visible<2->{$A\leftrightarrow \neg B$}
\bigskip

\Bname{} behauptet: "`\Cname{} lügt!"'\hfill \visible<3->{$B\leftrightarrow \neg C$}
\bigskip

\Cname{} behauptet: "`\Aname{} und \Bname{} lügen!"'\hfill \visible<4->{$C\leftrightarrow (\neg A\wedge \neg B)$}
\bigskip\bigskip

{\Large
\narrowcentering{\alert{Wer lügt?}}
}

\medskip
{\footnotesize\narrowcentering{(Und wie kann man das beweisen?)}}
% \bigskip

% 
% $A\leftrightarrow\neg B$
% $B\leftrightarrow\neg C$
% $C\leftrightarrow\neg A \wedge \neg B$

\end{frame}

\begin{frame}\frametitle{Logische Konsequenzen}

\defbox{Eine Wertzuweisung $w$ ist \redalert{Modell einer Formel $F$} wenn $w\models F$ (also wenn $w(F)=\mytrue$).
% Wir erweitern diese Notation auf (möglicherweise unendliche) Formelmengen: $w$ ist ein \redalert{Modell der Menge $\mathcal{F}$}
%  wenn $w\models F$ für alle $F\in\mathcal{F}$ gilt. In diesem Fall schreiben wir \redalert{$w\models\mathcal{F}$}.
% 
Ist $\mathcal{F}$ eine (möglicherweise unendliche) Menge von Formeln, dann ist $w$ ein \redalert{Modell der Menge $\mathcal{F}$}
 wenn $w\models F$ für alle $F\in\mathcal{F}$ gilt. In diesem Fall schreiben wir \redalert{$w\models\mathcal{F}$}.
}\pause

Die logischen Schlussfolgerungen aus einer Formel(menge) ergeben sich aus ihren Modellen:

\defbox{Sei $\mathcal{F}$ eine Menge von Formeln.
Eine Formel $G$ ist eine \redalert{logische Konsequenz} aus $\mathcal{F}$ wenn jedes Modell
von $\mathcal{F}$ auch ein Modell von $G$ ist. In diesem Fall schreiben wir $\mathcal{F}\models G$.
}

\begin{enumerate}[$\leadsto$]
\item Die Formeln in $\mathcal{F}$ schränken die möglichen Interpretationen \ghost{ein:}\\
Je mehr Formeln wahr sein sollen, desto weniger Freiheiten gibt es bei der Wahl der Modelle
\item $G$ ist eine logische Konsequenz wenn gilt: falls $\mathcal{F}$ wahr ist, dann ist auch $G$ garantiert wahr
% 
% Eine logische Konsequenz $G$ von $\mathcal{F}$ ist in jedem Fall (jedem Modell wahr in dem $\mathcal{F}$ wahr ist
\end{enumerate}

\end{frame}

\begin{frame}[t]\frametitle{Beispiel: Logelei}

\Aname{} behauptet: "`\Bname{} lügt!"'\hfill {$A\leftrightarrow \neg B$}\\[1ex]
\Bname{} behauptet: "`\Cname{} lügt!"'\hfill {$B\leftrightarrow \neg C$}\\[1ex]
\Cname{} behauptet: "`\Aname{} und \Bname{} lügen!"'\hfill {$C\leftrightarrow (\neg A\wedge \neg B)$}\\[2ex]

\begin{tabular}{c@{~ }c@{~ }cc@{\hspace{3mm}}c@{\hspace{3mm}}c}
\rowcolor{lightblue!20}
$w(A)$ & $w(B)$ & $w(C)$ & $w(A\leftrightarrow \neg B)$ & $w(B\leftrightarrow \neg C)$ & $w(C\leftrightarrow (\neg A\wedge \neg B))$\\
\myfalse & \myfalse & \myfalse & \myfalse& \myfalse & \myfalse\\
\rowcolor{gray!10}
\mytrue & \myfalse & \myfalse & \mytrue& \myfalse & \mytrue\\
\myfalse & \mytrue & \myfalse & \mytrue& \mytrue & \mytrue\\
\rowcolor{gray!10}
\mytrue & \mytrue & \myfalse & \myfalse& \mytrue & \mytrue\\
\myfalse & \myfalse & \mytrue & \myfalse& \mytrue & \mytrue\\
\rowcolor{gray!10}
\mytrue & \myfalse & \mytrue & \mytrue& \mytrue & \myfalse\\
\myfalse & \mytrue & \mytrue & \mytrue& \myfalse & \myfalse\\
\rowcolor{gray!10}
\mytrue & \mytrue & \mytrue & \myfalse& \myfalse & \myfalse\\
\end{tabular}

% {\Large
% \narrowcentering{\alert{Wer lügt?}}
% }
% 
% \medskip
% {\footnotesize\narrowcentering{(Und wie kann man das beweisen?)}}
% \bigskip


\end{frame}

\begin{frame}[t]\frametitle{Beispiel: Logelei (2)}

\Aname{} behauptet: "`\Bname{} lügt!"'\hfill {$A\leftrightarrow \neg B$}\\[1ex]
\Bname{} behauptet: "`\Cname{} lügt!"'\hfill {$B\leftrightarrow \neg C$}\\[1ex]
\Cname{} behauptet: "`\Aname{} und \Bname{} lügen!"'\hfill {$C\leftrightarrow (\neg A\wedge \neg B)$}\\[2ex]

\begin{tabular}{c@{~ }c@{~ }cc@{\hspace{3mm}}c@{\hspace{3mm}}c}
\rowcolor{lightblue!20}
$w(A)$ & $w(B)$ & $w(C)$ & $w(A\leftrightarrow \neg B)$ & $w(B\leftrightarrow \neg C)$ & $w(C\leftrightarrow (\neg A\wedge \neg B))$\\
% \myfalse & \myfalse & \myfalse & \myfalse& \myfalse & \myfalse\\
% \rowcolor{gray!10}
% \mytrue & \myfalse & \myfalse & \mytrue& \myfalse & \mytrue\\
\myfalse & \mytrue & \myfalse & \mytrue& \mytrue & \mytrue\\
% \rowcolor{gray!10}
% \mytrue & \mytrue & \myfalse & \myfalse& \mytrue & \mytrue\\
% \myfalse & \myfalse & \mytrue & \myfalse& \mytrue & \mytrue\\
% \rowcolor{gray!10}
% \mytrue & \myfalse & \mytrue & \mytrue& \mytrue & \myfalse\\
% \myfalse & \mytrue & \mytrue & \mytrue& \myfalse & \myfalse\\
% \rowcolor{gray!10}
% \mytrue & \mytrue & \mytrue & \myfalse& \myfalse & \myfalse\\
\end{tabular}

$\leadsto$ Genau ein Modell (bezüglich der relevanten Atome)
\pause\bigskip

\emph{Logische Konsequenzen:} Alle Formeln, die unter einer Wertzuweisung mit $w(A)=\myfalse$, $w(B)=\mytrue$ und $w(C)=\myfalse$ wahr sind, \ghost{z.B.:}
\begin{itemize}
\item $\neg A$ ("`\Aname{} lügt"')
\item $\neg C$ ("`\Cname{} lügt"')
\item $\neg A\wedge \neg C$ ("`\Aname{} und \Cname{} lügen"')
\item $B$ ("`\Bname{} sagt die Wahrheit"')
\item Aber auch: $D\vee \neg D$
\end{itemize}

% {\Large
% \narrowcentering{\alert{Wer lügt?}}
% }
% 
% \medskip
% {\footnotesize\narrowcentering{(Und wie kann man das beweisen?)}}
% \bigskip


\end{frame}


\begin{frame}\frametitle{Allgemeingültigkeit und Co.}

\defbox{Eine Formel $F$ ist:
\begin{itemize}
\item \redalert{unerfüllbar} (oder \alert{inkonsistent}), wenn sie keine Modelle hat
\item \redalert{erfüllbar} (oder \alert{konsistent}), wenn sie Modelle hat
\item \redalert{allgemeingültig} (oder eine \alert{Tautologie}), wenn alle Wertzuweisungen Modelle für die Formel sind
\item \redalert{widerlegbar}, wenn sie nicht allgemeingültig ist
\end{itemize}
Diese Begriffe kann man für Mengen von Formeln genauso definieren.
}

\examplebox{Beispiel: 
\begin{itemize}
\item $p\wedge\neg p$ ist unerfüllbar (und widerlegbar)
\item $p\vee\neg p$ ist allgemeingültig (und erfüllbar)
\item $p\wedge q$ ist erfüllbar und widerlegbar
\end{itemize}}

\end{frame}

\begin{frame}\frametitle{Allgemeingültigkeit und Unerfüllbarkeit}

% Speziell Allgemeingültigkeit und Unerfüllbarkeit sind für logische Schlussfolgerung von Bedeutung:

\theobox{Satz:
\begin{enumerate}[(1)]
\item Eine allgemeingültige Formel ist logische Konsequenz jeder anderen Formel(menge).
\item Eine unerfüllbare Formel(menge) hat jede andere Formel als logische Konsequenz.
\end{enumerate}}

\emph{Beweis:} Folgt direkt aus Definition von logischer Konsequenz. Für (2) ist es wichtig, dass eine Eigenschaft "`für alle Modelle"' gilt, wenn es keine Modelle gibt (sie gilt dann für "`alle null Modelle"'). \qed
\bigskip\pause

\anybox{purple}{
Der unerwartete(?) Effekt (2) ist im Deutschen sprichwörtlich:\medskip

\narrowcentering{"`Wenn das stimmt, dann bin ich der Kaiser von China!"'}\medskip

drückt aus, dass aus einer mutmaßlich falschen Annahme alles folgt, selbst wenn 
es offensichtlich unwahr ist.
}

\end{frame}


\begin{frame}\frametitle{Zusammenfassung und Ausblick}

Mithilfe der \redalert{Aussagenlogik} kann man logische Beziehungen atomarer Aussagen spezifizieren
\bigskip

\redalert{Wertzuweisungen} können eine aussagenlogische Formel erfüllen -- dann nennt man sie \redalert{Modell} -- oder widerlegen
\bigskip

Die Modelle einer Formel(menge) definieren ihre \redalert{logischen Konsequenzen}
(="`alles, was in diesen Fällen noch gilt"')\bigskip

\anybox{yellow}{
Offene Fragen:
\begin{itemize}
\item Geht logisches Schließen auch ohne Wahrheitswertetabellen?
\item Wie (in)effizient ist logisches Schließen?
\item Was hat das mit Sprachen, Berechnung und TMs zu tun?
\end{itemize}
}

\end{frame}


\end{document}