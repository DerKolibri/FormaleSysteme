\documentclass[onlymath]{beamer}
% \documentclass[onlymath,handout]{beamer}

% Macros used by all lectures, but not necessarily by excercises

%%% General setup and dependencies:

% \usetheme[ddcfooter,nosectionnum]{tud}
\usetheme[nosectionnum,pagenum,noheader]{tud}
% \usetheme[nosectionnum,pagenum]{tud}

% Increase body font size to a sane level:
\let\origframetitle\frametitle
% \renewcommand{\frametitle}[1]{\origframetitle{#1}\normalsize}
\renewcommand{\frametitle}[1]{\origframetitle{#1}\fontsize{10pt}{13.2}\selectfont}
\setbeamerfont{itemize/enumerate subbody}{size=\small} % tud defaults to scriptsize!
\setbeamerfont{itemize/enumerate subsubbody}{size=\small}
% \setbeamerfont{normal text}{size=\small}
% \setbeamerfont{itemize body}{size=\small}

\renewcommand{\emph}[1]{\textbf{#1}}

\def\arraystretch{1.3}% Make tables even less cramped vertically

\usepackage[ngerman]{babel}
\usepackage[utf8]{inputenc}
\usepackage[T1]{fontenc}

%\usepackage{graphicx}
\usepackage[export]{adjustbox} % loads graphicx
\usepackage{import}
\usepackage{stmaryrd}
\usepackage[normalem]{ulem} % sout command
% \usepackage{times}
\usepackage{txfonts}

% \usepackage[perpage]{footmisc} % reset footnote counter on each page -- fails with beamer (footnotes gone)
\usepackage{perpage}  % reset footnote counter on each page
\MakePerPage{footnote}

\usepackage{tikz}
\usetikzlibrary{arrows,positioning}
% Inspired by http://www.texample.net/tikz/examples/hand-drawn-lines/
\usetikzlibrary{decorations.pathmorphing}
\pgfdeclaredecoration{penciline}{initial}{
    \state{initial}[width=+\pgfdecoratedinputsegmentremainingdistance,
    auto corner on length=1mm,]{
        \pgfpathcurveto%
        {% From
            \pgfqpoint{\pgfdecoratedinputsegmentremainingdistance}
                      {\pgfdecorationsegmentamplitude}
        }
        {%  Control 1
        \pgfmathrand
        \pgfpointadd{\pgfqpoint{\pgfdecoratedinputsegmentremainingdistance}{0pt}}
                    {\pgfqpoint{-\pgfdecorationsegmentaspect
                     \pgfdecoratedinputsegmentremainingdistance}%
                               {\pgfmathresult\pgfdecorationsegmentamplitude}
                    }
        }
        {%TO 
        \pgfpointadd{\pgfpointdecoratedinputsegmentlast}{\pgfpoint{1pt}{1pt}}
        }
    }
    \state{final}{}
}
\tikzset{handdrawn/.style={decorate,decoration=penciline}}
\tikzset{every shadow/.style={fill=none,shadow xshift=0pt,shadow yshift=0pt}}
% \tikzset{module/.append style={top color=\col,bottom color=\col}}

% Use to make Tikz attributes with Beamer overlays
% http://tex.stackexchange.com/a/6155
\tikzset{onslide/.code args={<#1>#2}{%
  \only<#1| handout:0>{\pgfkeysalso{#2}} 
}}
\tikzset{onslideprint/.code args={<#1>#2}{%
  \only<#1>{\pgfkeysalso{#2}} 
}}

%%% Title -- always set this first

\newcommand{\defineTitle}[3]{
	\newcommand{\lectureindex}{#1}
	\title{Formale Systeme}
	\subtitle{\href{\lectureurl}{#1. Vorlesung: #2}}
	\author{\href{http://korrekt.org/}{Markus Kr\"{o}tzsch}}
%	\author{\href{http://www.sebastian-rudolph.de}{Sebastian Rudolph} in Vertretung von \href{http://korrekt.org/}{Markus Kr\"{o}tzsch}}
	\date{#3}
	\datecity{TU Dresden}
% 	\institute{Computational Logic}
}

%%% Table of contents:

\RequirePackage{ifthen}

\newcommand{\highlight}[2]{%
	\ifthenelse{\equal{#1}{\lectureindex}}{\alert{#2}}{#2}%
}

\def\myspace{-0.7ex}
\newcommand{\printtoc}{
\begin{tabular}{r@{$\quad$}l}
\highlight{1}{1.} & \highlight{1}{Willkommen/Einleitung formale Sprachen}\\[\myspace]
\highlight{2}{2.} & \highlight{2}{Grammatiken und die Chomsky-Hierarchie}\\[\myspace]
\highlight{3}{3.} & \highlight{3}{Endliche Automaten}\\[\myspace]
\highlight{4}{4.} & \highlight{4}{Complexity of FO query answering}\\[\myspace]
\highlight{5}{5.} & \highlight{5}{Conjunctive queries}\\[\myspace]
\highlight{6}{6.} & \highlight{6}{Tree-like conjunctive queries}\\[\myspace]
\highlight{7}{7.} & \highlight{7}{Query optimisation}\\[\myspace]
\highlight{8}{8.} & \highlight{8}{Conjunctive Query Optimisation / First-Order~Expressiveness}\\[\myspace]
\highlight{9}{9.} & \highlight{9}{First-Order~Expressiveness / Introduction to Datalog}\\[\myspace]
\highlight{10}{10.} & \highlight{10}{Expressive Power and Complexity of Datalog}\\[\myspace]
\highlight{11}{11.} & \highlight{11}{Optimisation and Evaluation of Datalog}\\[\myspace]
\highlight{12}{12.} & \highlight{12}{Evaluation of Datalog (2)}\\[\myspace]
\highlight{13}{13.} & \highlight{13}{Graph Databases and Path Queries}\\[\myspace]
\highlight{14}{14.} & \highlight{14}{Outlook: database theory in practice}
\end{tabular}
}

\newcommand{\overviewslide}{%
\begin{frame}\frametitle{Overview}
\printtoc
\medskip

Siehe \href{\lectureurl}{course homepage [$\Rightarrow$ link]} for more information and materials
\end{frame}
}

%%% Colours:

\usepackage{xcolor,colortbl}
\definecolor{redhighlights}{HTML}{FFAA66}
\definecolor{lightblue}{HTML}{55AAFF}
\definecolor{lightred}{HTML}{FF5522}
\definecolor{lightpurple}{HTML}{DD77BB}
\definecolor{lightgreen}{HTML}{55FF55}
\definecolor{darkred}{HTML}{CC4411}
\definecolor{darkblue}{HTML}{176FC0}%{1133AA}
\definecolor{nightblue}{HTML}{2010A0}%{1133AA}
\definecolor{alert}{HTML}{176FC0}
\definecolor{darkgreen}{HTML}{36AB14}
\definecolor{strongyellow}{HTML}{FFE219}
\definecolor{devilscss}{HTML}{666666}

\newcommand{\redalert}[1]{\textcolor{darkred}{#1}}

%%% Style commands

\newcommand{\quoted}[1]{\texttt{"}{#1}\texttt{"}}
\newcommand{\squote}{\texttt{"}} % straight quote
\newcommand{\Sterm}[1]{\ensuremath{\mathtt{\textcolor{purple}{#1}}}}    % letters in alphabets
\newcommand{\Snterm}[1]{\textsf{\textcolor{darkblue}{#1}}} % nonterminal symbols
\newcommand{\Sntermsub}[2]{\Snterm{#1}_{\Snterm{#2}}} % nonterminal symbols
\newcommand{\Slang}[1]{\textbf{\textcolor{black}{#1}}}    % languages
\newcommand{\Slangsub}[2]{\Slang{#1}_{\Slang{#2}}}    % languages
% Code
\newcommand{\Scode}[1]{\textbf{#1}}    % reserved words in program listings, e.g., "if"
\newcommand{\Scodelit}[1]{\textcolor{purple}{#1}}    % literals in program listings, e.g., strings
\newcommand{\Scomment}[1]{\textcolor{gray}{#1}}    % comment in program listings

\newcommand{\epstrastar}{\mathrel{\mathord{\stackrel{\epsilon}{\to}}{}^*}} % transitive reflexive closure of epsilon transitions in an epslion-NFA

\newcommand{\narrowcentering}[1]{\mbox{}\hfill#1\hfill\mbox{}}

\newcommand{\defeq}{\mathrel{:=}}

\newcommand{\Smach}[1]{\ensuremath{\mathcal{#1}}}    % machines

%%% Slide layout commands:

\newcommand{\sectionSlide}[1]{
\frame{\begin{center}
\LARGE
#1
\end{center}}
}
\newcommand{\sectionSlideNoHandout}[1]{
\frame<handout:0>{\begin{center}
\LARGE
#1
\end{center}}
}

\newcommand{\mydualbox}[3]{%
 \begin{minipage}[t]{#1}
 \begin{beamerboxesrounded}[upper=block title,lower=block body,shadow=true]%
    {\centering\usebeamerfont*{block title}#2}%
    \raggedright%
    \usebeamerfont{block body}
%     \small
    #3%
  \end{beamerboxesrounded}
  \end{minipage}
}
% 
\newcommand{\myheaderbox}[2]{%
 \begin{minipage}[t]{#1}
 \begin{beamerboxesrounded}[upper=block title,lower=block title,shadow=true]%
    {\centering\usebeamerfont*{block title}\rule{0pt}{2.6ex} #2}%
  \end{beamerboxesrounded}
  \end{minipage}
}

\newcommand{\mycontentbox}[2]{%
 \begin{minipage}[t]{#1}%
 \begin{beamerboxesrounded}[upper=block body,lower=block body,shadow=true]%
    {\centering\usebeamerfont*{block body}\rule{0pt}{2.6ex}#2}%
  \end{beamerboxesrounded}
  \end{minipage}
}

\newcommand{\mylcontentbox}[2]{%
 \begin{minipage}[t]{#1}%
 \begin{beamerboxesrounded}[upper=block body,lower=block body,shadow=true]%
    {\flushleft\usebeamerfont*{block body}\rule{0pt}{2.6ex}#2}%
  \end{beamerboxesrounded}
  \end{minipage}
}

% label=180:{\rotatebox{90}{{\footnotesize\textcolor{darkgreen}{Beispiel}}}}
% \hspace{-8mm}\ghost{\raisebox{-7mm}{\rotatebox{90}{{\footnotesize\textcolor{darkgreen}{Beispiel}}}}}\hspace{8mm}
\newcommand{\examplebox}[1]{%
	\begin{tikzpicture}[decoration=penciline, decorate]
		\pgfmathsetseed{1235}
		\node (n1) [decorate,draw=darkgreen, fill=darkgreen!10,thick,align=left,text width=\linewidth, inner ysep=2mm, inner xsep=2mm] at (0,0) {#1};
% 		\node (n2) [align=left,text width=\linewidth,inner sep=0mm] at (n1.92) {{\footnotesize\raisebox{3mm}{\textcolor{darkgreen}{Beispiel}}}};
% 		\node (n2) [decorate,draw=darkgreen, fill=darkgreen!10,thick, align=left,text width=\linewidth,inner sep=2mm] at (n1.90) {{\footnotesize\raisebox{0mm}{\textcolor{darkgreen}{Beispiel}}}};
	\end{tikzpicture}%
}%

\newcommand{\codebox}[1]{%
	\begin{tikzpicture}[decoration=penciline, decorate]
		\pgfmathsetseed{1236}
		\node (n1) [decorate,draw=strongyellow, fill=strongyellow!10,thick,align=left,text width=\linewidth, inner ysep=2mm, inner xsep=2mm] at (0,0) {#1};
	\end{tikzpicture}%
}%

\newcommand{\defbox}[1]{%
	\begin{tikzpicture}[decoration=penciline, decorate]
		\pgfmathsetseed{1237}
		\node (n1) [decorate,draw=darkred, fill=darkred!10,thick,align=left,text width=\linewidth, inner ysep=2mm, inner xsep=2mm] at (0,0) {#1};
	\end{tikzpicture}%
}%

\newcommand{\theobox}[1]{%
	\begin{tikzpicture}[decoration=penciline, decorate]
		\pgfmathsetseed{1240}
		\node (n1) [decorate,draw=darkblue, fill=darkblue!10,thick,align=left,text width=\linewidth, inner ysep=2mm, inner xsep=2mm] at (0,0) {#1};
	\end{tikzpicture}%
}%

\newcommand{\anybox}[2]{%
	\begin{tikzpicture}[decoration=penciline, decorate]
		\pgfmathsetseed{1240}
		\node (n1) [decorate,draw=#1, fill=#1!10,thick,align=left,text width=\linewidth, inner ysep=2mm, inner xsep=2mm] at (0,0) {#2};
	\end{tikzpicture}%
}%


\newsavebox{\mybox}%
\newcommand{\doodlebox}[2]{%
\sbox{\mybox}{#2}%
	\begin{tikzpicture}[decoration=penciline, decorate]
		\pgfmathsetseed{1238}
		\node (n1) [decorate,draw=#1, fill=#1!10,thick,align=left,inner sep=1mm] at (0,0) {\usebox{\mybox}};
	\end{tikzpicture}%
}%

% Common notation

\usepackage{amsmath,amssymb,amsfonts}
\usepackage{xspace}

\newcommand{\lectureurl}{https://iccl.inf.tu-dresden.de/web/FS2016}

\DeclareMathAlphabet{\mathsc}{OT1}{cmr}{m}{sc} % Let's have \mathsc since the slide style has no working \textsc

% Dual of "phantom": make a text that is visible but intangible
\newcommand{\ghost}[1]{\raisebox{0pt}[0pt][0pt]{\makebox[0pt][l]{#1}}}

\newcommand{\tuple}[1]{\langle{#1}\rangle}

%%% Annotation %%%

\usepackage{color}
\newcommand{\todo}[1]{{\tiny\color{red}\textbf{TODO: #1}}}



%%% Old macros below; move when needed

\newcommand{\blank}{\text{\textvisiblespace}} % empty tape cell for TM

% table syntax
\newcommand{\dom}{\textbf{dom}}
\newcommand{\adom}{\textbf{adom}}
\newcommand{\dbconst}[1]{\texttt{"#1"}}
\newcommand{\pred}[1]{\textsf{#1}}
\newcommand{\foquery}[2]{#2[#1]}
\newcommand{\ground}[1]{\textsf{ground}(#1)}
% \newcommand{\foquery}[2]{\{#1\mid #2\}} %% Notation as used in Alice Book
% \newcommand{\foquery}[2]{\tuple{#1\mid #2}}

\newcommand{\quantor}{\mathord{\reflectbox{$\text{\sf{Q}}$}}} % the generic quantor

% logic syntax
\newcommand{\Inter}{\mathcal{I}} %used to denote an interpretation
\newcommand{\Jnter}{\mathcal{J}} %used to denote another interpretation
\newcommand{\Knter}{\mathcal{K}} %used to denote yet another interpretation
\newcommand{\Zuweisung}{\mathcal{Z}} %used to denote a variable assignment

% query languages
\newcommand{\qlang}[1]{{\sf #1}} % Font for query languages
\newcommand{\qmaps}[1]{\textbf{QM}({\sf #1})} % Set of query mappings for a query language

%%% Complexities %%%

\hyphenation{Exp-Time} % prevent "Ex-PTime" (see, e.g. Tobies'01, Glimm'07 ;-)
\hyphenation{NExp-Time} % better that than something else

% \newcommand{\complclass}[1]{{\sc #1}\xspace} % font for complexity classes
\newcommand{\complclass}[1]{\ensuremath{\mathsc{#1}}\xspace} % font for complexity classes

\newcommand{\ACzero}{\complclass{AC$_0$}}
\newcommand{\LogSpace}{\complclass{L}}
\newcommand{\NLogSpace}{\complclass{NL}}
\newcommand{\PTime}{\complclass{P}}
\newcommand{\NP}{\complclass{NP}}
\newcommand{\coNP}{\complclass{coNP}}
\newcommand{\PH}{\complclass{PH}}
\newcommand{\PSpace}{\complclass{PSpace}}
\newcommand{\NPSpace}{\complclass{NPSpace}}
\newcommand{\ExpTime}{\complclass{ExpTime}}
\newcommand{\NExpTime}{\complclass{NExpTime}}
\newcommand{\ExpSpace}{\complclass{ExpSpace}}
\newcommand{\TwoExpTime}{\complclass{2ExpTime}}
\newcommand{\NTwoExpTime}{\complclass{N2ExpTime}}
\newcommand{\ThreeExpTime}{\complclass{3ExpTime}}
\newcommand{\kExpTime}[1]{\complclass{#1ExpTime}}
\newcommand{\kExpSpace}[1]{\complclass{#1ExpSpace}}


\defineTitle{14}{Abschlusseigenschaften kontextfreier Sprachen}{28. November 2016}

\begin{document}

\maketitle

\sectionSlide{Rückblick: Das Pumping Lemma}

\begin{frame}\frametitle{Pumpen für kontextfreie Sprachen}

\theobox{Satz (Pumping Lemma):
Für jede kontextfreie Sprache $\Slang{L}$\\
gibt es eine Zahl $n\geq 0$, so dass gilt:\\
~~für jedes Wort $z\in\Slang{L}$ mit $|z|\geq n$\\
~~gibt es eine Zerlegung $z=uvwxy$ mit $|vx|\geq 1$ und $|vwx|\leq n$, s.d.:\\
~~~~für jede Zahl $k\geq 0$ gilt: $u v^k w x^k y\in\Slang{L}$
}\medskip

\examplebox{Beispiel: Für die Sprache $\{\Sterm{a}^i\Sterm{b}^i\mid i\geq 0\}$ gilt der Satz. Wir wählen $n=2$.
Sei $z=\Sterm{a}^i\Sterm{b}^i$ mit $i\geq 1$ ein beliebiges Wort mit $|z|\geq 2$. Wir wählen die Zerlegung $u=\Sterm{a}^{i-1}$,
$v=\Sterm{a}$, $w=\epsilon$, $x=\Sterm{b}$ und $y=\Sterm{b}^{i-1}$.\\
Dann ist $u v^k w x^k y  = \Sterm{a}^{i-1}\Sterm{a}^{k}\Sterm{b}^{k}\Sterm{b}^{i-1}=\Sterm{a}^{i+k-1}\Sterm{b}^{i+k-1}\in\Slang{L}$ für alle $k\geq 0$.
}

\end{frame}


\begin{frame}[t]\frametitle{Ableitungsbäume aufpumpen}

\vspace{-1ex}
\ghost{\hspace{7.5cm}\raisebox{-6.5cm}{Abgeleitetes Wort:}}%
\ghost{\hspace{7.5cm}\raisebox{-6.9cm}{$uv^kwx^ky$}}%
Die Idee des Lemmas lässt sich gut am Ableitungsbaum darstellen:\\[1ex]

\narrowcentering{%
\scalebox{0.65}{
\begin{tikzpicture}[
	scale=1.50,
	decoration=penciline, decorate
]
% \path[use as bounding box] (-3.2,0) rectangle (3.5,-5); % add "draw" to see it
\pgfmathsetseed{6571}
% \draw[help lines] (0,0) grid (5,5);
\node (s) [circle,draw=none,inner sep=1pt] at (0,0) {\Snterm{S}};
% 
\draw[fill=none,decorate,line width=0.3mm]
(s.270) -- (-1.5,-3.5) -- (1.5,-3.5) -- cycle;
%
\node (a12) [circle,draw=none,inner sep=0pt] at (0,-1.5) {\Snterm{A}};
%
\draw[decorate,dashed,line width=0.2mm]
(s.270) -- (-0.1,-0.8) -- (a12.90);
%
\draw[fill=darkgreen!30,decorate,line width=0.3mm]
(a12.270) -- (-1.0,-3.50) -- (1.0,-3.52) -- cycle;
%
\node (a11) [circle,draw=none,inner sep=0pt] at (0,-2.5) {\Snterm{A}};
%
\draw[decorate,dashed,line width=0.2mm]
(a12.270) -- (a11.90);
%
\node (a1u) [circle,draw=none,inner sep=0pt] at (0,-3.5) {\phantom{\Snterm{A}}};
%
\draw[fill=darkgreen!30,decorate,line width=0.3mm]
(a11.270) -- (-1.0,-4.50)-- (-0.5,-4.50)-- (a1u.270)-- (0.5,-4.50) -- (1.0,-4.52) -- cycle;
\node (a1utop) [circle,draw=none,inner sep=0pt] at (0,-3.5) {\Snterm{A}};
%
\draw[decorate,dashed,line width=0.2mm]
(a11.270) -- (a1u.90);
%
\node (dots) [circle,draw=none,inner sep=0pt] at (0,-4.0) {$\vdots$};
%
\node (a1) [circle,draw=none,inner sep=0pt] at (0,-4.5) {\Snterm{A}};
%
% \draw[decorate,dashed,line width=0.2mm]
% (0,-2.7) -- (a1.90);
%
\draw[fill=darkgreen!30,decorate,line width=0.3mm]
(a1.270) -- (-1.0,-6.50) -- (1.0,-6.52) -- cycle;
%
\node (a2) [circle,draw=none,inner sep=0pt] at (0,-5.5) {\Snterm{A}};
%
\draw[decorate,dashed,line width=0.2mm]
(a1.270) -- (a2.90);
%
\draw[fill=darkred!30,decorate,line width=0.3mm]
(a2.270) -- (-0.5,-6.51) -- (0.5,-6.54) -- cycle;
%
\draw [thick, devilscss,decorate,decoration={brace,amplitude=3pt,mirror},xshift=0.4pt,yshift=-1mm](-1.5,-3.5) -- (-1.0,-3.5) node[black,midway,yshift=-0.25cm] {$u$};
\draw [thick, devilscss,decorate,decoration={brace,amplitude=3pt,mirror},xshift=0.4pt,yshift=-1.3mm](-1.0,-3.5) -- (-0.5,-3.5) node[black,midway,yshift=-0.28cm] {$v$};
\draw [thick, devilscss,decorate,decoration={brace,amplitude=3pt,mirror},xshift=0.4pt,yshift=-1mm](0.5,-3.5) -- (1,-3.5) node[black,midway,yshift=-0.25cm] {$x$};
\draw [thick, devilscss,decorate,decoration={brace,amplitude=3pt,mirror},xshift=0.4pt,yshift=-0.7mm](1.0,-3.5) -- (1.5,-3.5) node[black,midway,yshift=-0.26cm] {$y$};

\draw [thick, devilscss,decorate,decoration={brace,amplitude=3pt,mirror},xshift=0.4pt,yshift=-1.3mm](-1.0,-4.5) -- (-0.5,-4.5) node[black,midway,yshift=-0.28cm] {$v$};
\draw [thick, devilscss,decorate,decoration={brace,amplitude=3pt,mirror},xshift=0.4pt,yshift=-1mm](0.5,-4.5) -- (1,-4.5) node[black,midway,yshift=-0.25cm] {$x$};

\draw [thick, devilscss,decorate,decoration={brace,amplitude=3pt,mirror},xshift=0.4pt,yshift=-1.3mm](-1.0,-6.5) -- (-0.5,-6.5) node[black,midway,yshift=-0.28cm] {$v$};
\draw [thick, devilscss,decorate,decoration={brace,amplitude=3pt,mirror},xshift=0.4pt,yshift=-1mm](0.5,-6.5) -- (1,-6.5) node[black,midway,yshift=-0.25cm] {$x$};
\draw [thick, devilscss,decorate,decoration={brace,amplitude=3pt,mirror},xshift=0.4pt,yshift=-1.5mm](-0.5,-6.5) -- (0.5,-6.5) node[black,midway,yshift=-0.3cm] {$w$};
%
% \draw [thick, alert,decorate,decoration={brace,amplitude=10pt,mirror},xshift=0.4pt,yshift=-0.4cm](-1.5,-3.5) -- (1.5,-3.5) node[black,midway,yshift=-0.55cm] {\alert{abgeleitetes Wort}};
\end{tikzpicture}}}

\end{frame}

%% Frage eines Studenten: Warum ist |xv| nicht sogar > 1?

\begin{frame}\frametitle{Beispiel}

\begin{minipage}[t]{4.8cm}
CNF-Grammatik für
$\{\Sterm{a}^i\Sterm{b}^i\mid i\geq 1\}$:
\begin{align*}
\Snterm{S} & \to \Snterm{A}\Snterm{B}\mid \Snterm{A}\Snterm{C}\\
\Snterm{A} & \to \Sterm{a}\\
\Snterm{B} & \to \Sterm{b}\\
\Snterm{C} & \to \Snterm{S}\Snterm{B}
\end{align*}%
Konstante aus dem Beweis des Pumping-Lemma:
$n=2^4=16$
\medskip

Aber man kann auch schon ab\\$n=4$ pumpen
\bigskip


\end{minipage}\pause~\hfill
\begin{minipage}[t]{4.5cm}
Beispiel: Ableitung für $\Sterm{aaabbb}$\\[1ex]
\begin{tikzpicture}[decoration=penciline, decorate,xscale=0.8,yscale=0.9,baseline={(current bounding box.center)}]
% \draw[help lines] (0,0) grid (7,2);
\pgfmathsetseed{7729}

\visible<4->{
\draw[fill=darkgreen!30,decorate,draw=none]
(2,-1.5) -- (0.7,-2.7) -- (0.5,-6.3) -- (4.5,-6.3)-- (4.3,-3.7) -- (3.5,-2.7) -- cycle;
\draw[fill=darkred!30,decorate,draw=none]
(2.5,-3.5) -- (1.7,-4.7) -- (1.5,-6.3) -- (3.5,-6.3) -- (3.3,-4.7) -- cycle;
}
\visible<5->{
\draw [thick, devilscss,decorate,decoration={brace,amplitude=3pt,mirror},xshift=0.4pt,yshift=-0.7mm](-0.5,-6.3) -- (0.5,-6.3) node[black,midway,yshift=-0.26cm] {$u$};
\draw [thick, devilscss,decorate,decoration={brace,amplitude=3pt,mirror},xshift=0.4pt,yshift=-0.7mm](0.5,-6.3) -- (1.5,-6.3) node[black,midway,yshift=-0.26cm] {$v$};
\draw [thick, devilscss,decorate,decoration={brace,amplitude=3pt,mirror},xshift=0.4pt,yshift=-0.7mm](1.5,-6.3) -- (3.5,-6.3) node[black,midway,yshift=-0.26cm] {$w$};
\draw [thick, devilscss,decorate,decoration={brace,amplitude=3pt,mirror},xshift=0.4pt,yshift=-0.7mm](3.5,-6.3) -- (4.5,-6.3) node[black,midway,yshift=-0.26cm] {$x$};
\draw [thick, devilscss,decorate,decoration={brace,amplitude=3pt,mirror},xshift=0.4pt,yshift=-0.7mm](4.5,-6.3) -- (5.5,-6.3) node[black,midway,yshift=-0.26cm] {$y$};
}

{\node (a1) [circle,draw=none,inner sep=1pt] at (0,-6) {\Sterm{a}};}
{\node (a2) [circle,draw=none,inner sep=1pt] at (1,-6) {\Sterm{a}};}
{\node (a3) [circle,draw=none,inner sep=1pt] at (2,-6) {\Sterm{a}};}
{\node (b1) [circle,draw=none,inner sep=1pt] at (3,-6) {\Sterm{b}};}
{\node (b2) [circle,draw=none,inner sep=1pt] at (4,-6) {\Sterm{b}};}
{\node (b3) [circle,draw=none,inner sep=1pt] at (5,-6) {\Sterm{b}};}

{\node (A1) [circle,draw=none,inner sep=1pt] at (0,-1) {\Snterm{A}};}
{\node (A2) [circle,draw=none,inner sep=1pt] at (1,-3) {\Snterm{A}};}
{\node (A3) [circle,draw=none,inner sep=1pt] at (2,-5) {\Snterm{A}};}
{\node (B1) [circle,draw=none,inner sep=1pt] at (3,-5) {\Snterm{B}};}
{\node (B2) [circle,draw=none,inner sep=1pt] at (4,-4) {\Snterm{B}};}
{\node (B3) [circle,draw=none,inner sep=1pt] at (5,-2) {\Snterm{B}};}

{\node (SAB) [circle,draw=none,onslideprint={<3->{draw=black,dotted,line width=0.4mm}},inner sep=1pt] at (2.5,-4) {\Snterm{S}};}
{\node (CSB2) [circle,draw=none,inner sep=1pt] at (3.25,-3) {\Snterm{C}};}
{\node (SAC2) [circle,draw=none,onslideprint={<3->{draw=black,dotted,line width=0.4mm}},inner sep=1pt] at (2,-2) {\Snterm{S}};}
{\node (CSB1) [circle,draw=none,inner sep=1pt] at (3.5,-1) {\Snterm{C}};}
{\node (SAC1) [circle,draw=none,inner sep=1pt] at (1.75,0) {\Snterm{S}};}

\path[-,line width=0.3mm,onslideprint={<3->{dotted,line width=0.4mm}}](SAC1) edge (CSB1);
\path[-,line width=0.3mm](SAC1) edge (A1);
\path[-,line width=0.3mm,onslideprint={<3->{dotted,line width=0.4mm}}](CSB1) edge (SAC2);
\path[-,line width=0.3mm](CSB1) edge (B3);
\path[-,line width=0.3mm,onslideprint={<3->{dotted,line width=0.4mm}}](SAC2) edge (CSB2);
\path[-,line width=0.3mm](SAC2) edge (A2);
\path[-,line width=0.3mm,onslideprint={<3->{dotted,line width=0.4mm}}](CSB2) edge (SAB);
\path[-,line width=0.3mm](CSB2) edge (B2);
\path[-,line width=0.3mm](SAB) edge (A3);
\path[-,line width=0.3mm](SAB) edge (B1);

\path[-,line width=0.3mm](A1) edge (a1);
\path[-,line width=0.3mm](A2) edge (a2);
\path[-,line width=0.3mm](A3) edge (a3);
\path[-,line width=0.3mm](B1) edge (b1);
\path[-,line width=0.3mm](B2) edge (b2);
\path[-,line width=0.3mm](B3) edge (b3);
\end{tikzpicture}
\end{minipage}

\end{frame}

\begin{frame}\frametitle{Beispiel (2)}

\begin{minipage}[t]{4.8cm}
CNF-Grammatik für
$\{\Sterm{a}^i\Sterm{b}^i\mid i\geq 1\}$:
\begin{align*}
\Snterm{S} & \to \Snterm{A}\Snterm{B}\mid \Snterm{A}\Snterm{C}\\
\Snterm{A} & \to \Sterm{a}\\
\Snterm{B} & \to \Sterm{b}\\
\Snterm{C} & \to \Snterm{S}\Snterm{B}
\end{align*}%
Konstante aus dem Beweis des Pumping-Lemma:
$n=2^4=16$
\medskip

Aber man kann auch schon ab\\$n=4$ pumpen
\bigskip


\end{minipage}~~\hfill
\begin{minipage}[t]{4.5cm}
Beispiel: Ableitung für $\Sterm{aaabbb}$\\[1ex]
\begin{tikzpicture}[decoration=penciline, decorate,xscale=0.8,yscale=0.9,baseline={(current bounding box.center)}]
% \draw[help lines] (0,0) grid (7,2);

\visible<3->{
\draw[fill=darkgreen!30,decorate,draw=none]
(3.5,-0.5) -- (0.7,-2.5) -- (0.5,-6.3) -- (5.5,-6.3)-- (5.3,-1.7) -- cycle;
\draw[fill=darkred!30,decorate,draw=none]
(3.25,-2.5) -- (1.7,-4.5) -- (1.5,-6.3) -- (4.5,-6.3) -- (4.3,-3.7) -- cycle;

\draw [thick, devilscss,decorate,decoration={brace,amplitude=3pt,mirror},xshift=0.4pt,yshift=-0.7mm](-0.5,-6.3) -- (0.5,-6.3) node[black,midway,yshift=-0.26cm] {$u$};
\draw [thick, devilscss,decorate,decoration={brace,amplitude=3pt,mirror},xshift=0.4pt,yshift=-0.7mm](0.5,-6.3) -- (1.5,-6.3) node[black,midway,yshift=-0.26cm] {$v$};
\draw [thick, devilscss,decorate,decoration={brace,amplitude=3pt,mirror},xshift=0.4pt,yshift=-0.7mm](1.5,-6.3) -- (4.5,-6.3) node[black,midway,yshift=-0.26cm] {$w$};
\draw [thick, devilscss,decorate,decoration={brace,amplitude=3pt,mirror},xshift=0.4pt,yshift=-0.7mm](4.5,-6.3) -- (5.5,-6.3) node[black,midway,yshift=-0.26cm] {$x$};
% \draw [thick, devilscss,decorate,decoration={brace,amplitude=3pt,mirror},xshift=0.4pt,yshift=-0.7mm](4.5,-6.3) -- (5.5,-6.3) node[black,midway,yshift=-0.26cm] {$y$};
}

{\node (a1) [circle,draw=none,inner sep=1pt] at (0,-6) {\Sterm{a}};}
{\node (a2) [circle,draw=none,inner sep=1pt] at (1,-6) {\Sterm{a}};}
{\node (a3) [circle,draw=none,inner sep=1pt] at (2,-6) {\Sterm{a}};}
{\node (b1) [circle,draw=none,inner sep=1pt] at (3,-6) {\Sterm{b}};}
{\node (b2) [circle,draw=none,inner sep=1pt] at (4,-6) {\Sterm{b}};}
{\node (b3) [circle,draw=none,inner sep=1pt] at (5,-6) {\Sterm{b}};}

{\node (A1) [circle,draw=none,inner sep=1pt] at (0,-1) {\Snterm{A}};}
{\node (A2) [circle,draw=none,inner sep=1pt] at (1,-3) {\Snterm{A}};}
{\node (A3) [circle,draw=none,inner sep=1pt] at (2,-5) {\Snterm{A}};}
{\node (B1) [circle,draw=none,inner sep=1pt] at (3,-5) {\Snterm{B}};}
{\node (B2) [circle,draw=none,inner sep=1pt] at (4,-4) {\Snterm{B}};}
{\node (B3) [circle,draw=none,inner sep=1pt] at (5,-2) {\Snterm{B}};}

{\node (SAB) [circle,draw=none,inner sep=1pt] at (2.5,-4) {\Snterm{S}};}
{\node (CSB2) [circle,draw=none,inner sep=1pt,onslideprint={<2->{draw=black,dotted,line width=0.4mm}}] at (3.25,-3) {\Snterm{C}};}
{\node (SAC2) [circle,draw=none,inner sep=1pt] at (2,-2) {\Snterm{S}};}
{\node (CSB1) [circle,draw=none,inner sep=1pt,onslideprint={<2->{draw=black,dotted,line width=0.4mm}}] at (3.5,-1) {\Snterm{C}};}
{\node (SAC1) [circle,draw=none,inner sep=1pt] at (1.75,0) {\Snterm{S}};}

\path[-,line width=0.3mm,onslideprint={<2->{dotted,line width=0.4mm}}](SAC1) edge (CSB1);
\path[-,line width=0.3mm](SAC1) edge (A1);
\path[-,line width=0.3mm,onslideprint={<2->{dotted,line width=0.4mm}}](CSB1) edge (SAC2);
\path[-,line width=0.3mm](CSB1) edge (B3);
\path[-,line width=0.3mm,onslideprint={<2->{dotted,line width=0.4mm}}](SAC2) edge (CSB2);
\path[-,line width=0.3mm](SAC2) edge (A2);
\path[-,line width=0.3mm](CSB2) edge (SAB);
\path[-,line width=0.3mm](CSB2) edge (B2);
\path[-,line width=0.3mm](SAB) edge (A3);
\path[-,line width=0.3mm](SAB) edge (B1);

\path[-,line width=0.3mm](A1) edge (a1);
\path[-,line width=0.3mm](A2) edge (a2);
\path[-,line width=0.3mm](A3) edge (a3);
\path[-,line width=0.3mm](B1) edge (b1);
\path[-,line width=0.3mm](B2) edge (b2);
\path[-,line width=0.3mm](B3) edge (b3);
\end{tikzpicture}
\end{minipage}

\end{frame}

\begin{frame}\frametitle{Beispiel (3)}

\begin{minipage}[t]{4.8cm}
CNF-Grammatik für
$\Sterm{a}\Sterm{b}^+\Sterm{a}$:
\begin{align*}
\Snterm{S} & \to \Snterm{A}\Snterm{B}\\
\Snterm{A} & \to \Sterm{a}\\
\Snterm{B} & \to \Snterm{C}\Snterm{A}\\
\Snterm{C} & \to \Snterm{C}\Snterm{C}\mid \Sterm{b}
\end{align*}%
Konstante aus dem Beweis des Pumping-Lemma:
$n=2^4=16$
\medskip

Aber man kann auch schon ab\\$n=4$ pumpen
\bigskip


\end{minipage}~~\hfill
\begin{minipage}[t]{4.5cm}
Beispiel: Ableitung für $\Sterm{abba}$\\[1ex]
\narrowcentering{%
\begin{tikzpicture}[decoration=penciline, decorate,xscale=0.8,yscale=0.9,baseline={(current bounding box.center)}]
% \draw[help lines] (0,0) grid (7,2);

\visible<3->{
\draw[fill=darkgreen!30,decorate,draw=none]
(1.5,-3.5) -- (0.5,-5) -- (0.7,-6.3) -- (2.5,-6.3)-- (2.3,-4.5) -- cycle;
\draw[fill=darkred!30,decorate,draw=none]
(1,-4.5) -- (0.5,-5) -- (0.5,-6.3) -- (1.5,-6.3) -- (1.5,-5) -- cycle;
% 
\draw [thick, devilscss,decorate,decoration={brace,amplitude=3pt,mirror},xshift=0.4pt,yshift=-0.7mm](-0.5,-6.3) -- (0.5,-6.3) node[black,midway,yshift=-0.26cm] {$u$};
% \draw [thick, devilscss,decorate,decoration={brace,amplitude=3pt,mirror},xshift=0.4pt,yshift=-0.7mm](0.5,-6.3) -- (1.5,-6.3) node[black,midway,yshift=-0.26cm] {$v$};
\draw [thick, devilscss,decorate,decoration={brace,amplitude=3pt,mirror},xshift=0.4pt,yshift=-0.7mm](0.5,-6.3) -- (1.5,-6.3) node[black,midway,yshift=-0.26cm] {$w$};
\draw [thick, devilscss,decorate,decoration={brace,amplitude=3pt,mirror},xshift=0.4pt,yshift=-0.7mm](1.5,-6.3) -- (2.5,-6.3) node[black,midway,yshift=-0.26cm] {$x$};
\draw [thick, devilscss,decorate,decoration={brace,amplitude=3pt,mirror},xshift=0.4pt,yshift=-0.7mm](2.5,-6.3) -- (3.5,-6.3) node[black,midway,yshift=-0.26cm] {$y$};
{\node (v) [circle,draw=none,inner sep=1pt] at (1.5,-7.5) {$v=\epsilon$};}
}

{\node (a1) [circle,draw=none,inner sep=1pt] at (0,-6) {\Sterm{a}};}
{\node (b2) [circle,draw=none,inner sep=1pt] at (1,-6) {\Sterm{b}};}
{\node (b3) [circle,draw=none,inner sep=1pt] at (2,-6) {\Sterm{b}};}
{\node (a4) [circle,draw=none,inner sep=1pt] at (3,-6) {\Sterm{a}};}

{\node (A1) [circle,draw=none,inner sep=1pt] at (0,-5) {\Snterm{A}};}
{\node (C2) [circle,draw=none,inner sep=1pt,onslideprint={<2->{draw=black,dotted,line width=0.4mm}}] at (1,-5) {\Snterm{C}};}
{\node (C3) [circle,draw=none,inner sep=1pt] at (2,-5) {\Snterm{C}};}
{\node (A4) [circle,draw=none,inner sep=1pt] at (3,-5) {\Snterm{A}};}

{\node (CCC) [circle,draw=none,inner sep=1pt,onslideprint={<2->{draw=black,dotted,line width=0.4mm}}] at (1.5,-4) {\Snterm{C}};}
{\node (B) [circle,draw=none,inner sep=1pt] at (2.25,-3) {\Snterm{B}};}
{\node (S) [circle,draw=none,inner sep=1pt] at (1.125,-2) {\Snterm{S}};}

\path[-,line width=0.3mm,onslideprint={<2->{dotted,line width=0.4mm}}](S) edge (B);
\path[-,line width=0.3mm](S) edge (A1);
\path[-,line width=0.3mm,onslideprint={<2->{dotted,line width=0.4mm}}](B) edge (CCC);
\path[-,line width=0.3mm](B) edge (A4);
\path[-,line width=0.3mm,onslideprint={<2->{dotted,line width=0.4mm}}](CCC) edge (C2);
\path[-,line width=0.3mm](CCC) edge (C3);

\path[-,line width=0.3mm](A1) edge (a1);
\path[-,line width=0.3mm](C2) edge (b2);
\path[-,line width=0.3mm](C3) edge (b3);
\path[-,line width=0.3mm](A4) edge (a4);
\end{tikzpicture}}
\end{minipage}

\end{frame}

\sectionSlide{Abschlusseigenschaften kontextfreier Sprachen}

\newcommand{\mytabnote}[2]{\ghost{#1}\hspace{2cm}{\textcolor{devilscss}{(#2)}}}
\newcommand{\myhitabnote}[2]{\ghost{#1}\hspace{2cm}{\textcolor{darkred}{#2}}}
\begin{frame}\frametitle{Rückblick: Reguläre Sprachen}

\theobox{
Satz: Wenn $\Slang{L}$, $\Slangsub{L}{1}$ und $\Slangsub{L}{2}$ reguläre Sprachen sind, dann beschreiben auch die
folgenden Ausdrücke reguläre Sprachen:
\begin{enumerate}[(1)]
\item \mytabnote{$\Slangsub{L}{1}\cup\Slangsub{L}{2}$}{Abschluss unter Vereinigung}
\item \mytabnote{$\Slangsub{L}{1}\cap\Slangsub{L}{2}$}{Abschluss unter Schnitt}
\item \mytabnote{$\overline{\Slang{L}}$}{Abschluss unter Komplement}
\item \mytabnote{$\Slangsub{L}{1}\circ\Slangsub{L}{2}$}{Abschluss unter Konkatenation}
\item \mytabnote{$\Slang{L}^*$}{Abschluss unter Kleene-Stern}
\end{enumerate}
}
\bigskip

\narrowcentering{\alert{Wie sieht es bei den kontextfreien Sprachen aus?}}

\end{frame}

\begin{frame}\frametitle{Abschluss für kontextfreie Sprachen?}

Bei kontextfreien Sprachen ergibt sich ein anderes Bild:

\theobox{
Satz: Wenn $\Slang{L}$, $\Slangsub{L}{1}$ und $\Slangsub{L}{2}$ kontextfreie Sprachen sind, dann beschreiben auch die
folgenden Ausdrücke kontextfreie \ghost{Sprachen:}%
\begin{enumerate}[(1)]%
\item \mytabnote{$\Slangsub{L}{1}\cup\Slangsub{L}{2}$}{Abschluss unter Vereinigung}
\item \mytabnote{$\Slangsub{L}{1}\circ\Slangsub{L}{2}$}{Abschluss unter Konkatenation}
\item \mytabnote{$\Slang{L}^*$}{Abschluss unter Kleene-Stern}
\end{enumerate}
}

Aber: 
\theobox{
Satz: Es gibt kontextfreie Sprachen $\Slang{L}$, $\Slangsub{L}{1}$ und $\Slangsub{L}{2}$, so dass die folgenden
Ausdrücke keine kontextfreien Sprachen sind:
\begin{enumerate}[(1)]
\item \mytabnote{$\Slangsub{L}{1}\cap\Slangsub{L}{2}$}{Nichtabschluss unter Schnitt}
\item \mytabnote{$\overline{\Slang{L}}$}{Nichtabschluss unter Komplement}
\end{enumerate}
}

\end{frame}

% \begin{frame}\frametitle{Beweis der Abschlusseigenschaften}
% 
% \emph{Beweistechnik für reguläre Sprachen:}
% durch Angabe entsprechender Operationen auf endlichen Automaten
% \bigskip
% 
% \emph{Beweistechnik für kontextfreie Sprachen:}
% durch Angabe entsprechender Operationen auf formalen Grammatiken
% 
% \end{frame}

\begin{frame}\frametitle{Abschlusseigenschaften von Typ-2-Sprachen}

\theobox{
Satz: Wenn $\Slang{L}$, $\Slangsub{L}{1}$ und $\Slangsub{L}{2}$ kontextfreie Sprachen sind, dann beschreiben auch die
folgenden Ausdrücke kontextfreie \ghost{Sprachen:}%
\begin{enumerate}[(1)]%
\item \myhitabnote{$\Slangsub{L}{1}\cup\Slangsub{L}{2}$}{Abschluss unter Vereinigung}
\item \mytabnote{$\Slangsub{L}{1}\circ\Slangsub{L}{2}$}{Abschluss unter Konkatenation}
\item \mytabnote{$\Slang{L}^*$}{Abschluss unter Kleene-Stern}
\end{enumerate}
}

\end{frame}

\begin{frame}\frametitle{Abschluss unter Vereinigung}

Wir konstruieren eine Vereingungsgrammatik

\defbox{Gegeben seien zwei formale Grammatiken $G_1=\tuple{V_1,\Sigma,P_1,\Sntermsub{S}{1}}$
und $G_2=\tuple{V_2,\Sigma,P_2,\Sntermsub{S}{2}}$ mit $V_1\cap V_2=\emptyset$ (o.B.d.A.).
\medskip

Die \redalert{Vereinigungsgrammatik $G_1\uplus G_2$} ist gegeben durch\\[1ex]
%
\narrowcentering{$G_1\uplus G_2=\tuple{V_1\cup V_2\cup\{\Snterm{S}\},\Sigma,P_1\cup P_2 \cup \{ \Snterm{S}\to \Sntermsub{S}{1}\mid \Sntermsub{S}{2}\},\Snterm{S}}$,}\\[1ex]
%
wobei $\Snterm{S}$ ein neues Startsymbol ist, das nicht in $V_1\cup V_2$ vorkommt.
}

In Worten: Die neuen Ableitungsregeln $\Snterm{S}\to \Sntermsub{S}{1}\mid \Sntermsub{S}{2}$ ermöglichen es, dass $G_1\uplus G_2$ entweder Wörter aus $G_1$ oder aus $G_2$ generiert.
\bigskip\pause

Es ist daher leicht zu sehen:\medskip

\theobox{Satz: $\Slang{L}(G_1\uplus G_2)=\Slang{L}(G_1)\cup \Slang{L}(G_2)$.}

\end{frame}

% \begin{frame}\frametitle{Vereinigung: Beispiel}
% 
% TODO
% 
% \end{frame}

\begin{frame}\frametitle{Vereinigungen in Typ 2?}

Bisher haben wir nur erkannt:\medskip

\theobox{Satz: $\Slang{L}(G_1\uplus G_2)=\Slang{L}(G_1)\cup \Slang{L}(G_2)$.}\pause

Für die Abschlusseigenschaft sollte noch mehr gelten:

\theobox{Satz: Wenn $G_1$ und $G_2$ kontextfrei sind, dann ist auch $G_1\uplus G_2$ kontextfrei.}

Das ist leicht zu sehen, da die zusätzliche Regel kontextfrei ist.\pause
\medskip

Daher gilt sogar noch ein stärkerer Satz:

\theobox{Satz: Wenn $G_1$ und $G_2$ von Typ $i\in\{2,1,0\}$ sind, dann ist auch $G_1\uplus G_2$ von Typ $i$.}

\begin{enumerate}[$\leadsto$]
\item Typ-0-Sprachen, kontextsensitive Sprachen und kontextfreie Sprachen sind unter Vereinigung abgeschlossen\\
{\tiny(reguläre Sprachen auch, aber das haben wir anders gezeigt)}
\end{enumerate}
 
\end{frame}

\begin{frame}\frametitle{Abschlusseigenschaften von Typ-2-Sprachen}

\theobox{
Satz: Wenn $\Slang{L}$, $\Slangsub{L}{1}$ und $\Slangsub{L}{2}$ kontextfreie Sprachen sind, dann beschreiben auch die
folgenden Ausdrücke kontextfreie \ghost{Sprachen:}%
\begin{enumerate}[(1)]%
\item \mytabnote{$\Slangsub{L}{1}\cup\Slangsub{L}{2}$}{Abschluss unter Vereinigung}
\item \myhitabnote{$\Slangsub{L}{1}\circ\Slangsub{L}{2}$}{Abschluss unter Konkatenation}
\item \mytabnote{$\Slang{L}^*$}{Abschluss unter Kleene-Stern}
\end{enumerate}
}

\end{frame}

\begin{frame}\frametitle{Konkatenation von Grammatiken}

Wir erinnern uns: $\Slangsub{L}{1}\circ\Slangsub{L}{2}=\{w_1 w_2\mid w_1\in\Slangsub{L}{1}\text{ und }w_2\in\Slangsub{L}{2}\}$\medskip

Es ist nicht schwer, eine passende Grammatik zu finden:

\defbox{Gegeben seien zwei formale Grammatiken $G_1=\tuple{V_1,\Sigma,P_1,\Sntermsub{S}{1}}$
und $G_2=\tuple{V_2,\Sigma,P_2,\Sntermsub{S}{2}}$ mit $V_1\cap V_2=\emptyset$ (o.B.d.A.).
\medskip

Die Grammatik \redalert{$G_1\circ G_2$} ist gegeben durch\\[1ex]
%
\narrowcentering{$G_1\circ G_2=\tuple{V_1\cup V_2\cup\{\Snterm{S}\},\Sigma,P_1\cup P_2 \cup \{ \Snterm{S}\to \Sntermsub{S}{1}\Sntermsub{S}{2}\},\Snterm{S}}$,}\\[1ex]
%
wobei $\Snterm{S}$ ein neues Startsymbol ist, das nicht in $V_1\cup V_2$ vorkommt.
}

In Worten: Die neue Ableitungsregel $\Snterm{S}\to \Sntermsub{S}{1}\Sntermsub{S}{2}$ ermöglicht es, dass $G_1\circ G_2$ Wörter aus $G_1$ gefolgt von Wörtern aus $G_2$ generiert.

% Gilt auch hier $\Slang{L}(G_1\circ G_2)=$?

\end{frame}

\begin{frame}\frametitle{Beispiel Konkatenation}

Wir betrachten die Grammatiken
\begin{align*}
G_1:\quad  \Sntermsub{S}{1} &\to \Sterm{a}\Sntermsub{S}{1}\Sterm{b}\mid \epsilon\\
G_2:\quad  \Sntermsub{S}{2} &\to \Sterm{c}\Sntermsub{S}{2}\mid\epsilon
\end{align*}\pause
Damit ist $\Slang{L}(G_1)=\{\Sterm{a}^i\Sterm{b}^i\mid i\geq 0\}$ und $\Slang{L}(G_2)=\{\Sterm{c}^k\mid k\geq 0\}$.
\bigskip

\pause
Die Konkatenation der Grammatiken enthält die folgenden Regeln:
\begin{align*}
G_1\circ G_2: &&
& \Snterm{S}\to \Sntermsub{S}{1}\Sntermsub{S}{2}
& \Sntermsub{S}{1} &\to \Sterm{a}\Sntermsub{S}{1}\Sterm{b}\mid \epsilon
& \Sntermsub{S}{2} &\to \Sterm{c}\Sntermsub{S}{2}\mid\epsilon
\end{align*}\pause
Damit ergibt sich $\Slang{L}(G_1\circ G_2)=\{\Sterm{a}^i\Sterm{b}^i\Sterm{c}^k\mid i\geq 0, k\geq 0\}$.

\end{frame}

\begin{frame}\frametitle{Korrektheit}

\emph{Hypothese:} $\Slang{L}(G_1\circ G_2)=\Slang{L}(G_1)\circ \Slang{L}(G_2)$
\medskip\pause

\redalert{Leider stimmt das nicht:}\medskip

\examplebox{Gegenbeispiel:
$G_1$ sei die Grammatik mit der einen Regel $\Sntermsub{S}{1}\to \Sterm{a}$
und $G_2$ die Grammatik mit den Regeln $\Sntermsub{S}{2}\to \Sterm{b}$ und 
$\Sterm{a}\Sntermsub{S}{2}\to \Sterm{c}$.\medskip

Dann ist $\Slang{L}(G_1)=\{\Sterm{a}\}$ und $\Slang{L}(G_2)=\{\Sterm{b}\}$. 
Trotzdem erlaubt $G_1\circ G_2$ die Ableitung
$\Snterm{S}\Rightarrow \Sntermsub{S}{1}\Sntermsub{S}{2}\Rightarrow \Sterm{a}\Sntermsub{S}{2}\Rightarrow \Sterm{c}$.
}\pause

Korrekt ist dagegen:

\theobox{Satz: Wenn $G_1$ und $G_2$ kontextfrei sind, dann ist $\Slang{L}(G_1\circ G_2)=\Slang{L}(G_1)\circ \Slang{L}(G_2)$. Zudem ist $G_1\circ G_2$ in diesem Fall kontextfrei.}

\emph{Beweis:} siehe Tafel.\medskip

Damit erhalten wir den gewünschten Abschluss.

\end{frame}

\begin{frame}\frametitle{Abschlusseigenschaften von Typ-2-Sprachen}

\theobox{
Satz: Wenn $\Slang{L}$, $\Slangsub{L}{1}$ und $\Slangsub{L}{2}$ kontextfreie Sprachen sind, dann beschreiben auch die
folgenden Ausdrücke kontextfreie \ghost{Sprachen:}%
\begin{enumerate}[(1)]%
\item \mytabnote{$\Slangsub{L}{1}\cup\Slangsub{L}{2}$}{Abschluss unter Vereinigung}
\item \mytabnote{$\Slangsub{L}{1}\circ\Slangsub{L}{2}$}{Abschluss unter Konkatenation}
\item \myhitabnote{$\Slang{L}^*$}{Abschluss unter Kleene-Stern}
\end{enumerate}
}

\end{frame}

\begin{frame}\frametitle{Kleene-Stern für Grammatiken}
Wir erinnern uns: $\Slang{L}^*=\{w_1 w_2\cdots w_i\mid i\geq 0, w_1,\ldots,w_i\in\Slang{L}\}=\bigcup_{i\geq 0} \Slang{L}^i$\pause\medskip

Auch hier kann man leicht eine passende Grammatik finden:

\defbox{Gegeben sei eine formale Grammatik $G=\tuple{V,\Sigma,P,\Snterm{S}}$.
\medskip

Die Grammatik \redalert{$G^*$} ist gegeben durch\\[1ex]
%
\narrowcentering{$G^*=\tuple{V\cup\{\Snterm{S}'\},\Sigma,P \cup \{ \Snterm{S}'\to \epsilon\mid\Snterm{S}\Snterm{S}'\},\Snterm{S}'}$,}\\[1ex]
%
wobei $\Snterm{S}'$ ein neues Startsymbol ist, das nicht in $V$ vorkommt.
}

In Worten: Die neuen Ableitungsregeln $\Snterm{S}'\to \epsilon\mid\Snterm{S}\Snterm{S}'$ ermöglichen es, dass $G^*$ beliebig lange Ketten aus Wörtern aus $G$ generiert.

\end{frame}

\begin{frame}\frametitle{Beispiel Kleene-Stern}

Wir betrachten die Grammatik
\begin{align*}
G:\qquad \Snterm{S} &\to \Snterm{S}\Sterm{a}\Snterm{S}\Sterm{b}\mid \Snterm{S}\Sterm{b}\Snterm{S}\Sterm{a}\mid \epsilon
\end{align*}
\alert{Übung:} Welche Sprache erzeugt diese Grammatik?\medskip\pause

Der Kleene-Abschluss dieser Grammatik ist
\begin{align*}
G^*:\qquad \Snterm{S}' &\to \epsilon\mid\Snterm{S}\Snterm{S}' & \Snterm{S} &\to \Snterm{S}\Sterm{a}\Snterm{S}\Sterm{b}\mid \Snterm{S}\Sterm{b}\Snterm{S}\Sterm{a}\mid \epsilon
\end{align*}
\alert{Übung:} Welche Sprache erzeugt diese Grammatik? 

\end{frame}

\begin{frame}\frametitle{Korrektheit beim Kleene-Abschluss}

Wie schon beim Konkatenation funktioniert diese Operation auf kontextfreien Grammatiken wie
erwünscht:

\theobox{Satz: Wenn $G$ kontextfrei ist, dann ist $\Slang{L}(G^*)=\Slang{L}(G)^*$. Zudem ist $G^*$ in diesem Fall kontextfrei.}

\emph{Beweis:} siehe Tafel.
\medskip\pause

Für nicht kontextfreie Grammatiken gilt dieser Satz im Allgemeinen nicht:

\examplebox{Beispiel: Die (kontextsensitive) Grammatik $G$ mit der einen Regel $\Snterm{S}\Snterm{S}\to \Sterm{0}$
kann nichts hervorbringen. Trotzdem erlaubt die Grammatik $G^*$ die Ableitung eines Wortes:
\[ \Snterm{S}'\Rightarrow \Snterm{S}\Snterm{S}\Rightarrow \Sterm{0}\]
}

\end{frame}

\begin{frame}\frametitle{Nicht-Abschlusseigenschaften}

\theobox{
Satz: Es gibt kontextfreie Sprachen $\Slang{L}$, $\Slangsub{L}{1}$ und $\Slangsub{L}{2}$, so dass die folgenden
Ausdrücke keine kontextfreien Sprachen sind:
\begin{enumerate}[(1)]
\item \myhitabnote{$\Slangsub{L}{1}\cap\Slangsub{L}{2}$}{Nichtabschluss unter Schnitt}
\item \mytabnote{$\overline{\Slang{L}}$}{Nichtabschluss unter Komplement}
\end{enumerate}
}

\end{frame}

\begin{frame}\frametitle{Nicht-Abschluss unter $\cap$}

\emph{Beweisansatz:} Wir müssen kontextfreie Sprachen $\Slangsub{L}{1}$ und $\Slangsub{L}{2}$ finden, für die
$\Slangsub{L}{1}\cap\Slangsub{L}{2}$ nicht kontextfrei ist.\medskip

\alert{Welche nicht-kontextfreien Sprachen kennen wir?}\pause
\medskip

\narrowcentering{{\huge$\{\Sterm{a}^i\Sterm{b}^i\Sterm{c}^i\mid i\geq 0\}$}}
\bigskip\pause

\alert{Welche ähnlichen kontextfreien Sprachen kennen wir?}\pause
\medskip

\narrowcentering{{\huge$\{\Sterm{a}^i\Sterm{b}^i\Sterm{c}^k\mid i\geq 0, k\geq 0\}$}}

\end{frame}

\begin{frame}\frametitle{Nicht-Abschluss unter $\cap$ (2)}

\theobox{
Satz: Es gibt kontextfreie Sprachen $\Slang{L}$, $\Slangsub{L}{1}$ und $\Slangsub{L}{2}$, so dass $\Slangsub{L}{1}\cap\Slangsub{L}{2}$ keine kontextfreie Sprache ist.}

\emph{Beweis:} Die folgenden Sprachen sind kontextfrei:
\begin{align*}
\{\Sterm{a}^i\Sterm{b}^i\mid i\geq 0\} && \text{(zuvor gezeigt)}\\
\{\Sterm{b}^i\Sterm{c}^i\mid i\geq 0\} && \text{(analog)}\\
\{\Sterm{a}^i\mid i\geq 0\} && \text{(regulär)}\\
\{\Sterm{c}^i\mid i\geq 0\} && \text{(regulär, zuvor gezeigt)}
\end{align*}\pause
% 
Dank Abschluss unter Konkatenation sind also auch kontextfrei:
\begin{align*}
\Slangsub{L}{1}&=\{\Sterm{a}^i\Sterm{b}^i\mid i\geq 0\}\circ \{\Sterm{c}^i\mid i\geq 0\} = \{\Sterm{a}^i\Sterm{b}^i\Sterm{c}^k\mid i\geq 0, k\geq 0\}\\
\Slangsub{L}{2}&=\{\Sterm{a}^i\mid i\geq 0\}\circ \{\Sterm{b}^i\Sterm{c}^i\mid i\geq 0\} = \{\Sterm{a}^i\Sterm{b}^k\Sterm{c}^k\mid i\geq 0, k\geq 0\}.
\end{align*}\pause
Der Schnitt $\Slangsub{L}{1}\cap\Slangsub{L}{2}$ ist aber $\{\Sterm{a}^i\Sterm{b}^i\Sterm{c}^i\mid i\geq 0\}$ und
also nicht kontextfrei (zuvor gezeigt).\qed


\end{frame}

\begin{frame}\frametitle{Nicht-Abschlusseigenschaften}

\theobox{
Satz: Es gibt kontextfreie Sprachen $\Slang{L}$, $\Slangsub{L}{1}$ und $\Slangsub{L}{2}$, so dass die folgenden
Ausdrücke keine kontextfreien Sprachen sind:
\begin{enumerate}[(1)]
\item \mytabnote{$\Slangsub{L}{1}\cap\Slangsub{L}{2}$}{Nichtabschluss unter Schnitt}
\item \myhitabnote{$\overline{\Slang{L}}$}{Nichtabschluss unter Komplement}
\end{enumerate}
}

\end{frame}

\begin{frame}\frametitle{Nicht-Abschluss unter Komplement}

\theobox{
Satz: Es gibt eine kontextfreie Sprache $\Slang{L}$, so dass $\overline{\Slang{L}}$ keine kontextfreie Sprache ist.
}\pause

\emph{Beweis:}
Die Behauptung folgt unmittelbar aus dem bereits gezeigten\pause:
\begin{itemize}
\item Angenommen, Typ-2-Sprachen wären unter Komplement abgeschlossen\pause
\item Wir wissen bereits, dass Typ-2-Sprachen unter Vereinigung abgeschlossen sind\pause
\item Laut den Gesetzen der Mengenlehre gilt:\\[1ex]
\narrowcentering{$\Slangsub{L}{1}\cap\Slangsub{L}{2} = \overline{\overline{\Slangsub{L}{1}}\cup\overline{\Slangsub{L}{2}}}\qquad$ (De Morgan)}\pause
\item Daraus folgt, dass Typ-2-Sprachen unter Schnitt abgeschlossen sind -- Widerspruch\qed
\end{itemize}

\end{frame}

\begin{frame}\frametitle{Eine Warnung zum Nicht-Abschluss}

\begin{center}
\redalert{Nicht-Abschluss unter Schnitt und Komplement bedeutet nicht, dass Schnitte bzw. Komplemente kontextfreier Sprachen grundsätzlich nicht kontextfrei sein können.}
\end{center}\pause

\examplebox{Beispiel: Alle regulären Sprachen sind kontextfrei, aber ihre Schnitte und Komplemente sind
weiterhin regulär, also auch kontextfrei.}\pause

\examplebox{Beispiel: Das Komplement der nicht-regulären Sprache $\Slang{L}=\{\Sterm{a}^i\Sterm{b}^i\mid i\geq 0\}$ ist
die Vereinigung der folgenden Sprachen:\\[1ex]
\narrowcentering{$\Slangsub{L}{1}=\{\Sterm{a},\Sterm{b}\}^*\{\Sterm{ba}\}\{\Sterm{a},\Sterm{b}\}^*$\hfill$\Slangsub{L}{2}=\{\Sterm{a}\}^+ \Slang{L}$\hfill$\Slangsub{L}{3}=\Slang{L}\{\Sterm{b}\}^+$}\\[1ex]
% 
% \begin{itemize}
% \item $\Slangsub{L}{1}=\{\Sterm{a},\Sterm{b}\}^*\{\Sterm{ba}\}\{\Sterm{a},\Sterm{b}\}^*$
% \item $\Slangsub{L}{2}=\{\Sterm{a}\}^+ \Slang{L}$
% \item $\Slangsub{L}{3}=\Slang{L}\{\Sterm{b}\}^+$
% \end{itemize}
$\Slangsub{L}{1}$ ist regulär, also kontextfrei. $\Slangsub{L}{2}$ und $\Slangsub{L}{3}$ sind kontextfrei, da sie als Konkatenation zweier kontextfreier Sprachen entstehen. Die Vereinigung $\Slangsub{L}{1}\cup\Slangsub{L}{2}\cup\Slangsub{L}{3}$ ist also auch kontextfrei.
}

\end{frame}

\begin{frame}\frametitle{Ein nicht-kontextfreies Komplement}

Unser Widerspruchsbeweis zum Abschluss unter Komplement liefert uns noch kein 
konkretes Beispiel eines nicht-kontextfreien Komplements.\pause
\medskip

Wir können es aber aus den Beweisen ableiten:
\begin{itemize}
\item Seien $\Slangsub{L}{1}=\{\Sterm{a}^i\Sterm{b}^i\Sterm{c}^k\mid i\geq 0, k\geq 0\}$ 
und \ghost{$\Slangsub{L}{2}=\{\Sterm{a}^i\Sterm{b}^k\Sterm{c}^k\mid i\geq 0, k\geq 0\}$}
%
\item $\Slangsub{L}{1}\cap\Slangsub{L}{2}=\overline{\overline{\Slangsub{L}{1}}\cup\overline{\Slangsub{L}{2}}}=\{\Sterm{a}^i\Sterm{b}^i\Sterm{c}^i\mid i\geq 0\}=\Slang{L}$ ist nicht kontextfrei\pause
%
\item $\overline{\Slangsub{L}{1}}\cup\overline{\Slangsub{L}{2}}$ ist die Vereinigung der folgenden Typ-2-Sprachen:
\begin{itemize}
\item $\{\Sterm{a},\Sterm{b},\Sterm{c}\}^*\circ\{\Sterm{ba},\Sterm{ca},\Sterm{bc}\}\circ\{\Sterm{a},\Sterm{b},\Sterm{c}\}^*$ (falsche Reihenfolge)
\item $\{\Sterm{a}\}^+\circ \{\Sterm{a}^i\Sterm{b}^i\mid i\geq 0\}\circ\{\Sterm{c}\}^*$ \hspace{1.25cm}(zu viele $\Sterm{a}$ für $\Slangsub{L}{1}$)
\item $\{\Sterm{a}^i\Sterm{b}^i\mid i\geq 0\}\circ\{\Sterm{b}\}^+\circ\{\Sterm{c}\}^*$ \hspace{1.25cm}(zu viele $\Sterm{b}$ für $\Slangsub{L}{1}$)
\item $\{\Sterm{a}\}^*\circ\{\Sterm{b}\}^+\circ \{\Sterm{b}^i\Sterm{c}^i\mid i\geq 0\}$ \hspace{1.25cm}(zu viele $\Sterm{b}$ für $\Slangsub{L}{2}$)
\item $\{\Sterm{a}\}^*\circ \{\Sterm{b}^i\Sterm{c}^i\mid i\geq 0\}\circ\{\Sterm{c}\}^+$  \hspace{1.25cm}(zu viele $\Sterm{c}$ für $\Slangsub{L}{2}$)
\end{itemize}
%
\item $\overline{\Slangsub{L}{1}}\cup\overline{\Slangsub{L}{2}}$ ist daher kontextfrei, aber ihr Komplement nicht
\end{itemize}


\end{frame}

\begin{frame}\frametitle{Zusammenfassung und Ausblick}

Kontextfreie Sprachen sind \redalert{abgeschlossen unter Vereinigung, Konkatenation und Kleene-Stern}
\bigskip

Kontextfreie Sprachen sind \redalert{nicht abgeschlossen unter Komplement und Schnitt}
\bigskip

Abschlüsse beruhen auf Grammatikoperationen, die man auf beliebige Grammatiken anwenden könnte, aber nur
der \redalert{Abschluss unter $\cup$ für Typ 0 und Typ 1} kann so gezeigt werden.
\bigskip

\anybox{yellow}{
Offene Fragen:
\begin{itemize}
\item Haben kontextfreie Sprachen ein Berechnungsmodell?
\item Welche Probleme auf kontextfreien Grammatiken kann man lösen?
\item Was gibt es zu Typ 1 und Typ 0 zu sagen?
\end{itemize}
}

\end{frame}


\end{document}