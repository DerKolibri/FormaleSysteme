\documentclass[aspectratio=1610,onlymath]{beamer}
% \documentclass[aspectratio=1610,onlymath,handout]{beamer}

% Macros used by all lectures, but not necessarily by excercises

%%% General setup and dependencies:

% \usetheme[ddcfooter,nosectionnum]{tud}
\usetheme[nosectionnum,pagenum,noheader]{tud}
% \usetheme[nosectionnum,pagenum]{tud}

% Increase body font size to a sane level:
\let\origframetitle\frametitle
% \renewcommand{\frametitle}[1]{\origframetitle{#1}\normalsize}
\renewcommand{\frametitle}[1]{\origframetitle{#1}\fontsize{10pt}{13.2}\selectfont}
\setbeamerfont{itemize/enumerate subbody}{size=\small} % tud defaults to scriptsize!
\setbeamerfont{itemize/enumerate subsubbody}{size=\small}
% \setbeamerfont{normal text}{size=\small}
% \setbeamerfont{itemize body}{size=\small}

\renewcommand{\emph}[1]{\textbf{#1}}

\def\arraystretch{1.3}% Make tables even less cramped vertically

\usepackage[ngerman]{babel}
\usepackage[utf8]{inputenc}
\usepackage[T1]{fontenc}

%\usepackage{graphicx}
\usepackage[export]{adjustbox} % loads graphicx
\usepackage{import}
\usepackage{stmaryrd}
\usepackage[normalem]{ulem} % sout command
% \usepackage{times}
\usepackage{txfonts}

% \usepackage[perpage]{footmisc} % reset footnote counter on each page -- fails with beamer (footnotes gone)
\usepackage{perpage}  % reset footnote counter on each page
\MakePerPage{footnote}

\usepackage{tikz}
\usetikzlibrary{arrows,positioning}
% Inspired by http://www.texample.net/tikz/examples/hand-drawn-lines/
\usetikzlibrary{decorations.pathmorphing}
\pgfdeclaredecoration{penciline}{initial}{
    \state{initial}[width=+\pgfdecoratedinputsegmentremainingdistance,
    auto corner on length=1mm,]{
        \pgfpathcurveto%
        {% From
            \pgfqpoint{\pgfdecoratedinputsegmentremainingdistance}
                      {\pgfdecorationsegmentamplitude}
        }
        {%  Control 1
        \pgfmathrand
        \pgfpointadd{\pgfqpoint{\pgfdecoratedinputsegmentremainingdistance}{0pt}}
                    {\pgfqpoint{-\pgfdecorationsegmentaspect
                     \pgfdecoratedinputsegmentremainingdistance}%
                               {\pgfmathresult\pgfdecorationsegmentamplitude}
                    }
        }
        {%TO 
        \pgfpointadd{\pgfpointdecoratedinputsegmentlast}{\pgfpoint{1pt}{1pt}}
        }
    }
    \state{final}{}
}
\tikzset{handdrawn/.style={decorate,decoration=penciline}}
\tikzset{every shadow/.style={fill=none,shadow xshift=0pt,shadow yshift=0pt}}
% \tikzset{module/.append style={top color=\col,bottom color=\col}}

% Use to make Tikz attributes with Beamer overlays
% http://tex.stackexchange.com/a/6155
\tikzset{onslide/.code args={<#1>#2}{%
  \only<#1| handout:0>{\pgfkeysalso{#2}} 
}}
\tikzset{onslideprint/.code args={<#1>#2}{%
  \only<#1>{\pgfkeysalso{#2}} 
}}

%%% Title -- always set this first

\newcommand{\defineTitle}[3]{
	\newcommand{\lectureindex}{#1}
	\title{Formale Systeme}
	\subtitle{\href{\lectureurl}{#1. Vorlesung: #2}}
	\author{\href{http://korrekt.org/}{Markus Kr\"{o}tzsch}}
%	\author{\href{http://www.sebastian-rudolph.de}{Sebastian Rudolph} in Vertretung von \href{http://korrekt.org/}{Markus Kr\"{o}tzsch}}
	\date{#3}
	\datecity{TU Dresden}
% 	\institute{Computational Logic}
}

%%% Table of contents:

\RequirePackage{ifthen}

\newcommand{\highlight}[2]{%
	\ifthenelse{\equal{#1}{\lectureindex}}{\alert{#2}}{#2}%
}

\def\myspace{-0.7ex}
\newcommand{\printtoc}{
\begin{tabular}{r@{$\quad$}l}
\highlight{1}{1.} & \highlight{1}{Willkommen/Einleitung formale Sprachen}\\[\myspace]
\highlight{2}{2.} & \highlight{2}{Grammatiken und die Chomsky-Hierarchie}\\[\myspace]
\highlight{3}{3.} & \highlight{3}{Endliche Automaten}\\[\myspace]
\highlight{4}{4.} & \highlight{4}{Complexity of FO query answering}\\[\myspace]
\highlight{5}{5.} & \highlight{5}{Conjunctive queries}\\[\myspace]
\highlight{6}{6.} & \highlight{6}{Tree-like conjunctive queries}\\[\myspace]
\highlight{7}{7.} & \highlight{7}{Query optimisation}\\[\myspace]
\highlight{8}{8.} & \highlight{8}{Conjunctive Query Optimisation / First-Order~Expressiveness}\\[\myspace]
\highlight{9}{9.} & \highlight{9}{First-Order~Expressiveness / Introduction to Datalog}\\[\myspace]
\highlight{10}{10.} & \highlight{10}{Expressive Power and Complexity of Datalog}\\[\myspace]
\highlight{11}{11.} & \highlight{11}{Optimisation and Evaluation of Datalog}\\[\myspace]
\highlight{12}{12.} & \highlight{12}{Evaluation of Datalog (2)}\\[\myspace]
\highlight{13}{13.} & \highlight{13}{Graph Databases and Path Queries}\\[\myspace]
\highlight{14}{14.} & \highlight{14}{Outlook: database theory in practice}
\end{tabular}
}

\newcommand{\overviewslide}{%
\begin{frame}\frametitle{Overview}
\printtoc
\medskip

Siehe \href{\lectureurl}{course homepage [$\Rightarrow$ link]} for more information and materials
\end{frame}
}

%%% Colours:

\usepackage{xcolor,colortbl}
\definecolor{redhighlights}{HTML}{FFAA66}
\definecolor{lightblue}{HTML}{55AAFF}
\definecolor{lightred}{HTML}{FF5522}
\definecolor{lightpurple}{HTML}{DD77BB}
\definecolor{lightgreen}{HTML}{55FF55}
\definecolor{darkred}{HTML}{CC4411}
\definecolor{darkblue}{HTML}{176FC0}%{1133AA}
\definecolor{nightblue}{HTML}{2010A0}%{1133AA}
\definecolor{alert}{HTML}{176FC0}
\definecolor{darkgreen}{HTML}{36AB14}
\definecolor{strongyellow}{HTML}{FFE219}
\definecolor{devilscss}{HTML}{666666}

\newcommand{\redalert}[1]{\textcolor{darkred}{#1}}

%%% Style commands

\newcommand{\quoted}[1]{\texttt{"}{#1}\texttt{"}}
\newcommand{\squote}{\texttt{"}} % straight quote
\newcommand{\Sterm}[1]{\ensuremath{\mathtt{\textcolor{purple}{#1}}}}    % letters in alphabets
\newcommand{\Snterm}[1]{\textsf{\textcolor{darkblue}{#1}}} % nonterminal symbols
\newcommand{\Sntermsub}[2]{\Snterm{#1}_{\Snterm{#2}}} % nonterminal symbols
\newcommand{\Slang}[1]{\textbf{\textcolor{black}{#1}}}    % languages
\newcommand{\Slangsub}[2]{\Slang{#1}_{\Slang{#2}}}    % languages
% Code
\newcommand{\Scode}[1]{\textbf{#1}}    % reserved words in program listings, e.g., "if"
\newcommand{\Scodelit}[1]{\textcolor{purple}{#1}}    % literals in program listings, e.g., strings
\newcommand{\Scomment}[1]{\textcolor{gray}{#1}}    % comment in program listings

\newcommand{\epstrastar}{\mathrel{\mathord{\stackrel{\epsilon}{\to}}{}^*}} % transitive reflexive closure of epsilon transitions in an epslion-NFA

\newcommand{\narrowcentering}[1]{\mbox{}\hfill#1\hfill\mbox{}}

\newcommand{\defeq}{\mathrel{:=}}

\newcommand{\Smach}[1]{\ensuremath{\mathcal{#1}}}    % machines

%%% Slide layout commands:

\newcommand{\sectionSlide}[1]{
\frame{\begin{center}
\LARGE
#1
\end{center}}
}
\newcommand{\sectionSlideNoHandout}[1]{
\frame<handout:0>{\begin{center}
\LARGE
#1
\end{center}}
}

\newcommand{\mydualbox}[3]{%
 \begin{minipage}[t]{#1}
 \begin{beamerboxesrounded}[upper=block title,lower=block body,shadow=true]%
    {\centering\usebeamerfont*{block title}#2}%
    \raggedright%
    \usebeamerfont{block body}
%     \small
    #3%
  \end{beamerboxesrounded}
  \end{minipage}
}
% 
\newcommand{\myheaderbox}[2]{%
 \begin{minipage}[t]{#1}
 \begin{beamerboxesrounded}[upper=block title,lower=block title,shadow=true]%
    {\centering\usebeamerfont*{block title}\rule{0pt}{2.6ex} #2}%
  \end{beamerboxesrounded}
  \end{minipage}
}

\newcommand{\mycontentbox}[2]{%
 \begin{minipage}[t]{#1}%
 \begin{beamerboxesrounded}[upper=block body,lower=block body,shadow=true]%
    {\centering\usebeamerfont*{block body}\rule{0pt}{2.6ex}#2}%
  \end{beamerboxesrounded}
  \end{minipage}
}

\newcommand{\mylcontentbox}[2]{%
 \begin{minipage}[t]{#1}%
 \begin{beamerboxesrounded}[upper=block body,lower=block body,shadow=true]%
    {\flushleft\usebeamerfont*{block body}\rule{0pt}{2.6ex}#2}%
  \end{beamerboxesrounded}
  \end{minipage}
}

% label=180:{\rotatebox{90}{{\footnotesize\textcolor{darkgreen}{Beispiel}}}}
% \hspace{-8mm}\ghost{\raisebox{-7mm}{\rotatebox{90}{{\footnotesize\textcolor{darkgreen}{Beispiel}}}}}\hspace{8mm}
\newcommand{\examplebox}[1]{%
	\begin{tikzpicture}[decoration=penciline, decorate]
		\pgfmathsetseed{1235}
		\node (n1) [decorate,draw=darkgreen, fill=darkgreen!10,thick,align=left,text width=\linewidth, inner ysep=2mm, inner xsep=2mm] at (0,0) {#1};
% 		\node (n2) [align=left,text width=\linewidth,inner sep=0mm] at (n1.92) {{\footnotesize\raisebox{3mm}{\textcolor{darkgreen}{Beispiel}}}};
% 		\node (n2) [decorate,draw=darkgreen, fill=darkgreen!10,thick, align=left,text width=\linewidth,inner sep=2mm] at (n1.90) {{\footnotesize\raisebox{0mm}{\textcolor{darkgreen}{Beispiel}}}};
	\end{tikzpicture}%
}%

\newcommand{\codebox}[1]{%
	\begin{tikzpicture}[decoration=penciline, decorate]
		\pgfmathsetseed{1236}
		\node (n1) [decorate,draw=strongyellow, fill=strongyellow!10,thick,align=left,text width=\linewidth, inner ysep=2mm, inner xsep=2mm] at (0,0) {#1};
	\end{tikzpicture}%
}%

\newcommand{\defbox}[1]{%
	\begin{tikzpicture}[decoration=penciline, decorate]
		\pgfmathsetseed{1237}
		\node (n1) [decorate,draw=darkred, fill=darkred!10,thick,align=left,text width=\linewidth, inner ysep=2mm, inner xsep=2mm] at (0,0) {#1};
	\end{tikzpicture}%
}%

\newcommand{\theobox}[1]{%
	\begin{tikzpicture}[decoration=penciline, decorate]
		\pgfmathsetseed{1240}
		\node (n1) [decorate,draw=darkblue, fill=darkblue!10,thick,align=left,text width=\linewidth, inner ysep=2mm, inner xsep=2mm] at (0,0) {#1};
	\end{tikzpicture}%
}%

\newcommand{\anybox}[2]{%
	\begin{tikzpicture}[decoration=penciline, decorate]
		\pgfmathsetseed{1240}
		\node (n1) [decorate,draw=#1, fill=#1!10,thick,align=left,text width=\linewidth, inner ysep=2mm, inner xsep=2mm] at (0,0) {#2};
	\end{tikzpicture}%
}%


\newsavebox{\mybox}%
\newcommand{\doodlebox}[2]{%
\sbox{\mybox}{#2}%
	\begin{tikzpicture}[decoration=penciline, decorate]
		\pgfmathsetseed{1238}
		\node (n1) [decorate,draw=#1, fill=#1!10,thick,align=left,inner sep=1mm] at (0,0) {\usebox{\mybox}};
	\end{tikzpicture}%
}%

% Common notation

\usepackage{amsmath,amssymb,amsfonts}
\usepackage{xspace}

\newcommand{\lectureurl}{https://iccl.inf.tu-dresden.de/web/FS2016}

\DeclareMathAlphabet{\mathsc}{OT1}{cmr}{m}{sc} % Let's have \mathsc since the slide style has no working \textsc

% Dual of "phantom": make a text that is visible but intangible
\newcommand{\ghost}[1]{\raisebox{0pt}[0pt][0pt]{\makebox[0pt][l]{#1}}}

\newcommand{\tuple}[1]{\langle{#1}\rangle}

%%% Annotation %%%

\usepackage{color}
\newcommand{\todo}[1]{{\tiny\color{red}\textbf{TODO: #1}}}



%%% Old macros below; move when needed

\newcommand{\blank}{\text{\textvisiblespace}} % empty tape cell for TM

% table syntax
\newcommand{\dom}{\textbf{dom}}
\newcommand{\adom}{\textbf{adom}}
\newcommand{\dbconst}[1]{\texttt{"#1"}}
\newcommand{\pred}[1]{\textsf{#1}}
\newcommand{\foquery}[2]{#2[#1]}
\newcommand{\ground}[1]{\textsf{ground}(#1)}
% \newcommand{\foquery}[2]{\{#1\mid #2\}} %% Notation as used in Alice Book
% \newcommand{\foquery}[2]{\tuple{#1\mid #2}}

\newcommand{\quantor}{\mathord{\reflectbox{$\text{\sf{Q}}$}}} % the generic quantor

% logic syntax
\newcommand{\Inter}{\mathcal{I}} %used to denote an interpretation
\newcommand{\Jnter}{\mathcal{J}} %used to denote another interpretation
\newcommand{\Knter}{\mathcal{K}} %used to denote yet another interpretation
\newcommand{\Zuweisung}{\mathcal{Z}} %used to denote a variable assignment

% query languages
\newcommand{\qlang}[1]{{\sf #1}} % Font for query languages
\newcommand{\qmaps}[1]{\textbf{QM}({\sf #1})} % Set of query mappings for a query language

%%% Complexities %%%

\hyphenation{Exp-Time} % prevent "Ex-PTime" (see, e.g. Tobies'01, Glimm'07 ;-)
\hyphenation{NExp-Time} % better that than something else

% \newcommand{\complclass}[1]{{\sc #1}\xspace} % font for complexity classes
\newcommand{\complclass}[1]{\ensuremath{\mathsc{#1}}\xspace} % font for complexity classes

\newcommand{\ACzero}{\complclass{AC$_0$}}
\newcommand{\LogSpace}{\complclass{L}}
\newcommand{\NLogSpace}{\complclass{NL}}
\newcommand{\PTime}{\complclass{P}}
\newcommand{\NP}{\complclass{NP}}
\newcommand{\coNP}{\complclass{coNP}}
\newcommand{\PH}{\complclass{PH}}
\newcommand{\PSpace}{\complclass{PSpace}}
\newcommand{\NPSpace}{\complclass{NPSpace}}
\newcommand{\ExpTime}{\complclass{ExpTime}}
\newcommand{\NExpTime}{\complclass{NExpTime}}
\newcommand{\ExpSpace}{\complclass{ExpSpace}}
\newcommand{\TwoExpTime}{\complclass{2ExpTime}}
\newcommand{\NTwoExpTime}{\complclass{N2ExpTime}}
\newcommand{\ThreeExpTime}{\complclass{3ExpTime}}
\newcommand{\kExpTime}[1]{\complclass{#1ExpTime}}
\newcommand{\kExpSpace}[1]{\complclass{#1ExpSpace}}


\defineTitle{23}{Logisches Schließen}{15. Januar 2018}

\begin{document}

\maketitle

\sectionSlideNoHandout{Rückblick}

\begin{frame}\frametitle{Logik: Glossar}

\begin{tabular}{@{}rll@{}}
\rowcolor{darkblue!10}
Atom & kleinste mögliche Formel & $p\in\Slang{P}$\\
%
Teilformel & Unterausdruck, der Formel ist & $\textsf{Sub}(F)$ \\
%
\rowcolor{darkblue!10}
Wertzuweisung & Funktion & $w:\Slang{P}\to\{\mytrue,\myfalse\}$\\
%
Modell & erfüllende Wertzuweisung & $w\models F$ / $w(F)=\mytrue$\\
%
\rowcolor{darkblue!10}
Literal & Atom oder negiertes Atom & $p$ oder $\neg p$\\
%
Klausel & Disjunktion von Literalen & $L_1\vee\ldots\vee L_n$ \\
%
\rowcolor{darkblue!10}
Monom & Konjunktion von Literalen & $L_1\wedge\ldots\wedge L_n$ \\
%
logische & Teilmengenbeziehung der  & $F\models G$ oder \\[-0.8ex]
Konsequenz & Modellmengen; entspricht $\to$ &  $\mathcal{F}\models G$\\
%
\rowcolor{darkblue!10}
semantische & Gleichheit der  & $F\equiv G$ oder \\[-0.8ex]
\rowcolor{darkblue!10}
Äquivalenz & Modellmengen; entspricht $\leftrightarrow$ &  $\mathcal{F}\equiv\mathcal{G}$\\
%
Tautologie & allgemeingültige Formel & $\models F$

\end{tabular}

\end{frame}

\begin{frame}\frametitle{Vereinfachungen und Normalformen}

Jede Formel ist semantisch äquivalent zu einer, die nur folgende Junktoren verwendet:
\begin{itemize}
\item $\wedge$ und $\neg$
\item $\vee$ und $\neg$
\item und verschiedene andere Kombinationen \ldots\ (z.B. $\to$ und $\neg$; oder auch $\uparrow$)
\end{itemize}\medskip

\emph{Normalformen}
\begin{itemize}
\item \alert{Negationsnormalform}\\
	stets linear
\item \alert{Konjunktive Normalform}\\
	im schlimmsten Fall exponentiell; mit Hilfsatomen linear
\item \alert{Disjunktive Normalform}\\
	im schlimmsten Fall exponentiell
\end{itemize}

\end{frame}

\sectionSlide{Logisches Schließen}

\begin{frame}\frametitle{Logisches Schließen}

Allgemein umfasst \redalert{logisches Schließen} jede Berechnung,
bei der semantische Eigenschaften von Formeln ermittelt werden.
\bigskip

Vielfach hat man es mit Entscheidungsproblemen zu tun, z.B.:

\defbox{\redalert{Schlussfolgerung}:
\begin{itemize}
\item \emph{Eingabe:} endliche Formelmenge $\mathcal{F}$, Formel $G$
\item \emph{Ausgabe:} "`ja"' wenn $\mathcal{F}\models G$ gilt; sonst "`nein."'
\end{itemize}
}

\defbox{\redalert{Allgemeingültigkeit}:
\begin{itemize}
\item \emph{Eingabe:} Formel $F$
\item \emph{Ausgabe:} "`ja"' wenn $F$ allgemeingültig ist; sonst "`nein."'
\end{itemize}
}

\defbox{\redalert{Unerfüllbarkeit}:
\begin{itemize}
\item \emph{Eingabe:} Formel $F$
\item \emph{Ausgabe:} "`ja"' wenn $F$ unerfüllbar ist; sonst "`nein."'
\end{itemize}
}

\end{frame}

\begin{frame}\frametitle{Viele logische Probleme -- viele Algorithmen?}

Oft kann man unterschiedliche Probleme leicht ineinander umwandeln:
\medskip

\begin{itemize}
\item $\mathcal{F}\models G$ (für endliche $\mathcal{F}$) \alert{gdw.}\\ $\bigwedge_{F\in\mathcal{F}}F\to G$ allgemeingültig \alert{gdw.} $\bigwedge_{F\in\mathcal{F}}F\wedge\neg G$ unerfüllbar
\item $F$ allgemeingültig \alert{gdw.} $\emptyset\models F$ \alert{gdw.} $\neg F$ unerfüllbar
\item $F$ unerfüllbar \alert{gdw.} $F\models \bot$ \alert{gdw.} $\neg F$ allgemeingültig
\end{itemize}
\begin{enumerate}[$\leadsto$]
\item Jedes Verfahren, welches eines dieser Probleme lösen kann, lässt sich prinzipiell auch zur Lösung der anderen einsetzen.
\end{enumerate}

{\footnotesize\textcolor{devilscss}{%
(Es gibt auch Probleme des logischen Schließens, für die das höchstwahrscheinlich nicht zutrifft, z.B. "`Gegeben eine Formel $F$, entscheide ob es eine kürzere Formel gibt, die zu $F$ äquivalent ist."')}
}

\end{frame}

\begin{frame}\frametitle{Schließen mit Wahrheitswertetabellen}

Die genannten Probleme sind mittels Wahrheitswertetabelle lösbar:%
\begin{itemize}
\item \alert{Allgemeingültigkeit:} Ist der Wert von $F$ für jede Belegung $\mytrue$?
\item \alert{Unerfüllbarkeit:} Ist der Wert von $F$ für jede Belegung $\myfalse$?
\item \alert{Schlussfolgerung:} Ist der Wert von $G$ für jede Belegung $\mytrue$, unter der alle Formeln von $\mathcal{F}$ den Wert $\mytrue$ annehmen?
\end{itemize}\pause

\emph{Nachteil:} Die Tabelle hat $2^n$ Zeilen, wenn die betrachteten Formeln $n$ Atome haben
\bigskip

\examplebox{Beispiel: Für die Modellierung von Sudoku nutzen wir $9\times 9\times 9$ Atome. Die entsprechende Tabelle hat also $2^{729}$ (ca. $2.8\times 10^{219}$) Zeilen. Zum Vergleich:
\begin{itemize}\pause
\item Anzahl der Neuronen im menschlichen Gehirn: ca. $1.5\times 10^{14}$\pause
% \item Zahl der Sterne im beobachtbaren Universum: unter $10^{24}$\pause
\item Jemals digital gespeicherte Daten in Bytes: $<<10^{24}$\pause
\item Nanosekunden bis der Kern der Sonne ausbrennt: ca. $10^{29}$\pause
\item Anzahl der Atome im beobachtbaren Universum: $10^{78}$ -- $10^{82}$
\end{itemize}
}

\end{frame}

\begin{frame}[t]\frametitle{Schließen mit DNF}

Man kann (Un)Erfüllbarkeit leicht aus DNF ablesen:\medskip

\theobox{Satz: Eine Formel in DNF ist genau dann erfüllbar, wenn eines ihrer Monome erfüllbar ist. Dies ist genau
dann der Fall, wenn das Monom kein Atom gleichzeitig negiert und nichtnegiert enthält.
}

Mögliches Verfahren:
\begin{itemize}
\item Führe logisches Problem auf (Un)Erfüllbarkeit zurück
\item Bilde die DNF und prüfe jedes Monom auf komplementäre Literale
\end{itemize}
\emph{Nachteil:} Die DNF kann ebenfalls exponentiell groß werden (muss sie aber nicht in jedem Fall)

\end{frame}

\begin{frame}[t]\frametitle{Schließen mit KNF}

Man kann Widerlegbarkeit/Allgemeingültigkeit aus KNF ablesen:\medskip

\theobox{Satz: Eine Formel in KNF ist genau dann widerlegbar, wenn eine ihrer Klauseln widerlegbar ist. Dies ist genau
dann der Fall, wenn die Klausel kein Atom gleichzeitig negiert und nichtnegiert enthält.
}

Mögliches Verfahren:
\begin{itemize}
\item Führe logisches Problem auf Allgemeingültigkeit zurück
\item Bilde die KNF und prüfe jede Klausel auf komplementäre Literale
\end{itemize}
\emph{Nachteil:} Die KNF kann ebenfalls exponentiell groß werden (muss sie aber nicht in jedem Fall)
\bigskip\pause

\alert{Können wir durch Hilfsatome eine kleinere KNF erzeugen?}\\\pause
\redalert{Ja, aber dabei wird nur Erfüllbarkeit erhalten, nicht \ghost{Allgemeingültigkeit!}}\\
$\leadsto$ nicht geeignet, um das Problem effizienter zu lösen

\end{frame}

\sectionSlide{Resolution}

\newcommand{\Aname}{Anna}
\newcommand{\Bname}{Barbara}
\newcommand{\Cname}{Chris}

\begin{frame}\frametitle{Resolution}

\emph{Ziel:} Verfahren, um \redalert{(Un)Erfüllbarkeit} einer \redalert{KNF} zu bestimmen
\bigskip

Leider ist (Un)Erfüllbarkeit nicht an einzelnen Klauseln erkennbar.
\medskip

\examplebox{Beispiel: Die Logelei aus Vorlesung 21 haben wir mit den Formeln $\mathcal{F}=\{A\leftrightarrow \neg B, B\leftrightarrow \neg C,C\leftrightarrow (\neg A\wedge \neg B)\}$ dargestellt. Wir wollen beweisen, dass
\Bname{} die Wahrheit sagt, also $\mathcal{F}\models B$. Dazu kann man zeigen, dass $\mathcal{F}\cup\{\neg B\}$ unerfüllbar
ist. Bei Anwendung des Distributivgesetztes erhalten wir die folgende KNF:
\begin{align*}
(\neg A\vee \neg B)\wedge (A\vee B)&\wedge{}\\
(\neg B\vee \neg C)\wedge (B\vee C)&\wedge{}\\
(\neg C\vee \neg A)\wedge(A\vee B\vee C)&\wedge{}\\
\neg B
\end{align*}
}

\end{frame}

\begin{frame}\frametitle{Die Klauselform}

Wenn man grundsätzlich mit KNF arbeitet, dann bieten sich einige syntaktische Vereinfachungen an:
\begin{itemize}
\item Eine Klausel $L_1\vee\ldots\vee L_n$ wir dargestellt als Menge $\{L_1,\ldots,L_n\}$
\item Eine Konjunktion von Klauseln $K_1\wedge\ldots\wedge K_\ell$ wird dargestellt als Menge $\{K_1,\ldots,K_\ell\}$
\end{itemize}
Eine Menge von Mengen von Literalen unter dieser Interpretation heißt \redalert{Klauselform}.
\medskip\pause

\examplebox{Beispiel: Die KNF\vspace{-3ex}
\begin{align*}
(\neg A\vee \neg B)\wedge (A\vee B)&\wedge{}\\
(\neg B\vee \neg C)\wedge (B\vee C)&\wedge{}\\
(\neg C\vee \neg A)\wedge(A\vee B\vee C)&\wedge{}\\
\neg B
\end{align*}\vspace{-5ex}

kann in Klauselform dargestellt werden als:\vspace{-1ex}
\[\{\{\neg A, \neg B\}, \{A, B\},\{\neg B, \neg C\},\{B, C\},
\{\neg C, \neg A\},\{A, B, C\},\{\neg B\}\}\]
}

\end{frame}

\begin{frame}\frametitle{Ableitungen mittels Resolution}

Beim Resolutionsverfahren leiten wir schrittweise neue Klauseln aus den gegebenen ab.
\bigskip

Ein einzelner Resolutionschritt funktioniert wie folgt:

\defbox{
Gegeben seien zwei Klauseln $K_1$ und $K_2$ für die es ein Atom $p\in\Slang{P}$ gibt mit $p\in K_1$ und $\neg p\in K_2$.
Die \redalert{Resolvente von $K_1$ und $K_2$ bezüglich $p$} ist die Klausel \alert{$(K_1\setminus\{p\})\cup(K_2\setminus\{\neg p\})$}.\medskip

Eine Klausel $R$ ist eine \redalert{Resolvente einer Klauselmenge $\mathcal{K}$} wenn $R$ Resolvente zweier Klauseln  $K_1,K_2\in\mathcal{K}$ ist.
}\pause

\examplebox{Beispiel: Resolventen für die Menge $\{\{\neg A, \neg B\},\allowbreak \{A, B\},\allowbreak\{\neg B, \neg C\},\allowbreak\{B, C\},\allowbreak
\{\neg C, \neg A\},\allowbreak\{A, B, C\},\allowbreak\{\neg B\}\}$ sind z.B.
\begin{itemize}
\item $\{B,\neg A\}$ aus den Klauseln $\{B, C\}$ und $\{\neg C,\neg A\}$
\item $\{A\}$ aus den Klauseln $\{A, B\}$ und $\{\neg B\}$
\item $\{B,\neg B\}$ aus den Klauseln $\{B,C\}$ und $\{\neg B, \neg C\}$
\end{itemize}

}

\end{frame}

\begin{frame}\frametitle{Bedeutung von Resolutionsschritten}

Wir betrachten Klauseln $\{L_1,\ldots,L_n,p\}$ und $\{\neg p,M_1,\ldots,M_\ell\}$:\pause
\begin{itemize}
\item $\{L_1,\ldots,L_n,p\}\equiv (L_1\vee\ldots\vee L_n\vee p) \equiv \alert{(\neg L_1\wedge\ldots\wedge\neg L_n)\to p}$\pause
\item $\{\neg p,M_1,\ldots,M_\ell\}\equiv (\neg p\vee M_1\vee\ldots\vee M_\ell)\equiv \alert{p\to (M_1\vee\ldots\vee M_\ell)}$\pause
\end{itemize}\medskip

Daraus folgt unmittelbar \alert{$(\neg L_1\wedge\ldots\wedge\neg L_n)\to (M_1\vee\ldots\vee M_\ell)$}\\
$\leadsto$ dies entspricht der Klausel $\{L_1,\ldots,L_n,M_1,\ldots,M_\ell\}$\pause
\bigskip

\theobox{Satz: Wenn $R$ Resolvente der Klauseln $K_1$ und $K_2$ ist, dann gilt $\{K_1,K_2\}\models R$.
}

Resolutionsschritte produzieren also logische Schlüsse. Diese Eigenschaft einer Ableitungsregel wird \redalert{Korrektheit} genannt ("`jede Ableitung ist tatsächlich eine logische Konsequenz"').


\end{frame}

\begin{frame}\frametitle{Die leere Klausel}

Die Resolvente kann eine leere Menge $\emptyset$ sein.\\[1ex]

\examplebox{Beispiel: Die Resolvente der Klauseln $\{p\}$ und $\{\neg p\}$ ist leer.
}\medskip

\emph{Was bedeutet so eine \redalert{leere Klausel}?}\pause
\begin{itemize}
\item Klauseln sind Disjunktionen
\item Die leere Klausel entspricht einer Disjunktion von $0$ Literalen
\item Dieser Ausdruck sollte das neutrale Element der Disjunktion sein, d.h. immer den Wert $\myfalse$ annehmen
\end{itemize}
$\leadsto$ Wir bezeichnen die leere Klausel mit $\bot$.
\bigskip\pause

\emph{Was bedeutet es, wenn $\bot$ in Klauselform auftaucht?}\pause
\begin{itemize}
\item Die Klauselform beschreibt eine Konjunktion von Klauseln
\item Wenn eine der Klauseln $\bot$ ist, dann ist die gesamte Formel unerfüllbar
\end{itemize}
$\leadsto$ Die Ableitung von $\bot$ zeigt Unerfüllbarkeit.

\end{frame}

\begin{frame}\frametitle{Das Resolutionskalkül}

Wir können damit das gesamte Verfahren angeben:

\codebox{\emph{Resolution}\\
Gegeben: Formel $\mathcal{F}$ in Klauselform\\
Gesucht: Ist $\mathcal{F}$ erfüllbar oder unerfüllbar?\\[1ex]

\begin{enumerate}[(1)]
\item Finde ein Klauselpaar $K_1,K_2\in \mathcal{F}$ mit Resolvente $R\notin \mathcal{F}$
\item Setze $\mathcal{F}\defeq \mathcal{F}\cup\{R\}$
\item Wiederhole Schritt (1) und (2) bis keine neuen Resolventen gefunden werden können
\item Falls $\bot\in \mathcal{F}$, dann gib "`unerfüllbar"' aus;\\andernfalls gib "`erfüllbar"' aus
\end{enumerate}
}\pause

\emph{Beobachtung:} Unerfüllbarkeit steht fest, sobald $\bot$ abgeleitet \ghost{wurde}\\
$\leadsto$ dann kann man das Verfahren frühzeitig abbrechen
\medskip

Erfüllbarkeit kann dagegen erst erkannt werden, wenn alle Resolventen erschöpfend gebildet worden sind

\end{frame}

\begin{frame}\frametitle{Beispiel}

\begin{tabular}{rlll}
(1) & $\{\neg A, \neg B\}$ && \textcolor{devilscss}{\Aname{} oder \Bname{} lügen.}\\
(2) & $\{A, B\}$  && \textcolor{devilscss}{\Aname{} oder \Bname{} sagen die Wahrheit.} \\
(3) & $\{\neg B, \neg C\}$  && \textcolor{devilscss}{\Bname{} oder \Cname{} lügen.} \\
(4) & $\{B, C\}$ && \textcolor{devilscss}{\Bname{} oder \Cname{} sagen die Wahrheit.} \\
(5) & $\{\neg C, \neg A\}$ && \textcolor{devilscss}{\Cname{} oder \Aname{} lügen.} \\
(6) & $\{A, B, C\}$ && \textcolor{devilscss}{Jemand sagt die Wahrheit.}  \\
(7) & $\{\neg B\}$ && \textcolor{devilscss}{\Bname{} lügt.} \\\pause
(8) & $\{C\}$   &  (4)+(7) & \textcolor{devilscss}{\Cname{} sagt die Wahrheit.} \\\pause
(9) & $\{\neg A\}$ & (8)+(5) & \textcolor{devilscss}{\Aname{} lügt.}\\\pause
(10) & $\{B\}$  & (2)+(9) & \textcolor{devilscss}{\Bname{} sagt die Wahrheit.} \\\pause
(11) & $\bot$ & (10)+(7) & \textcolor{devilscss}{Widerspruch.} 
\end{tabular}\bigskip

$\leadsto$ Es gilt $\{\{\neg A, \neg B\},\{A, B\},\{\neg B, \neg C\},\{B, C\},\{\neg C, \neg A\},\{A, B, C\}\}\models B$

\end{frame}

\begin{frame}[t]\frametitle{Resolution: Eigenschaften}

Resolution eignet sich tatsächlich zum logischen Schließen:\medskip

\theobox{Satz: Das Resolutionskalkül hat die folgenden Eigenschaften:
\begin{itemize}
\item \alert{Korrektheit:} Wenn $\bot$ aus $\mathcal{F}$ abgeleitet wird, dann gilt $\mathcal{F}\models\bot$.
\item \alert{Vollständigkeit:} Wenn $\mathcal{F}\models\bot$, dann kann $\bot$ aus $\mathcal{F}$ abgeleitet werden.
\item \alert{Terminierung:} Das Verfahren endet nach endlich vielen Schritten.
\end{itemize}
}\pause

\emph{Beweis:} \alert{Korrektheit} haben wir schon für einzelne Resolutionsschritte gezeigt.
Die Behauptung gilt daher auch, wenn beliebig viele Schritte nacheinander ausgeführt werden (Induktion
mit Hypothese: "`Die neu abgeleitete Klauselmenge folgt aus der ursprünglichen."').

\end{frame}

\begin{frame}[t]\frametitle{Resolution: Eigenschaften (2)}

Resolution eignet sich tatsächlich zum logischen Schließen:\medskip

\theobox{Satz: Das Resolutionskalkül hat die folgenden Eigenschaften:
\begin{itemize}
\item \alert{Korrektheit:} Wenn $\bot$ aus $\mathcal{F}$ abgeleitet wird, dann gilt $\mathcal{F}\models\bot$.
\item \alert{Vollständigkeit:} Wenn $\mathcal{F}\models\bot$, dann kann $\bot$ aus $\mathcal{F}$ abgeleitet werden.
\item \alert{Terminierung:} Das Verfahren endet nach endlich vielen Schritten.
\end{itemize}
}

\emph{Beweis:} \alert{Terminierung} ist leicht zu sehen.\pause{}
Jede neu abgeleitete Klausel enthält nur Literale,
die auch in der ursprünglichen Klauselmenge vorkamen. Die Zahl dieser Literale ist endlich; wir bezeichnen sie mit
$\ell$. Es gibt nur $2^\ell$ Klauseln mit diesen Literalen. Man kann also weniger als $2^\ell$ Resolventen hinzufügen bevor das Verfahren terminiert.

\end{frame}

\begin{frame}[t]\frametitle{Resolution: Eigenschaften (3)}

Resolution eignet sich tatsächlich zum logischen Schließen:\medskip

\theobox{Satz: Das Resolutionskalkül hat die folgenden Eigenschaften:
\begin{itemize}
\item \alert{Korrektheit:} Wenn $\bot$ aus $\mathcal{F}$ abgeleitet wird, dann gilt $\mathcal{F}\models\bot$.
\item \alert{Vollständigkeit:} Wenn $\mathcal{F}\models\bot$, dann kann $\bot$ aus $\mathcal{F}$ abgeleitet werden.
\item \alert{Terminierung:} Das Verfahren endet nach endlich vielen Schritten.
\end{itemize}
}

\emph{Beweis:} \alert{Vollständigkeit} ist etwas aufwändiger.\medskip

Vorgehen: Wir zeigen, wie
man eine erfüllende Zuweisung finden kann, wenn die leere Klausel nicht abgeleitet wurde.

\end{frame}

\begin{frame}\frametitle{Widerspruchsfreie Resolution $\leadsto$ Modell}

% Wenn $\bot$ nicht abgeleitet wurde, dann kann man ein Modell der Formel wie folgt finden.
% \medskip

Für jedes Atom $p$, das in der Formel vorkommt, bestimmen wir einen Wahrheitswert $w(p)$:\medskip

\codebox{
Gegeben: widerspruchsfreie Klauselmenge $\mathcal{F}$ nach Resolution\\[1ex]

Für jedes Atom $p$ in $\mathcal{F}$ (in beliebiger Reihenfolge):
\begin{enumerate}[(a)]
\item Wenn $\{p\}\in\mathcal{F}$, definiere $w(p)\defeq\mytrue$
\item Wenn $\{\neg p\}\in\mathcal{F}$, definiere $w(p)\defeq\myfalse$
\item Wenn weder $\{p\}\in\mathcal{F}$ noch $\{\neg p\}\in\mathcal{F}$, dann setze $w(p)\defeq\mytrue$
und $\mathcal{F}\defeq\mathcal{F}\cup\{\{p\}\}$ und führe nochmals Resolution durch, bis keine weiteren Klauseln abgeleitet
werden.
\end{enumerate}
}\smallskip

Die so bestimmte Wertzuweisung $w$ ist ein Modell für $\mathcal{F}$.\bigskip

{\tiny  Beweisskizze:
\begin{enumerate}[(1)]
\item In Schritt (c) wird nie die leere Klausel abgeleitet (Intuition: Dazu wäre eine Klausel $\{\neg p\}$ nötig, die es aber in diesem Fall nicht geben darf und die auch nicht neu abgeleitet werden kann $\leadsto$ Induktion)
\item Dank (1) enthält $\mathcal{F}$ auch nach Bestimmung aller Werte keine leere Klausel (einfache Induktion)
\item Daher werden alle Klauseln in $\mathcal{F}$ erfüllt (Wenn eine nicht erfüllt wäre, dann könnte man aus ihr unter Verwendung der einelementigen Klauseln $\bot$ ableiten, was aber laut (2) nicht abgeleitet wird)
\end{enumerate}}
% 
% Man kann zeigen: $w\models\mathcal{F}$, insbesondere ist $w$ ein Modell der ursprünglichen Formel

% Wir benötigen eine Wertzuweisung für jedes Atom, das in der Formel vorkommt.
% 
% \begin{itemize}
% \item Wenn eine Klausel $\{p\}$ oder $\{\neg p\}$ abgeleitet wurde, dann kann man die
% Wertzuweisung ablesen ($p\mapsto\mytrue$ bzw. $p\mapsto\myfalse$)
% \item Wenn für ein Atom $p$ weder $\{p\}$ noch $\{\neg p\}$ abgeleitet wurde, dann können wir eines
% davon frei wählen (z.B. $\{p\}$) und mit dieser zusätzlichen Klausel Resolution durchführen (dies kann
% keinen Widerspruch erzeugen!)
% \end{itemize}

\end{frame}

\begin{frame}[t]\frametitle{Resolution: Eigenschaften (4)}

Resolution eignet sich tatsächlich zum logischen Schließen:\medskip

\theobox{Satz: Das Resolutionskalkül hat die folgenden Eigenschaften:
\begin{itemize}
\item \alert{Korrektheit:} Wenn $\bot$ aus $\mathcal{F}$ abgeleitet wird, dann gilt $\mathcal{F}\models\bot$.
\item \alert{Vollständigkeit:} Wenn $\mathcal{F}\models\bot$, dann kann $\bot$ aus $\mathcal{F}$ abgeleitet werden.
\item \alert{Terminierung:} Das Verfahren endet nach endlich vielen Schritten.
\end{itemize}
}

\emph{Beweis:} \alert{Vollständigkeit} ist etwas aufwändiger.\medskip

Vorgehen: Wir zeigen, wie
man eine erfüllende Zuweisung finden kann, wenn die leere Klausel nicht abgeleitet wurde.\qed
\bigskip

\emph{Anmerkung:} "`Vollständigkeit"' bezieht sich nur auf die Ableitung von $\bot$. Resolution kann nicht
jede beliebige Klausel ableiten, die logisch folgt!

\end{frame}

\begin{frame}\frametitle{Beispiel: widerspruchsfreie Resolution}

\begin{minipage}{4cm}
\tiny
\begin{tabular}[t]{rlll}
(1) & $\{\neg A, \neg B\}$ &\\
(2) & $\{A, B\}$  &\\
(3) & $\{\neg B, \neg C\}$  &\\
(4) & $\{B, C\}$ &\\
(5) & $\{\neg A,\neg C\}$ &\\
(6) & $\{A, B, C\}$ &\\
%
(7) & $\{B,\neg B\}$ & (2)+(1)\\
(8) & $\{B,\neg C\}$ & (2)+(5)\\
(9) & $\{A,\neg A\}$ & (2)+(1)\\
(10) & $\{A,\neg C\}$ & (2)+(3)\\
(11) & $\{\neg A,C\}$ & (4)+(1)\\
(12) & $\{C,\neg C\}$ & (4)+(3)\\
(13) & $\{\neg A,B\}$ & (4)+(5)\\
(14) & $\{B,\neg B,C\}$ & (6)+(1)\\
(15) & $\{B,C,\neg C\}$ & (6)+(5)\\
(16) & $\{A,\neg A, C\}$ & (6)+(1)\\
(17) & $\{A,C, \neg C\}$ & (6)+(3)\\
(18) & $\{A,B,\neg B\}$ & (6)+(3)\\
(19) & $\{A,\neg A, B\}$ & (6)+(3)\\
%
(20) & \textcolor<2->{darkred}{$\{B\}$} & (2)+(13)\\
(21) & \textcolor<2->{darkred}{$\{\neg C\}$} & (8)+(3)\\
(22) & \textcolor<2->{darkred}{$\{\neg A\}$} & (13)+(1)\\
(23) & $\{\neg A,B,\neg B\}$ & (14)+(5)\\
(24) & $\{\neg A,C,\neg C\}$ & (15)+(1)
\end{tabular}\end{minipage}
\begin{minipage}{5cm}{\tiny
\begin{tabular}[t]{rlll}
(25) & $\{A,\neg A,\neg B\}$ & (16)+(3)\\
(26) & $\{\neg B,C,\neg C\}$ & (17)+(1)\\
(27) & $\{B,\neg B,\neg C\}$ & (18)+(5)\\
(28) & $\{A,\neg A,\neg C\}$ & (19)+(3)\\
%
(29) & $\{\neg A,\neg B,\neg C\}$ & (23)+(3)\\
(30) & $\{A,\neg A,B,\neg B\}$ & (6)+(29)\\
(31) & $\{A,\neg A,C,\neg C\}$ & (6)+(29)\\
(32) & $\{B,\neg B,C,\neg C\}$ & (6)+(29)\\
%
(33) & $\{A,\neg A, B,\neg B,C,\neg C\}$ & (30)+(31)\\
(34) & $\{\neg A, B,\neg B,C,\neg C\}$ & (33)+(22)\\
(35) & $\{A, \neg A, B,C,\neg C\}$ & (20)+(33)\\
(36) & $\{A, \neg A, B,\neg B,\neg C\}$ & (33)+(21)\\
(37) & $\{\neg A, B,C,\neg C\}$ & (20)+(34)\\
(38) & $\{\neg A, B,\neg B,\neg C\}$ & (34)+(21)\\
(39) & $\{A, \neg A, B,\neg C\}$ & (35)+(21)\\
(40) & $\{\neg A, B,\neg C\}$ & (39)+(22)\\
(41) & $\{A, B,\neg B,C,\neg C\}$ & (2)+(33)\\
(42) & $\{A,\neg A, \neg B,C,\neg C\}$ & (33)+(3)\\
(43) & $\{A,\neg A, B,\neg B,C\}$ & (4)+(33)\\
(44) & \ldots und so weiter
\end{tabular}}\pause
\vspace{5mm}

Es ergibt sich ein Modell $w$ mit
$w(A)=\myfalse$, $w(B)=\mytrue$ und $w(C)=\myfalse$.
\end{minipage}

\end{frame}

\begin{frame}\frametitle{Resolution: Komplexität}

Leider können \redalert{im schlimmsten Fall exponentiell} viele Resolventen gebildet werden
(wobei die Formel erfüllbar ist, so dass kein frühzeitiger Abbruch möglich ist)
\bigskip

\textcolor{devilscss}{(Die Verwendung von Hilfsatomen zur Ermittlung einer linear großen Klauselform löst dieses Problem nicht)}
\bigskip

\begin{itemize}
\item Bezüglich der algorithmischen Komplexität ist Resolution schlimmstenfalls
\alert{nicht besser als andere Verfahren}.
\item Das hier vorgestellte einfachste Verfahren ist \alert{hoffnungslos ineffizient}.
\item Praktische Implementierungen verwenden aber sehr \alert{viele weitere Optimierungen}.
\end{itemize}

\end{frame}

\begin{frame}\frametitle{Aussagenlogisches Schließen in der Praxis}

Systeme zum aussagenlogischen Schließen heißen \redalert{SAT~Solver}.
\bigskip

Moderne SAT-Solver \ldots
\begin{enumerate}[$\ldots$]
\item sind \alert{stark optimiert} und können Probleme mit tausenden oder gar hunderttausenden von Atomen lösen
\item verwenden in der Regel \alert{keine (reine) Resolution}, bilden aber manchmal Resolventen
\item werden auch \alert{jenseits der Aussagenlogik} verwendet, indem Probleme zunächst in Aussagenlogik übersetzt werden
\end{enumerate}\bigskip

Resolution ist dennoch bedeutsam, da sie gut auf Prädikatenlogik und verwandte Logiken anwendbar ist, was für andere aussagenlogische Verfahren nicht zutrifft.

\end{frame}

\sectionSlide{Horn-Logik}

\begin{frame}\frametitle{Horn-Logik}

Bisher waren alle unsere Ansätze für aussagenlogisches Schließen exponentiell -- geht es auch einfacher?
\bigskip\pause

\emph{Idee:} Beschränke die Form von Formeln
\bigskip

\defbox{Eine \redalert{Horn-Klausel}${}^*$ ist eine Klausel, die höchstens ein nicht-negiertes Literal enthält.
Eine \redalert{Horn-Formel} ist eine Formel in KNF, welche nur Horn-Klauseln enthält.\\
{\tiny ${}^*$) nach Alfred Horn, 1918--2001, US-amerikanischer Mathematiker}}

\pause
\examplebox{Beispiele:
\begin{itemize}
\item $\neg p\vee \neg q \vee r$ ist eine Horn-Klausel
\item $q\vee p$ ist keine Horn-Klausel
\item $p$ ist eine Horn-Klausel
\item $\neg p\vee \neg q$ ist eine Horn-Klausel
\item $p\vee p$ ist keine Horn-Klausel (aber äquivalent zu einer)
\end{itemize}
}

\end{frame}

\begin{frame}\frametitle{Horn-Klauseln als Implikationen}

Jede Horn-Klausel kann als eine Implikation ohne Negation und Disjunktion ausgedrückt werden
\bigskip

\examplebox{Beispiele:
\begin{align*}
\neg p\vee \neg q \vee r \qquad \equiv &&(p\wedge q) &\to r\\
p \qquad \equiv &&\top&\to p\\
\neg p\vee \neg q \qquad \equiv &&(p\wedge q)&\to \bot
\end{align*}
}

\begin{itemize}
\item Wir verwenden $\top$ für die leere Prämisse und $\bot$ für die leere Konsequenz (weil so die gewünschte logische Äquivalenz gilt)
\item Als Implikationen geschriebene Horn-Klauseln werden oft als \redalert{(Horn-)Regeln} bezeichnet
\item Es ist üblich, die Klammern der Konjunktion in der Prämisse wegzulassen
\end{itemize}

\end{frame}

\begin{frame}\frametitle{Ausblick}

\emph{Nächste Vorlesung:} Wir werden logisches Schließen für Horn-Regeln noch etwas genauer betrachten
\begin{itemize}
\item Resolution kann in diesem Fall spezialisiert werden
\item Dadurch erhält man polynomielle Algorithmen
\end{itemize}

\end{frame}


\begin{frame}\frametitle{Zusammenfassung und Ausblick}

Das \redalert{logische Schließen} umfasst viele Fragen (meist Entscheidungsprobleme), die aber oft
auf Erfüllbarkeit zurückgeführt werden können
\bigskip

\redalert{Resolution} ist ein bekanntes Widerlegungskalkül, welches über die Aussagenlogik hinaus von Bedeutung ist
\bigskip

Die betrachteten Beweisverfahren für Aussagenlogik benötigen schlimmstenfalls \redalert{exponentiell viel Zeit}; bei
\redalert{Horn-Formeln} gibt es dagegen polynomielle Schlussfolgerungsalgorithmen\bigskip

\anybox{yellow}{
Offene Fragen:
\begin{itemize}
\item Gibt es auch sub-exponentielle Algorithmen für aussagenlogisches Schließen?
\item Was hat Aussagenlogik mit Sprachen, Berechnung und TMs zu tun?
\end{itemize}
}

\end{frame}


\end{document}
