
\begin{exercise}
	\begin{enumerate}
\item[S15)] Betrachten Sie die Grammatik $G=(\{ S,U,X,T,V,W,Y,D,E,A,B,C\},\Sigma, P,S)$ \\mit
  $\Sigma=\{a,b,c\}$ und
\begin{eqnarray*}
P&=&\{
      S\longrightarrow UT,  S\longrightarrow VW,
      U\longrightarrow XB, U\longrightarrow AB,\\
&&
      X \longrightarrow AU,
      T \longrightarrow TC, T \longrightarrow c,
      V \longrightarrow AV,\\
&&V \longrightarrow a,
      W \longrightarrow BY, W \longrightarrow BC,
      Y \longrightarrow WC,\\
&&    D \longrightarrow BC, D \longrightarrow BB, D \longrightarrow  b,
      E \longrightarrow AB,\\
&&  E \longrightarrow AA,
      A\longrightarrow a,
      B\longrightarrow b,
      C\longrightarrow c\}\;.
\end{eqnarray*}
   Verwenden Sie den CYK-Algorithmus (mit der Matrix-Notation aus der
    Vorlesung), um f\"ur die W\"orter $w_1 = aabcc$ und $w_2 = aabbcc$ zu entscheiden, ob
    $w_i\in L(G)$ ist.\\

\item[S16)] Gegeben sind das Wort $w=aaaab$ und die Grammatik $G=(V,\Sigma ,P,S)$ mit\\
$V=\{S,A,B,C\}, \Sigma =\{a,b\}$ und
\begin{eqnarray*}
P&=&\{ S\longrightarrow AB, S\longrightarrow BC, S\longrightarrow bab,\\
&& A\longrightarrow BA, A\longrightarrow a,\\
&& B\longrightarrow ABC, B\longrightarrow b,\\
&& C\longrightarrow AB, C\longrightarrow a, C\longrightarrow \varepsilon\}\;.
\end{eqnarray*}

\begin{enumerate}
\item[a)] Transformieren Sie die Grammatik $G$ in eine $\varepsilon$-freie Grammatik $G'$.
\item[b)] Transformieren Sie die Grammatik $G'$ in ihre {\it
    Chomsky}-Normalform.
\item[c)] Entscheiden Sie mithilfe des
  Cocke-Younger-Kasami-Algorithmus, ob $w\in L(G)$
  gilt.
\end{enumerate}
\end{enumerate}

\end{exercise}
