
\begin{exercise}
Beweisen oder widerlegen Sie unter Verwendung von Resultaten aus der Vorlesung folgende Aussagen.
\begin{enumerate}
\item F\"ur die Grammatik $G=(\{S,X,Y,Z\},\{a,b\},\{S\rightarrow Y,\;X\rightarrow b,\;Y\rightarrow aYYb,\;aY\rightarrow aZ,\;ZY\rightarrow ZX,\;Z\rightarrow a\},S)$ gilt: $abab\in L(G)$.
\item Kann eine Sprache $L$ von einem DFA erkannt werden, so gibt es auch einen
  $\varepsilon$-NFA $\mathcal M$ mit $L({\mathcal M})=L$.
\item F\"ur jeden NFA $\mathcal M$ mit Wort\"uberg\"angen gibt es einen \"aquivalenten NFA.
\item Es gibt eine regul\"are Sprache, f\"ur welche die Anzahl der \"Aquivalenzklassen der zugeh\"origen {\emph{Nerode}}-Rechtskongruenz endlich ist.
\item Wenn es f\"ur eine Sprache $L$ ein $n\in \mathbb N$ gibt, so dass die {\emph{Nerode}}-Rechtskongruenz $\simeq_L$ höchstens $n$ Äquivalenzklassen hat, so
  kann $L$ von einem DFA erkannt werden.
\item F\"ur jede Sprache $L$ gilt: $L = \bigcup\limits_{u \in L} [u]_{\simeq_{L}}\;$, d.\,h. $L$ ist die Vereinigung von $\simeq_{L}$-Klassen.
\end{enumerate}
\end{exercise}

