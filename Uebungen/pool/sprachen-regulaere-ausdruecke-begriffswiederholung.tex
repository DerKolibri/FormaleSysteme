
\begin{exercise}
\begin{enumerate}
\item[S11)] Sei $\Sigma_1 = \{a,b\}$ und $\Sigma_2 =\{a,b,c\}$. Geben Sie für jede der folgenden Sprachen
  $L_i$ einen regulären Ausdruck $\alpha_i$ mit $L_i=L(\alpha_i)$ an. Begr\"unden
  Sie die von Ihnen gew\"ahlten regulären Ausdrücke $\alpha_i$.
                                %
  \begin{enumerate}
  \item $L_1 = \{ w\in \Sigma_1^* \mid w \text{ beginnt mit}\; a \text{ und }
    |w|_b \text{ ist gerade}\}$
  \item $L_2 = \{ w\in \Sigma_2^* \mid w \text{ beginnt mit}\; a \text{ und }
    |w|_b \text{ ist gerade}\}$
  \item $L_3 = \{ w\in \Sigma_1^* \mid \text{es gibt kein} \;u,v\in
    \Sigma_1^* \text{ mit} \;w=uaav\}$
  \item $L_4 = \{ w\in \Sigma_2^* \mid \text{es gibt kein} \;u,v\in \Sigma_2^* \text{ mit} \;w=uaav\}$

\end{enumerate}
\item[S12)] Wiederholen Sie die Begriffe Potenzmengenkonstruktion, erreichbarer Zustand, äquivalente Zustände, Quotientenautomat, reduzierter Automat und
  {\emph{Nerode}}-Rechtskongruenz.
\end{enumerate}

\end{exercise}
