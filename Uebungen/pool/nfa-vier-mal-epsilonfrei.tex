\begin{exercise}
  Gegeben sind die Automaten
  \begin{align*}
     \mathcal{M}_1 &=(\{q_0,\dots,q_3\},\{a,b\},\delta_1,\{q_0\},\{q_2,q_3\}),
    &\mathcal{M}_2 &=(\{p_0,\dots,p_3\},\{a,b\},\delta_2,\{p_0\},\{p_3\}),\\
     \mathcal{M}_3 &=(\{s_0,\dots,s_3\},\{a,b\},\delta_3,\{s_0\},\{s_3\})\ \text{und}
    &\mathcal{M}_4 &=(\{t_0,\dots,t_5\},\{a,b\},\delta_4,\{t_0\},\{t_3,t_5\})
  \end{align*}
  mit

  \begin{center}
    \begin{tikzpicture}[%
        ->,
        >=stealth',
        semithick,
        initial text=,
        shorten <=2pt,
        shorten >=2pt,
        auto,
        on grid,
        node distance=7ex and 6em,
        every state/.style={minimum size=0pt,inner sep=2pt,text height=1.5ex,text depth=.25ex},
        bend angle=30]
      \node                  (delta_1)                          {$\delta_1$:};
      \node[state,initial]   (q_0)     [below right=of delta_1] {$q_0$};
      \node[state]           (q_1)     [right=of q_0]           {$q_1$};
      \node[state,accepting] (q_2)     [right=of q_1]           {$q_2$};
      \node[state,accepting] (q_3)     [right=of q_2]           {$q_3$};
      \path[->] (q_0) edge              node {$a$}   (q_1)
                (q_1) edge              node {$a$}   (q_2)
                (q_2) edge [loop above] node {$a$}   (q_2)
                      edge              node {$b$}   (q_3)
                (q_3) edge [loop above] node {$a$}   (q_3);
      \node                  (delta_2) [below left=of q_0]      {$\delta_2$:};
      \node[initial,state]   (p_0)     [below right=of delta_2] {$p_0$};
      \node[state]           (p_1)     [right=of p_0]           {$p_1$};
      \node[state]           (p_2)     [right=of p_1]           {$p_2$};
      \node[state,accepting] (p_3)     [right=of p_2]           {$p_3$};
      \path[->] (p_0) edge [loop above]  node {$a$} (p_0)
                (p_0) edge               node {$b$} (p_1)
                (p_1) edge               node {$b$} (p_2)
                (p_1) edge [bend left]   node {$a$} (p_3)
                (p_2) edge               node {$a$} (p_3)
                (p_3) edge [loop above]  node {$a$} (p_3);
      \node                  (delta_3) [below left=of p_0]      {$\delta_3$:};
      \node[state,initial]   (s_0)     [below right=of delta_3] {$s_0$};
      \node[state]           (s_1)     [right=of s_0]           {$s_1$};
      \node[state]           (s_2)     [above right=of s_1]     {$s_2$};
      \node[state,accepting] (s_3)     [below right=of s_1]     {$s_3$};
      \path[->] (s_0) edge [loop above] node {$a$}   (s_0)
                (s_0) edge              node {$b$}   (s_1)
                (s_1) edge              node {$a$}   (s_2)
                (s_2) edge              node {$a$}   (s_3)
                (s_3) edge              node {$b$}   (s_1);
      \node                  (delta_4) [below left=of s_0]      {$\delta_4$:};
      \node[initial,state]   (t_0)     [below right=of delta_4] {$t_0$};
      \node[initial,state]   (t_0)     [below right=of delta_4] {$t_0$};
      \node[state]           (t_1)     [right=of t_0]           {$t_1$};
      \node[state]           (t_2)     [below=of t_1]           {$t_2$};
      \node[state,accepting] (t_3)     [right=of t_1]           {$t_3$};
      \node[state]           (t_4)     [right=of t_3]           {$t_4$};
      \node[state,accepting] (t_5)     [below=of t_4]           {$t_5$};
      \path[->] (t_0) edge              node {$a,b$} (t_1)
                (t_1) edge [bend left]  node {$a$}   (t_2)
                (t_1) edge              node {$b$}   (t_3)
                (t_2) edge [bend left]  node {$a$}   (t_1)
                (t_3) edge              node {$b$}   (t_4)
                (t_4) edge              node {$b$}   (t_5)
                (t_5) edge [loop left]  node {$a$}   (t_5);
    \end{tikzpicture}   
  \end{center}

  \vspace*{2ex}

  \begin{enumerate}
    \item Konstruieren Sie einen $\varepsilon$-freien NFA~$\mathcal{M}_a$ mit
      $L(\mathcal{M}_a)=L(\mathcal{M}_1)\cap L(\mathcal{M}_2)$.
      Dabei dürfen Sie sich auf die vom Startzustand erreichbaren Zustände
      beschränken.
    \item Konstruieren Sie einen $\varepsilon$-freien NFA~$\mathcal{M}_b$ mit
      $L(\mathcal{M}_b)=L(\mathcal{M}_1)^*$.
    \item Konstruieren Sie einen $\varepsilon$-freien NFA~$\mathcal{M}_c$ mit
      $L(\mathcal{M}_c)=L(\mathcal{M}_3)\cup L(\mathcal{M}_4)$.
    \item Konstruieren Sie einen $\varepsilon$-freien NFA~$\mathcal{M}_d$ mit
      $L(\mathcal{M}_d)=L(\mathcal{M}_3)\circ L(\mathcal{M}_4)$.
  \end{enumerate}
\end{exercise}



