\begin{exercise}
  \begin{enumerate}
  \item[S5)] Es sei der $\varepsilon$-NFA
    $\mathcal M=(\{q_0,\ldots ,q_4\},\{a,b\}, \delta , \{q_0\}, \{q_2\})$
    gegeben mit $\delta$ wie unten graphisch dargestellt:\\[1ex]
    \begin{center}
      \begin{exercise}
  \begin{enumerate}
  \item[S5)] Es sei der $\varepsilon$-NFA
    $\mathcal M=(\{q_0,\ldots ,q_4\},\{a,b\}, \delta , \{q_0\}, \{q_2\})$
    gegeben mit $\delta$ wie unten graphisch dargestellt:\\[1ex]
    \begin{center}
      \begin{exercise}
  \begin{enumerate}
  \item[S5)] Es sei der $\varepsilon$-NFA
    $\mathcal M=(\{q_0,\ldots ,q_4\},\{a,b\}, \delta , \{q_0\}, \{q_2\})$
    gegeben mit $\delta$ wie unten graphisch dargestellt:\\[1ex]
    \begin{center}
      \begin{exercise}
  \begin{enumerate}
  \item[S5)] Es sei der $\varepsilon$-NFA
    $\mathcal M=(\{q_0,\ldots ,q_4\},\{a,b\}, \delta , \{q_0\}, \{q_2\})$
    gegeben mit $\delta$ wie unten graphisch dargestellt:\\[1ex]
    \begin{center}
      \input{pool/graphics/nfa-wiederholung-einfach}
    \end{center}
    Konstruieren Sie einen zu $\mathcal M$ äquivalenten DFA $\mathcal M'$.
  \item[S6)] Es sei $\Sigma = \{a,b,c\}$. Geben Sie NFAs ${\mathcal{M}}_1$,
    ${\mathcal{M}}_2$ an mit
    \begin{enumerate}
    \item $L({\mathcal{M}}_1)=\{w\in \Sigma^* \mid (|w|_a \;\text{ist ungerade und}\;|w|_b \;\text{ist gerade}) \;\text{oder}$\\[0.5ex]
      \hspace*{3.245cm}$(\text{es gibt }u,v\in \Sigma^* \;\text{mit} \;w = u ccc v) \}$
    \item $L({\mathcal{M}}_2) =  \{w\in \Sigma^* \mid (\text{es gibt} \;u,v\in \Sigma^* \;\text{mit}\; w = u babc v)\;\text{und}$\\[0.5ex]
      \hspace*{3.28cm}$(\text{es gibt }u,v\in \Sigma^* \;\text{mit}\; w = u ccc v ) \;\text{und}$\\[0.5ex]
      \hspace*{3.3cm}$(\text{es gibt kein}\;u \in \Sigma^* \;\text{mit}\; w = au)\}$
    \end{enumerate}
  \end{enumerate}
\end{exercise}

    \end{center}
    Konstruieren Sie einen zu $\mathcal M$ äquivalenten DFA $\mathcal M'$.
  \item[S6)] Es sei $\Sigma = \{a,b,c\}$. Geben Sie NFAs ${\mathcal{M}}_1$,
    ${\mathcal{M}}_2$ an mit
    \begin{enumerate}
    \item $L({\mathcal{M}}_1)=\{w\in \Sigma^* \mid (|w|_a \;\text{ist ungerade und}\;|w|_b \;\text{ist gerade}) \;\text{oder}$\\[0.5ex]
      \hspace*{3.245cm}$(\text{es gibt }u,v\in \Sigma^* \;\text{mit} \;w = u ccc v) \}$
    \item $L({\mathcal{M}}_2) =  \{w\in \Sigma^* \mid (\text{es gibt} \;u,v\in \Sigma^* \;\text{mit}\; w = u babc v)\;\text{und}$\\[0.5ex]
      \hspace*{3.28cm}$(\text{es gibt }u,v\in \Sigma^* \;\text{mit}\; w = u ccc v ) \;\text{und}$\\[0.5ex]
      \hspace*{3.3cm}$(\text{es gibt kein}\;u \in \Sigma^* \;\text{mit}\; w = au)\}$
    \end{enumerate}
  \end{enumerate}
\end{exercise}

    \end{center}
    Konstruieren Sie einen zu $\mathcal M$ äquivalenten DFA $\mathcal M'$.
  \item[S6)] Es sei $\Sigma = \{a,b,c\}$. Geben Sie NFAs ${\mathcal{M}}_1$,
    ${\mathcal{M}}_2$ an mit
    \begin{enumerate}
    \item $L({\mathcal{M}}_1)=\{w\in \Sigma^* \mid (|w|_a \;\text{ist ungerade und}\;|w|_b \;\text{ist gerade}) \;\text{oder}$\\[0.5ex]
      \hspace*{3.245cm}$(\text{es gibt }u,v\in \Sigma^* \;\text{mit} \;w = u ccc v) \}$
    \item $L({\mathcal{M}}_2) =  \{w\in \Sigma^* \mid (\text{es gibt} \;u,v\in \Sigma^* \;\text{mit}\; w = u babc v)\;\text{und}$\\[0.5ex]
      \hspace*{3.28cm}$(\text{es gibt }u,v\in \Sigma^* \;\text{mit}\; w = u ccc v ) \;\text{und}$\\[0.5ex]
      \hspace*{3.3cm}$(\text{es gibt kein}\;u \in \Sigma^* \;\text{mit}\; w = au)\}$
    \end{enumerate}
  \end{enumerate}
\end{exercise}

    \end{center}
    Konstruieren Sie einen zu $\mathcal M$ äquivalenten DFA $\mathcal M'$.
  \item[S6)] Es sei $\Sigma = \{a,b,c\}$. Geben Sie NFAs ${\mathcal{M}}_1$,
    ${\mathcal{M}}_2$ an mit
    \begin{enumerate}
    \item $L({\mathcal{M}}_1)=\{w\in \Sigma^* \mid (|w|_a \;\text{ist ungerade und}\;|w|_b \;\text{ist gerade}) \;\text{oder}$\\[0.5ex]
      \hspace*{3.245cm}$(\text{es gibt }u,v\in \Sigma^* \;\text{mit} \;w = u ccc v) \}$
    \item $L({\mathcal{M}}_2) =  \{w\in \Sigma^* \mid (\text{es gibt} \;u,v\in \Sigma^* \;\text{mit}\; w = u babc v)\;\text{und}$\\[0.5ex]
      \hspace*{3.28cm}$(\text{es gibt }u,v\in \Sigma^* \;\text{mit}\; w = u ccc v ) \;\text{und}$\\[0.5ex]
      \hspace*{3.3cm}$(\text{es gibt kein}\;u \in \Sigma^* \;\text{mit}\; w = au)\}$
    \end{enumerate}
  \end{enumerate}
\end{exercise}
