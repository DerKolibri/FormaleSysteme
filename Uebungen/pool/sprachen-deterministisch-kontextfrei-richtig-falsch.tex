\begin{exercise}
Welche der folgenden Aussagen sind wahr und welche nicht? Begr"unden Sie Ihre
Antworten -- dabei d"urfen Sie den gesamten Stoff und alle Resultate
der Vorlesung und "Ubung verwenden.
 \begin{enumerate}
    \item[a)] Es gibt eine Sprache, die von einem nichtdeterministischen Kellerautomaten erkannt wird, nicht aber von einem deterministischen Kellerautomaten.
    \item[b)] Mithilfe des Pumping-Lemmas f\"ur kontextfreie Sprachen kann
              bewiesen werden, dass eine Sprache $L$ kontextfrei ist. 
    \item[c)] F\"ur eine beliebige Sprache $L$ gilt: $L$ ist regul\"ar, wenn es eine nat\"urliche Zahl $n_0\ge 1$ gibt, so dass sich jedes Wort $w\in L$ mit $|w|\ge n_0$ zerlegen l\"asst in $w=xyz$ mit $y\not = \varepsilon, xy^kz\in L$ f\"ur alle $k\ge 0$.
 \end{enumerate}
\end{exercise}

