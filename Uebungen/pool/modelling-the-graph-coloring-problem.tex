\begin{exercise}
  \label{logic:modelling-the-graph-coloring-problem}
  Eine \emph{$k$-Färbung} für einen endlichen Graphen $G$ ist eine
  Zuordnung der Knoten von $G$ zu Werten (\enquote{Farben}) in $\{\,1, \dots,
  k\,\}$, so dass Knoten, die in $G$ durch eine Kante verbunden sind, nicht
  denselben Wert zugeordnet bekommen.

  Geben Sie für einen endlichen Graphen $G = (V,E)$ mit $n$ Knoten und einen
  Wert $k$ eine aussagenlogische Formel $\phi_{G,k}$ an, so dass $\phi_{G,k}$
  genau dann erfüllbar ist, wenn es eine $k$-Färbung von $G$ gibt.
\end{exercise}

