
\begin{exercise}
\vspace*{0.1cm}
\begin{itemize}
\item[a)] Erklären Sie, wann zwei NFAs  $\mathcal{M}_1$ und $\mathcal{M}_2$ äquivalent sind. 
\item[b)] Geben Sie einen DFA  $\mathcal{M}'$ an, der zum NFA  $\mathcal{M}=(\{q_0,q_1,q_2\},\{a,b\},\delta, \{q_0\}, \{q_2\})$ äquivalent ist;
  für $\mathcal{M}$ ist die Übergangsfunktion $\delta$ grafisch angegeben:\\[0.5cm]
\begin{center}
  
\begin{tikzpicture}[->,
        >=stealth',
        semithick,
        initial text=,
        shorten <=2pt,
        shorten >=2pt,
        auto,
        on grid,
        node distance=5ex and 5em,
        every state/.style={minimum size=0pt,inner sep=2pt,text height=1.5ex,text depth=.25ex},
        bend angle=15]

      \begin{scope}
        \node[state, initial]   (q_0) {$q_{0}$};
        \node[state]            (q_1) [right=4 of q_0] {$q_{1}$};
        \node[state, accepting] (q_2) [below right=2 and 2 of q_0] {$q_{2}$};

        \path
        (q_0) edge[bend left] node {$b$} (q_1)
        (q_1) edge[bend left] node {$b$} (q_0)
        (q_0) edge            node[below left] {$a,b$} (q_2)
        (q_1) edge            node {$b$} (q_2);
      \end{scope}

\end{tikzpicture}

\end{center}
\end{itemize}
\end{exercise}

