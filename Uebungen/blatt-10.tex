\documentclass[german]{latteachCD}
\usepackage{mdframed}
\usepackage{amsmath}
\usepackage{amsfonts}
\usepackage{amssymb}
\usepackage{fdsymbol}
%\usepackage{wasysym}
%\usepackage{stmaryrd}
%\usepackage{fixltx2e}
%\usepackage{enumitem}
%\usepackage{extarrows}
%\newcommand{\abs}[1]{\lvert#1\rvert}

\usepackage{tikz}
\usetikzlibrary{arrows,automata,positioning,shapes,calc,decorations.pathmorphing,matrix}

\tikzstyle{automaton}=[->, >=stealth', initial text=, auto, node distance=20mm, bend angle=20, semithick, x=20mm, y=20mm]
\tikzset{
  every state/.style={
    inner sep=0pt,
    minimum size=8mm
  },
  small state/.style={
      state,
      minimum size=3mm
  },
  ellipse state/.style={
    draw,
    shape=ellipse,
    minimum width=20mm,
    minimum height=12.36mm,
    text width=14.5mm,
    inner sep=0mm,
    path picture={
      \draw (path picture bounding box.east) ellipse [x radius=9mm, y radius=9mm];
    }
  },
  accepting ellipse state/.style={
    ellipse state,
    path picture={
      \clip (path picture bounding box.east) ellipse [x radius=9.4mm, y radius=9.4mm];
      \draw[double] (path picture bounding box.east) ellipse [x radius=9mm, y radius=9mm];
      \draw[double] (path picture bounding box.center) ellipse [x radius=10mm, y radius=6.18mm];
    }
  }
}


\newcommand{\T}{\ensuremath\mathcal{M}}
\newcommand{\bin}{\ensuremath\mathit{bin}}
\newcommand{\poly}{\text{\normalfont poly}}

%%%%%%%%%%%%%%%%%%%%%%%%%%%%%%%%%%%%%%%%%%%%%%%%%%%%%%%%%%%%%%%%%%%%%%%%%%%%%%%%%%%%%%%%%%%%

\usepackage{xspace}

\author{~}
\term{Wintersemester 2017/18}
\title{\Large 10.\@ Übungsblatt}
\course{\LARGE Formale Systeme}

\usepackage{csquotes}
\usepackage{booktabs}
\usepackage{amsmath}
\usepackage{amsfonts}
\usepackage{amssymb}
\usepackage{mathtools}
\usepackage{wasysym}
\usepackage{stmaryrd}
\usepackage{enumitem}
\usepackage{tikz}
\usepackage{makecmds}

\renewcommand{\epsilon}{\varepsilon}
\renewcommand{\phi}{\varphi}
\renewcommand{\rho}{\varrho}
\renewcommand{\theta}{\vartheta}
\newcommand{\tuple}[1]{\langle{#1}\rangle} 

\newcommand{\size}[1]{\ensuremath{\lvert #1\rvert}}
\newcommand{\gdw}{\mathrel{\mathrm{gdw.}}}
\newcommand{\falls}{\mathrel{\mathrm{falls}}}

\provideenvironment{solution}{\textbf{Lösung}:}{}
\usepackage{comment}

\usepackage{etex,etoolbox}

\DeclareRobustCommand{\NN}{\ensuremath{\mathbb{N}}}

\newbool{Baader}
\newbool{Kroetzsch}
\booltrue{Kroetzsch}

\DeclareMathOperator{\Var}{Var}
\ifbool{Baader}{%
  \DeclareMathOperator{\Unt}{Unt}
  \DeclareMathOperator{\Res}{Res}
}{}
\ifbool{Kroetzsch}{%
  \DeclareMathOperator{\Unt}{Sub}
  \DeclareMathOperator{\Res}{Res}
  \usepackage{multicol}         % for resolution
}{}

\excludecomment{solution}

\begin{document}

\maketitle

%\begin{center}
%\begin{mdframed}
%  \renewcommand{\theexercise}{zur Selbstkontrolle
%  (diese werden in den Übungen nicht besprochen)}
%%  
\begin{exercise}

\begin{enumerate}
\item[S17)] Gegeben ist der folgende NFA $\mathcal{M}_1=(\{q_0,q_1,q_2,q_3\},\{a,b\},\delta,\{q_0\},\{q_3\})$ mit\\[0.2cm] 
$\delta$:
\begin{center}
  
\begin{exercise}

\begin{enumerate}
\item[S17)] Gegeben ist der folgende NFA $\mathcal{M}_1=(\{q_0,q_1,q_2,q_3\},\{a,b\},\delta,\{q_0\},\{q_3\})$ mit\\[0.2cm] 
$\delta$:
\begin{center}
  
\begin{exercise}

\begin{enumerate}
\item[S17)] Gegeben ist der folgende NFA $\mathcal{M}_1=(\{q_0,q_1,q_2,q_3\},\{a,b\},\delta,\{q_0\},\{q_3\})$ mit\\[0.2cm] 
$\delta$:
\begin{center}
  \input{pool/graphics/sprachen-wiederholung}
\end{center}
\begin{enumerate}
    \item[a)] Berechnen Sie mithilfe des {\it{Arden}}-Lemmas einen regul\"aren Ausdruck $\alpha$ mit $L(\mathcal{M}_1)=L(\alpha)$.
    \item[b)] Geben Sie einen DFA $\overline{\mathcal{M}_2}$ an, der das Komplement von $L$ akzeptiert, indem Sie aus ${\mathcal{M}}_1$ einen DFA ${\mathcal{M}}_2$
       f\"ur $L$ und aus ${\mathcal{M}}_2$ anschlie\ss{}end den Komplementautomaten $\overline{\mathcal{M}_2}$ bilden.\\
 \end{enumerate} 
  
\item[S18)]  
\begin{enumerate}
  \item[a)] Gegeben sind die folgenden Grammatiken $G_i$ mit $1\le i\le 4$:
    \begin{itemize}
      \item $G_1=(\{S\},\{a,b\},\{S\rightarrow aS,S\rightarrow Sb,S\rightarrow a\},S)$
      \item $G_2=(\{S\},\{a,b\},\{S\rightarrow aS,S\rightarrow SbS,S\rightarrow a\},S)$
      \item $G_3=(\{S,B\},\{a,b\},\{S\rightarrow \varepsilon,S\rightarrow aSb,aS\rightarrow aB, B\rightarrow bB, B\rightarrow b\},S)$
      \item $G_4=(\{S,A\},\{a,b\},\{S\rightarrow a,A\rightarrow b\},S)$ 
    \end{itemize}
   Geben Sie f\"ur jede Grammatik $G_i$ den maximalen Chomsky-Typ $j$ an. Begr\"unden Sie Ihre Antwort.
   \item[b)] Gegeben sind die folgenden Sprachen $L_i$ mit $1\le i\le 4$:
 \begin{itemize}
      \item $L_1=\{a^nb^na^n \;|\;n\in \mathbb{N}, n\ge 1\}$   
      \item $L_2=\{\varepsilon,a\}$   
      \item $L_3=\{a^nb^m \;|\;n,m\in \mathbb{N}\setminus \{0\}, n> m\}$   
      \item $L_4=L(\{a\}\circ \{a\}^*\circ \{b\}\circ \{b\}^*)\setminus L_3$   
 \end{itemize}
 Geben Sie f\"ur jede Sprache $L_i$ den maximalen Chomsky-Typ $j$ an. Begr\"unden Sie Ihre Antwort. Die Darlegung der Beweisidee ist ausreichend.
 \end{enumerate}
 
 \end{enumerate}
 
 \end{exercise}
 
 
\end{center}
\begin{enumerate}
    \item[a)] Berechnen Sie mithilfe des {\it{Arden}}-Lemmas einen regul\"aren Ausdruck $\alpha$ mit $L(\mathcal{M}_1)=L(\alpha)$.
    \item[b)] Geben Sie einen DFA $\overline{\mathcal{M}_2}$ an, der das Komplement von $L$ akzeptiert, indem Sie aus ${\mathcal{M}}_1$ einen DFA ${\mathcal{M}}_2$
       f\"ur $L$ und aus ${\mathcal{M}}_2$ anschlie\ss{}end den Komplementautomaten $\overline{\mathcal{M}_2}$ bilden.\\
 \end{enumerate} 
  
\item[S18)]  
\begin{enumerate}
  \item[a)] Gegeben sind die folgenden Grammatiken $G_i$ mit $1\le i\le 4$:
    \begin{itemize}
      \item $G_1=(\{S\},\{a,b\},\{S\rightarrow aS,S\rightarrow Sb,S\rightarrow a\},S)$
      \item $G_2=(\{S\},\{a,b\},\{S\rightarrow aS,S\rightarrow SbS,S\rightarrow a\},S)$
      \item $G_3=(\{S,B\},\{a,b\},\{S\rightarrow \varepsilon,S\rightarrow aSb,aS\rightarrow aB, B\rightarrow bB, B\rightarrow b\},S)$
      \item $G_4=(\{S,A\},\{a,b\},\{S\rightarrow a,A\rightarrow b\},S)$ 
    \end{itemize}
   Geben Sie f\"ur jede Grammatik $G_i$ den maximalen Chomsky-Typ $j$ an. Begr\"unden Sie Ihre Antwort.
   \item[b)] Gegeben sind die folgenden Sprachen $L_i$ mit $1\le i\le 4$:
 \begin{itemize}
      \item $L_1=\{a^nb^na^n \;|\;n\in \mathbb{N}, n\ge 1\}$   
      \item $L_2=\{\varepsilon,a\}$   
      \item $L_3=\{a^nb^m \;|\;n,m\in \mathbb{N}\setminus \{0\}, n> m\}$   
      \item $L_4=L(\{a\}\circ \{a\}^*\circ \{b\}\circ \{b\}^*)\setminus L_3$   
 \end{itemize}
 Geben Sie f\"ur jede Sprache $L_i$ den maximalen Chomsky-Typ $j$ an. Begr\"unden Sie Ihre Antwort. Die Darlegung der Beweisidee ist ausreichend.
 \end{enumerate}
 
 \end{enumerate}
 
 \end{exercise}
 
 
\end{center}
\begin{enumerate}
    \item[a)] Berechnen Sie mithilfe des {\it{Arden}}-Lemmas einen regul\"aren Ausdruck $\alpha$ mit $L(\mathcal{M}_1)=L(\alpha)$.
    \item[b)] Geben Sie einen DFA $\overline{\mathcal{M}_2}$ an, der das Komplement von $L$ akzeptiert, indem Sie aus ${\mathcal{M}}_1$ einen DFA ${\mathcal{M}}_2$
       f\"ur $L$ und aus ${\mathcal{M}}_2$ anschlie\ss{}end den Komplementautomaten $\overline{\mathcal{M}_2}$ bilden.\\
 \end{enumerate} 
  
\item[S18)]  
\begin{enumerate}
  \item[a)] Gegeben sind die folgenden Grammatiken $G_i$ mit $1\le i\le 4$:
    \begin{itemize}
      \item $G_1=(\{S\},\{a,b\},\{S\rightarrow aS,S\rightarrow Sb,S\rightarrow a\},S)$
      \item $G_2=(\{S\},\{a,b\},\{S\rightarrow aS,S\rightarrow SbS,S\rightarrow a\},S)$
      \item $G_3=(\{S,B\},\{a,b\},\{S\rightarrow \varepsilon,S\rightarrow aSb,aS\rightarrow aB, B\rightarrow bB, B\rightarrow b\},S)$
      \item $G_4=(\{S,A\},\{a,b\},\{S\rightarrow a,A\rightarrow b\},S)$ 
    \end{itemize}
   Geben Sie f\"ur jede Grammatik $G_i$ den maximalen Chomsky-Typ $j$ an. Begr\"unden Sie Ihre Antwort.
   \item[b)] Gegeben sind die folgenden Sprachen $L_i$ mit $1\le i\le 4$:
 \begin{itemize}
      \item $L_1=\{a^nb^na^n \;|\;n\in \mathbb{N}, n\ge 1\}$   
      \item $L_2=\{\varepsilon,a\}$   
      \item $L_3=\{a^nb^m \;|\;n,m\in \mathbb{N}\setminus \{0\}, n> m\}$   
      \item $L_4=L(\{a\}\circ \{a\}^*\circ \{b\}\circ \{b\}^*)\setminus L_3$   
 \end{itemize}
 Geben Sie f\"ur jede Sprache $L_i$ den maximalen Chomsky-Typ $j$ an. Begr\"unden Sie Ihre Antwort. Die Darlegung der Beweisidee ist ausreichend.
 \end{enumerate}
 
 \end{enumerate}
 
 \end{exercise}
 
 
%%  {\bfseries Hinweis:} Die Aufgaben *) und **)
%%  dienen der Selbstkontrolle und werden in der
%%  Übung nicht besprochen.
%\end{mdframed}
%\end{center}

%\vspace*{0.5cm}
%{\bf{Anmerkung}}\\
%Mit der 9. \"Ubung ist in den verschiedenen \"Ubungsgruppen abzusichern, dass alle bisher aus Zeitgr\"unden noch nicht besprochenen Aufgaben der \"Ubungsbl\"atter 1 bis 8 abgearbeitet sind.

\setcounter{exercise}{0}

% diese aufgabe hat sich erledigt - wurde in vl komplett gezeigt:
%\input{pool/sprachen-turingmaschine-zweierpotenz}

% Zur Widerholung beim nächsten Mal...
%\input{pool/sprachen-turingmaschine-ai_bi_ci}

\begin{exercise}

Gegeben ist die nichtdeterministische $3$-Band Turingmaschine
 $$\T \ = \ (\{q_0,q_1,q_2\}, \{a,b\}, \{a,b,\Box\}, \delta,q_0,
\Box,\{q_2\})
$$
mit
$$\begin{array}{lcl}
  \delta(q_0, a,\Box,\Box) & = & \{ (q_0, \langle a,R\rangle, \langle  a,R
  \rangle ,\langle  \Box,N  \rangle ), \\
  &  &  (q_1, \langle a,R\rangle, \langle  a,N
  \rangle ,\langle  \Box,L  \rangle )   \} \\
  
  \delta(q_0, b,\Box,\Box) & = & \{ (q_0, \langle b,R\rangle, \langle  \Box,N
  \rangle ,\langle  b,R  \rangle ), \\
  &  &  (q_1, \langle b,R\rangle, \langle  \Box,L
  \rangle ,\langle  b,N  \rangle )    \} \\

  \delta(q_1, a,a,b) & = & \{ (q_1, \langle a,R\rangle, \langle  \Box,L
  \rangle ,\langle  b,N  \rangle )    \} \\

  \delta(q_1, a,a,\Box) & = & \{ (q_1, \langle a,R\rangle, \langle  \Box,L
  \rangle ,\langle  \Box,N  \rangle )    \} \\

  \delta(q_1, b,a,b) & = & \{ (q_1, \langle b,R\rangle, \langle  a,N
  \rangle ,\langle  \Box,L  \rangle )    \} \\

  \delta(q_1, b,\Box,b) & = & \{ (q_1, \langle b,R\rangle, \langle  \Box,N
  \rangle ,\langle  \Box,L  \rangle )    \} \\

  \delta(q_1, \Box,\Box,\Box) & = & \{ (q_2, \langle \Box,N\rangle, \langle  \Box,N
  \rangle ,\langle  \Box,N  \rangle )    \} \\

\end{array}
$$

Welche Sprache akzeptiert $\T$ ? 
Hinweis: Sie können die Arbeitsweise von
$\T$ gerne mithilfe einer graphischen 
Repräsentation der Übergangsrelation~$\delta$ 
nachvollziehen.

\end{exercise}

\begin{exercise}
	
Geben Sie pr\"azise eine deterministische Turingmaschine $\T$ zur Erkennung der Sprache 
$L \ = \  \bigl\{ a^nb^mc^k: n,m,k \geq 1, n=2m  \ \text{oder} \ m = k\}$ an.
Sie k\"onnen wahlweise eine Ein- oder Mehrband-DTM verwenden.

\begin{itemize}
\item [(a)]

Begr\"unden Sie, warum ${\mathcal L}(\T) = L$. 

\item [(b)]
Geben Sie die Berechnungen f\"ur $w_1 = abcc$ und $w_2 = aabc$ an.
\end{itemize}
\end{exercise}



\begin{exercise}
Wie in der Vorlesung dargelegt wurde, werden Turingmaschinen als allgemeines Rechenmodell verstanden (18.\,Vorlesung, Folie 19).\\[0.2cm]
Geben Sie Turingmaschinen an, die folgende Funktionen 
berechnen. Dabei wird eine Eingabe $n\in \mathbb N$ als $\emptyset^n$ mit $\emptyset\in \Sigma$ dargestellt. Es kann vorausgesetzt werden, dass die Eingabe wohlgeformt auf dem Band vorliegt. Am Ende der Berechnung h\"alt die Turingmaschine in einem Finalzustand und das Band enth\"alt nur das Berechnungsergebnis. 
\begin{enumerate}
\item[a)] Die Turingmaschine $\mathcal M_{0}$ berechnet die
  Funktion $f:\mathbb N\rightarrow \mathbb N,\,n\mapsto 0$, d.\,h. das Eingabewort auf dem
  Band wird gelöscht.
\item[b)] Die Turingmaschine $\mathcal M_{succ}$ berechnet die
  Funktion $f:\mathbb N\rightarrow \mathbb N,\,n\mapsto n+1$.
\item[c)] Für $i,n\in \mathbb N$ berechnet die Turingmaschine $\mathcal M_{n}^i$
  die Funktion $f^i_n:\mathbb N^n\rightarrow \mathbb
  N$ mit $(x_1,\ldots,x_n)\mapsto x_i$. Es wird empfohlen, zunächst die
  Turingmaschine $\mathcal M_{4}^2$ anzugeben und diese dann zu
  $\mathcal M_{n}^i$ zu verallgemeinern.\\[1ex]
  \textit{Hinweis}: $(3,2,4,0)$ in der Eingabe wird dargestellt als \mbox{$(\emptyset\emptyset\emptyset,\emptyset\emptyset,\emptyset\emptyset\emptyset\emptyset,)\textvisiblespace$ }. 
\end{enumerate} 
\end{exercise}



%%%%%%%%%%%%%%%%%%%%%%%%%%%

% zu komplex für den moment
%\input{pool/baier/14b}

%%%%%%%%%%%%%%%%%%%%%%%%%%%

\end{document}
